\section{Modulator}
\subsection{Frequency response}

Second-order system with unit damping:
\begin{equation}
	H_{mod}(f) =  \frac{1}{1 + 2jf/f_c - (f/f_c)^2}
\end{equation}

The modulator bandwidth is related to $f_c$ by
\begin{equation}
f_{3dB} = \sqrt{\sqrt{2}-1}f_c = 0.64359f_c
\end{equation}

Group delay:
\begin{equation}
	\Delta\tau_g = \frac{2}{2\pi f_c}
\end{equation}


\subsection{Extinction ratio}
\begin{equation}
	r_{ex} = \frac{P_{min}}{P_{max}}
\end{equation}

This relation is enforced in the levels of the PAM signal. Thus, if the levels of a M-PAM signal are $\{a_0, \ldots, a_{M-1}\}$ we have

\begin{equation}
	r_{ex} = \frac{a_0}{a_{M-1}}
\end{equation}

In the case of equally-spaced levels, when a target transmitted power is specified $P_{tx}$, we can determine the power of the lowest level
\begin{equation}
	r_{ex} = \frac{P_{min}}{2P_{tx} - P_{min}} \rightarrow P_{min} = 2P_{tx}\frac{r_{ex}}{1 + r_{ex}}
\end{equation}

Thus the scaled levels $\{a'_0, \ldots, a'_{M-1}\}$ such that the signal has power $P_{tx}$ and extinction ratio $r_{ex}$ can be related to the original levels $\{a_0, \ldots, a_{M-1}\}$ by:
\begin{align} \nonumber
	a'_i &= \frac{2P_{tx} - 2P_{min}}{a_{M-1}}a_i + P_{min} = 2\bigg(P_{tx} - 2P_{tx}\frac{r_{ex}}{1 + r_{ex}}\bigg)\frac{a_i}{a_{M-1}} + P_{min} \\
	&= \frac{2P_{tx}}{a_{M-1}}\bigg(1 - 2\frac{r_{ex}}{1 + r_{ex}}\bigg)a_i + P_{min} = \frac{2P_{tx}}{a_{M-1}}\frac{1-r_{ex}}{1 + r_{ex}}a_i + P_{min} 
\end{align}
where $a_0 = 0$.

For optimized level spacing the extinction ratio is enforced in the optimization process, where at the beginning of the $i$th iteration, the lowest level is set to be $a_0^{(i)} = r_{ex}a_{M-1}^{(i-1)}$

\subsection{RIN}
Modeled as an AWG noise added to the optical power at the transmitter.

The value of RIN is defined as the ratio between the noise power divide by the noise bandwidth and the signal power \cite{agilent-RIN-measurement}: 
\begin{equation}
	RIN = \frac{P_{noise}}{B_{noise}P_{signal}}
\end{equation}

Thus the \textbf{one-sided RIN PSD} and \textbf{RIN variance} at a certain instant are given by
\begin{align}
	& S_{RIN}(t) = RIN\cdot P(t)^2 \\
	& \sigma^2_{RIN}(t) = S_{RIN}(t)\frac{f_{s, sim}}{2}
\end{align}
where $f_{s, sim}$ is the sampling frequency to simulate continuous time.

Output optical power $P(t)$ is given by
\begin{equation}
	P(t) = P_s(t) + w_{RIN}(t)
\end{equation}
where $P_s(t)$ is the signal-only optical power (after modulator frequency response and extinction ratio), and $w_{RIN}(t)\sim\mathcal{N}(0, \sigma^2_{RIN}(t))$.

\subsection{Chirp}
Assuming transient chirp dominant, we have the phase change caused by intensity modulation is given by
\begin{equation}
	\Delta\phi(t) = \frac{\alpha}{2}\ln(P(t))
\end{equation}
where $\alpha$ is the chirp parameter, which is always positive for directly-modulated lasers.

Output electric field of the modulator:
\begin{equation}
	E_{out}(t) = \sqrt{P(t)}e^{j\Delta\phi(t)}
\end{equation}

\section{Fiber}
\subsection{Linear propagation}
\begin{equation}
	E(f, L) = E(f, 0)e^{j\theta}e^{-\frac{1}{2}\frac{att(\lambda)L}{10^4}}
\end{equation}
where $E(f, L)$ is the frequency response of the electric field at distance $L$ in meters; $\theta = -1/2\beta_2(2\pi f)^2L$; $\beta_2 = -\frac{D(\lambda)\lambda^2}{2\pi c}$; and $att(\lambda)$ is the fiber attenuation at wavelength $\lambda$ in dB/km.

For SMF28 the fiber dispersion is specified in terms of the zero-dispersion ($\lambda_0$) wavelength and the dispersion slope ($S_0$):
\begin{equation}
	D(\lambda) = \frac{S_0}{4}\bigg(\lambda - \frac{\lambda_0^4}{\lambda^3}\bigg), \text{for}~1200~\text{nm} < \lambda < 1600~\text{nm}
\end{equation}

\subsection{Small-signal fiber frequency response}
Assuming transient chirp dominant, chromatic dispersion, and attenuation, the small-signal fiber frequency response is given by
\begin{equation}
	H(f) = \frac{P(f, L)}{P(f, 0)} = \Big(\cos(\theta) - \alpha\sin(\theta)\Big)e^{-\frac{att(\lambda)L}{10^4}}
\end{equation}


\section{Using Saddlepoint approximation in the inversion formula of a Moment Generating Function} 

The moment generating function is defined as
\begin{align}
M(s) = E[e^{sx}] &= \begin{cases}
\int_{-\infty}^{\infty} p(x)e^{sx}dx, & \text{continuos} \\
\sum p(x)e^{sx}, & \text{discrete}
\end{cases}
\end{align}
for both continuous and discrete random variables.

The inversion formula is given by
\begin{equation}
p(x) = \frac{1}{2\pi j}\int_{c-j\infty}^{c+j\infty}M(s)e^{-sx}ds= \frac{1}{2\pi j}\int_{c-j\infty}^{c+j\infty}e^{K(s, x)}ds
\end{equation}
Where $K(s,x) = \log M(s) - sx$.

The contour of integration is the straight line in the $s$ plane such that $\mathrm{Re}\{s\} = c$. 

$c$ should be within the interval of convergence for real $s$.

In the saddlepoint approximation $c = \mathrm{Re}\{\hat{s}\}$, where $\frac{\partial}{\partial s} K(s, x)|_{s=\hat{s}} = 0$. Thus, $c$ is the real part of the saddlepoint (of the exponent of the inverse formula). Therefore, $s = \hat{s} + j\omega$. Expanding $K(s,x)$ near $\hat{s}$

\begin{align} \nonumber
K(s, x) &= K(\hat{s}, x) + \frac{(s-\hat{s})^2}{2!}\frac{\partial^2}{\partial s^2}K(s,x)\bigg|_{s=\hat{s}} \\
&= K(\hat{s}, x)  -\frac{\omega^2}{2}K^{\prime\prime}(\hat{s},x)
\end{align}

Applying the saddlepoint approximation in the inversion formula results

\begin{align} \nonumber
p(x) = \frac{e^{K(\hat{s}, x)}}{2\pi }\int_{-\infty}^{\infty}e^{  -\frac{\omega^2}{2}K^{\prime\prime}(\hat{s},x)}d\omega
\approx e^{K(\hat{s},x)}\bigg(\frac{1}{2\pi K^{\prime\prime}(\hat{s},x)}\bigg)^{1/2}
\end{align}
where $K^{\prime\prime}(s,x) = \frac{\partial^2}{\partial s} K(s,x)$.

\subsection{Calculating tail probabilities}
\begin{align}
Q(x) &= P(X \geq x) \\
&= \int_x^{\infty} p(y)dy \\ \nonumber
&= \int_x^{\infty}\frac{1}{2\pi j}\int_{c-j\infty}^{c+j\infty}e^{\ln M(s) - sy}dsdy \\ \nonumber
&= \frac{1}{2\pi j}\int_{c-j\infty}^{c+j\infty}\int_x^{\infty}e^{\ln M(s) - sy}dyds \\ \nonumber
&= \frac{1}{2\pi j}\int_{c-j\infty}^{c+j\infty}\frac{1}{s}e^{\ln M(s) - sx}ds \\ 
&= \frac{1}{2\pi j}\int_{c-j\infty}^{c+j\infty}e^{\ln M(s) -\ln s- sx}ds 
\end{align}

We can define $K(s,x) = \ln\frac{M(s)}{s} - sx$. Applying the saddlepoint approximation with $K(s)$ results in

\begin{equation}
Q(x) \approx e^{K(\hat{s},x)}\bigg(\frac{1}{2\pi K^{\prime\prime}(\hat{s},x)}\bigg)^{1/2}
\end{equation}
where in this case $K^{\prime}(\hat{s},x) = 0$.
Using the saddlepoint approximation in this form might lead to inaccurate results because of the pole at $s = 0$ (i.e., $\hat{s}$ small).