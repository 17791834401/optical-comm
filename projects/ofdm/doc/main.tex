\documentclass[a4paper]{article}

\usepackage[english]{babel}
\usepackage[utf8]{inputenc}
\usepackage{amsmath}
\usepackage{amsfonts}
\usepackage{graphicx}
\usepackage{subcaption}
\usepackage[numbered]{bookmark}
\usepackage[colorinlistoftodos]{todonotes}
\usepackage{algorithm}
\usepackage{algpseudocode}
\usepackage{pifont}
\usepackage{tikz}
\usepackage{pgfplots}
\usepackage{bm}
\usepackage{placeins}
\usetikzlibrary{arrows}
%\usetikzlibrary{external}\tikzexternalize[prefix=figs/]

\DeclareGraphicsExtensions{.eps,.pdf,.png}
\graphicspath{{figs/}}

\pgfmathdeclarefunction{gauss}{2}{%
	\pgfmathparse{1/(#2*sqrt(2*pi))*exp(-((x-#1)^2)/(2*#2^2))}%
}

% Expectation 
\DeclareMathOperator{\E}{\mathbb{E}} 
	
% Convolution 
\newcommand{\Conv}{\mathop{\scalebox{1.5}{\raisebox{-0.2ex}{$\ast$}}}}%

\title{OFDM}

\author{JKP}

\date{\today}

\begin{document}
\maketitle

\section{Transmitter}
\subsection{Intensity Noise}
Intensity noise is modeled as an additive white Gaussian noise (AWGN) added to the optical power at the transmitter.

The value of the relative intensity noise (RIN) is defined as the ratio between the noise power divided by the noise bandwidth and the signal power \cite{agilent-RIN-measurement}: 
\begin{equation}
RIN = \frac{P_{noise}}{B_{noise}P_{signal}}
\end{equation}

Hence, the \textbf{one-sided RIN PSD} and \textbf{RIN variance} at time $t$ are given by
\begin{align}
& S_{RIN}(t) = RIN\cdot P(t)^2 \\
& \sigma^2_{RIN}(t) = S_{RIN}(t)\frac{f_{s, sim}}{2}
\end{align}
where $f_{s, sim}$ is the sampling frequency to simulate continuous time. Obviously, the variance as defined here only make sense in simulations. Since the intensity noise is assumed to be white, it'd have infinite variance.

Output optical power $P(t)$ is given by
\begin{equation}
P(t) = P_s(t) + w_{RIN}(t)
\end{equation}
where $P_s(t)$ is the signal-only optical power (after modulator frequency response and extinction ratio), and $w_{RIN}(t)\sim\mathcal{N}(0, \sigma^2_{RIN}(t))$.

\subsection{Modulator Bandwidth Limitations}

\subsubsection{Mach-Zenhder modulator}
\cite{Barros2009, Ho2005}

Limited by loss and velocity mismatch

\begin{equation}
H_{mod}(f) = \frac{1-e^{-\alpha(f)L+j2\pi fd_{12}L}}{\alpha(f)L-j2\pi fd_{12}L}
\end{equation}
where $\alpha(f)$ is the frequency-dependent loss, $d_{12}$ is the velocity mismatch between the optical and electrical waveguides, and $L$ is the interaction length.

\begin{equation}
d_{12} = \frac{n_m-n_r}{c}
\end{equation}
where $n_r \approx 2.15$ is the refractive index of the coplanar waveguide for TM input light. If $n_m$ is only $95\%$ of $n_r$ a significant reduction in bandwidth may occur due to velocity mismatch \cite{Ho2005}.

\subsubsection{Modulator limited by parasitics}
This modulator is modeled as a critically damped second-order system. This is based on the assumption that parasitics capacitances and inductances are the limiting factor in the bandwidth of these devices.

Second-order system with unit damping:
\begin{equation}
H_{mod}(f) =  \frac{1}{1 + 2jf/f_c - (f/f_c)^2}
\end{equation}

The modulator bandwidth is related to $f_c$ by
\begin{equation}
f_{3dB} = \sqrt{\sqrt{2}-1}f_c = 0.64359f_c
\end{equation}

Group delay:
\begin{equation}
\Delta\tau_g = \frac{2}{2\pi f_c}
\end{equation}

\subsection{MZM modulator model}

The MZM modulator realized by the function \texttt{f/mzm.m} operates as an intensity modulator or a IQ modulator depending on whether the driving signal is real or complex, respectively. 

If the driving signal is real, the modulator will generate the following output:
\begin{equation}
E_{out}(t) = E_{in}(t)\sin\Big(\frac{\pi}{2}V_{in}(t)\Big)
\end{equation}
Hence, it assumes that the driving signal $V_{in}(t)$ is normalized by $V_{\pi}/2$.

If the driving signal is imaginary, the modulator will behave as an I-Q modulator:
\begin{equation}
E_{out}(t) = E_{in}(t)\Big[\sin\Big(\frac{\pi}{2}\mathrm{Re}\{V_{in}(t)\}\Big) + j\sin\Big(\frac{\pi}{2}\mathrm{Im}\{V_{in}(t)\}\Big)\Big]
\end{equation}

$E_{out}(t)$ is then normalized so that $E(|E_{out}(t)|^2) = E(|E_{in}(t)|^2)$.

\FloatBarrier
\begin{figure}[h!]
	\centering
	\resizebox{\linewidth}{!}{\def\Vpi{1}
\begin{tikzpicture}
\begin{axis}[domain=-\Vpi/2:\Vpi/2, samples=100, no markers,
axis lines*=center,
legend pos=south east,
xtick={-0.5, 0.5},
xticklabels={$-\frac{V_{\pi}}{2}$, $\frac{V_{\pi}}{2}$},
ytick={-1, 1},
xlabel={$V_{in}$}
]
	\addplot [very thick,black!50] {sin(deg(pi*x/\Vpi))}; 
	\addplot [very thick,black] {(sin(deg(pi*x/\Vpi)))^2};
	\legend{$\frac{E_{out}}{E_{in}}$, $\big|\frac{E_{out}}{E_{in}}\big|^2$}
\end{axis}
\end{tikzpicture}
}
	\caption{MZM modulator transfer function.}
\end{figure}
\FloatBarrier

\section{Fiber Propagation}
\subsection{Chromatic Dispersion}
\begin{equation} \label{eq:Hdisp}
H(f; L) = \frac{E(f; L)}{E(f, 0)} = \exp\Big(-1/2j\beta_2(2\pi f)^2L\Big)
\end{equation}
where $H(f; L)$ is fiber frequency response due to dispersion after $L$ meters, and $\beta_2 = -\frac{D(\lambda)\lambda^2}{2\pi c}$. 

Fiber attenuation can be included with the factor $e^{-\frac{1}{2}\frac{att(\lambda)L}{10^4}}$, where $att(\lambda)$ is the fiber attenuation at wavelength $\lambda$ in dB/km.

For SMF28 the fiber dispersion is specified in terms of the zero-dispersion ($\lambda_0$) wavelength and the dispersion slope ($S_0$):
\begin{equation}
D(\lambda) = \frac{S_0}{4}\bigg(\lambda - \frac{\lambda_0^4}{\lambda^3}\bigg), 1200~\text{nm} < \lambda < 1600~\text{nm}
\end{equation}
where $\lambda_0 = 1310$ nm and $S_0 = 0.092$ ps/nm.

\section{OFDM}

\subsection{Symbol Rate and Sampling Rate}

We define an OFDM signal with $N_c$ orthogonal subcarriers, from which $N_c - N_u$ subcarriers are used for oversampling, leaving $N_u$ data-bearing subcarriers. Hence, the oversampling rate is defined as

\begin{equation}
r_{os} = \frac{N_c}{N_u}
\end{equation}

Different variations of OFDM may not use all the $N_u$ subcarriers available. A few examples are the case of when Hermitian symmetry is enforced to produce a real signal, or when half of the subcarriers are set to zero to produce a single-side band (SSB) signal, or when all the even subcarriers are not modulated to allow asymmetric clipping in asymmetrically clipped (ACO)-OFDM. This loss in spectral efficiency will be characterized by a factor $p$. Note that the definition of $N_u$ is different from the one used in the ``Multicarrier'' paper, where $N_u$ denoted the number of ``used'' subcarriers i.e., subcarriers not set to zero. 

Given a certain bit rate $R_b$, the spacing between each subcarrier is given by
\begin{equation} \label{eq:ofdm-subcarrier-spacing}
\Delta f = p\frac{R_b}{N_u\log_2 CS},
\end{equation}
where $CS$ is the nominal or average constellation size, and $p$ is the factor that accounts for the loss of spectral efficiency. For instance, $p = 2$ when Hermitian symmetry is enforced or when the negative subcarriers are set to zero in the case of SSB-OFDM. In the case of ACO-OFDM, $p = 4$ since in addition to Hermitian symmetry, all the even subcarriers are set to zero. 

Fig.~\ref{fig:ofdm-spectra} illustrates the OFDM spectra for DC- and ACO-OFDM.

\FloatBarrier
\begin{figure}[h!]
	\centering
	\begin{subfigure}[h!]{\textwidth}
		\centering
		\resizebox{\linewidth}{!}{%% DC-OFDM spectrum
\begin{tikzpicture} 
\begin{axis}[
width=4.52in,
height=3.54in,
  no markers, domain = 0:520, samples = 400,
  xlabel=$f$,
  every axis y label/.style={at=(current axis.above origin),anchor=south},
  every axis x label/.style={at=(current axis.right of origin),anchor=west},
  height=5cm, width=7cm,
  ymax=1.5,
  xtick={360}, 
  xticklabels={$\frac{R_b}{\log_2M}$},
  ytick=\empty,
  enlargelimits=false, clip=false, axis on top,
  grid = major
  ]
  	\foreach \i in {0, 1, 2, 3, 4, 5, 6, 8, 9}
    {
    \addplot [domain = 0.1:520, samples = 400]
            	{abs(((\i/15)^2+1)*sin(5*(x-\i*36))/ ((x-\i*36)*pi/36)};  
    }
    %\addplot [very thick, color=blue, domain = 0.1:324, samples = 400]
%     	{1};  
    %\addplot [very thick, color=blue, domain = 324.1:360, samples = 400]
%     	{abs(sin(5*(x-9*36))/ ((x-9*36)*pi/36)};
    
    \draw[-] (250,100) -- (250,100) node[above] {$\ldots$};
    
%     \addplot [very thick, color = blue, domain = 0.1:520, samples = 400]
%     	{abs(sin(5*(x-9*36))/ ((x-9*36)*pi/36) 
%         + sin(5*(x-8*36))/ ((x-8*36)*pi/36) 
%         + sin(5*(x-7*36))/ ((x-7*36)*pi/36) 
%         + sin(5*(x-6*36))/ ((x-6*36)*pi/36)
%         + sin(5*(x-5*36))/ ((x-5*36)*pi/36)
%         + sin(5*(x-4*36))/ ((x-4*36)*pi/36)
%         + sin(5*(x-3*36))/ ((x-3*36)*pi/36)
%         + sin(5*(x-2*36))/ ((x-2*36)*pi/36)
%         + sin(5*(x-1*36))/ ((x-1*36)*pi/36)
%         + sin(5*(x-0*36))/ ((x-0*36)*pi/36)
%         + sin(5*(x+1*36))/ ((x+1*36)*pi/36)
%         + sin(5*(x+2*36))/ ((x+2*36)*pi/36)
%         + sin(5*(x+3*36))/ ((x+3*36)*pi/36))};
\end{axis}
\end{tikzpicture}
}
		\caption{DC-OFDM}
	\end{subfigure}%
	
	\begin{subfigure}[h!]{\textwidth}
		\centering
		\resizebox{\linewidth}{!}{%% ACO-OFDM
\begin{tikzpicture} 
\begin{axis}[
width=4.52in,
height=3.54in,
  no markers, domain = 0:520, samples = 400,
  xlabel=$f$,
  every axis y label/.style={at=(current axis.above origin),anchor=south},
  every axis x label/.style={at=(current axis.right of origin),anchor=west},
  height=5cm, width=7cm,
  ymax=1.5,
  xtick={360}, 
  xticklabels={$2\frac{R_b}{\log_2M}$},
  ytick=\empty,
  enlargelimits=false, clip=false, axis on top,
  grid = major
  ]
  	\foreach \i in {1, 3, 5, 7, 9}
    {
    \addplot [domain = 0.1:520, samples = 400]
            	{abs(((\i/15)^2+1)*sin(5*(x-\i*36))/ ((x-\i*36)*pi/36)};  
    }
    %\addplot [very thick, color=blue, domain = 0.1:324, samples = 400]
%     	{1};  
    %\addplot [very thick, color=blue, domain = 324.1:360, samples = 400]
%     	{abs(sin(5*(x-9*36))/ ((x-9*36)*pi/36)};
    
    \draw[-] (287,100) -- (287,100) node[above] {$\ldots$};
    
    %\addplot [very thick, color = blue, domain = 0.1:520, samples = 400]
%     	{abs(sin(5*(x-9*36))/ ((x-9*36)*pi/36) 
%         + sin(5*(x-7*36))/ ((x-7*36)*pi/36) 
%         + sin(5*(x-5*36))/ ((x-5*36)*pi/36)
%         + sin(5*(x-3*36))/ ((x-3*36)*pi/36)
%         + sin(5*(x-1*36))/ ((x-1*36)*pi/36)
%         + sin(5*(x-0*36))/ ((x-0*36)*pi/36)
%         + sin(5*(x+1*36))/ ((x+1*36)*pi/36)
%         + sin(5*(x+3*36))/ ((x+3*36)*pi/36))};
\end{axis}
\end{tikzpicture}
}
		\caption{ACO-OFDM}
	\end{subfigure}
	\caption{(a) DC-OFDM and (b) ACO-OFDM spectra.} \label{fig:ofdm-spectra}
\end{figure}
\FloatBarrier

Relating the OFDM subcarrier spacing from \eqref{eq:ofdm-subcarrier-spacing} to the sampling time $T_s^{\prime}$ yields
\begin{equation}
T_s^{\prime} = \frac{1}{N_c\Delta f},
\end{equation}

In practice, the sampling rate must be increased to account for the insertion of a cyclic prefix of length $N_{CP}$, hence the sampling time is reduce
\begin{equation}
T = \frac{N_c}{N_c + N_{CP}}T^{\prime} = \frac{1}{(N_c + N_{CP})\Delta f}.
\end{equation}

The increased OFDM sampling rate is given by
\begin{align} \nonumber
f_s &= (N_c + N_{CP})\Delta f = p\frac{N_c + N_{CP}}{N_u}\frac{R_b}{\log_2 CS} \\
&= p\frac{N_c + N_{CP}}{N_c}r_{os}\frac{R_b}{\log_2 CS}.
\end{align}
The factors $p$, $\frac{N_c + N_{CP}}{N_c}$, and $r_{os}$ account for the penalty due to loss in spectral efficiency, cyclic prefix, and oversampling, respectively. 

\subsection{Gaussian approximation}

By the central limit theorem, an OFDM signal with large enough number of subcarriers is approximately Gaussian distributed with variance given by
\begin{equation}
\sigma^2 = \sum_{N_u} P_n = 2\sum_{n=1}^{N_u/2} P_n,
\end{equation}
where the second equality assumes Hermitian symmetry, hence $P_n = P_{-n}$. 

\subsection{Extinction ratio}
\subsubsection{DC-OFDM}
\begin{align} \nonumber
& P_{pen} = \frac{r\sigma + \Delta}{r\sigma} = 1 + \frac{\Delta}{r\sigma} \\
&r_{ex} = \frac{P_{max}}{P_{min}} = \frac{2r\sigma + \Delta}{\Delta} \\ \nonumber
&\frac{\Delta}{r\sigma} = \frac{2}{r_{ex}-1} \\ 
& P_{pen} = 1 + \frac{\Delta}{r\sigma} = 1 + \frac{2}{r_{ex}-1} = \frac{r_{ex}+1}{r_{ex}-1}
\end{align}

\subsubsection{ACO-OFDM}
\begin{align} \nonumber
& P_{pen} = \frac{\frac{\sigma}{\sqrt{2\pi}} + \Delta}{\frac{\sigma}{\sqrt{2\pi}}} = 1 + \sqrt{2\pi}\frac{\Delta}{\sigma} \\
&r_{ex} = \frac{P_{max}}{P_{min}} = \frac{r\sigma + \Delta}{\Delta} \\ \nonumber
&\frac{\Delta}{\sigma} = \frac{r}{r_{ex}-1} \\ 
& P_{pen} = 1 + \sqrt{2\pi}\frac{\Delta}{\sigma} = 1 + \sqrt{2\pi}\frac{r}{r_{ex}-1} = \frac{r_{ex} + (r\sqrt{2\pi} - 1)}{r_{ex} - 1}
\end{align}

Comparing to DC-OFDM, the penalty due to finite extinction rate will be higher for ACO-OFDM when $r\sqrt{2\pi}-1 > 1 \to r > \frac{2}{\sqrt{2\pi}} \approx 0.8$.

\FloatBarrier
\begin{figure}[h!]
	\centering
	\begin{subfigure}[h!]{\textwidth}
		\centering
		\resizebox{\linewidth}{!}{% This file was created by matlab2tikz v0.4.7 (commit e48c22d065f6aed2b831ddad09bcb59bf008f39b) running on MATLAB 8.5.
% Copyright (c) 2008--2014, Nico Schlömer <nico.schloemer@gmail.com>
% All rights reserved.
% Minimal pgfplots version: 1.3
% 
% The latest updates can be retrieved from
%   http://www.mathworks.com/matlabcentral/fileexchange/22022-matlab2tikz
% where you can also make suggestions and rate matlab2tikz.
% 

\begin{tikzpicture}
\begin{axis}[%
width=4.52in,
height=3.54in,
xlabel=\Huge Time,
scale only axis,
%separate axis lines,
% every outer x axis line/.append style={white!15!black},
% every x tick label/.append style={font=\color{white!15!black}},
xmin=1,
xmax=1050,
every axis y label/.style={at=(current axis.above origin),anchor=south},
%every axis x label/.style={at=(current axis.right of origin),anchor=west},
ymin=-0.5,
ymax=2.5,
ytick={0.01, 1, 2}, 
yticklabels={\Huge 0, \Huge $\bar{P} = r\sigma$, \Huge $P_{max}$},
xtick={1050},
xticklabels={\Huge $T_s$},
grid=major,
]
\addplot [line width = 1.5pt, color=black,dotted]
  table[row sep=crcr]{%
1	-0.183884422632121\\
2	-0.177409190240241\\
3	-0.155660917942332\\
4	-0.119873542586939\\
5	-0.0720396649911914\\
6	-0.0147000215112703\\
7	0.0492956121987101\\
8	0.117023608847397\\
9	0.185696302798263\\
10	0.252877605464125\\
11	0.316792713757472\\
12	0.376697590064492\\
13	0.433057998396404\\
14	0.487355912395488\\
15	0.54162548629292\\
16	0.597936880026256\\
17	0.657973161458312\\
18	0.722759280595411\\
19	0.792549685752308\\
20	0.866851145906614\\
21	0.94454378946453\\
22	1.0240603661914\\
23	1.1035880405761\\
24	1.18126087530027\\
25	1.2552670357485\\
26	1.32374421223811\\
27	1.38449911526351\\
28	1.43485759178368\\
29	1.47187814493774\\
30	1.49282480384062\\
31	1.49565747743151\\
32	1.47938488271363\\
33	1.44422632603823\\
34	1.3915883313773\\
35	1.32389450379758\\
36	1.24431976058135\\
37	1.15648030455127\\
38	1.06412327585316\\
39	0.970850322413793\\
40	0.879915140950151\\
41	0.794143756497055\\
42	0.715965520290821\\
43	0.647447660719481\\
44	0.590247651988647\\
45	0.545508362564503\\
46	0.513766593000571\\
47	0.494917305948398\\
48	0.488243442742059\\
49	0.492502996478297\\
50	0.506056000473078\\
51	0.527011580330895\\
52	0.553376628836385\\
53	0.583191530649053\\
54	0.614641675051536\\
55	0.646118746334397\\
56	0.676188204572014\\
57	0.703479412761343\\
58	0.72661425720354\\
59	0.744262906237915\\
60	0.755291483117682\\
61	0.758915409874958\\
62	0.754803514211788\\
63	0.743114550180409\\
64	0.724469163497451\\
65	0.699871787136762\\
66	0.67060123080954\\
67	0.638088654454646\\
68	0.603798321021745\\
69	0.569125403572982\\
70	0.535366421660618\\
71	0.503864907559111\\
72	0.476297175114895\\
73	0.454830957362793\\
74	0.441955513695004\\
75	0.440075035538006\\
76	0.451077380777375\\
77	0.476015110325821\\
78	0.514948069962139\\
79	0.566944714899216\\
80	0.630210893350422\\
81	0.702303215152261\\
82	0.780383565909171\\
83	0.861476808353902\\
84	0.94270510825479\\
85	1.02151907238326\\
86	1.09599633152589\\
87	1.16517735055575\\
88	1.22922857968642\\
89	1.28928139283012\\
90	1.34703484936613\\
91	1.40430815096717\\
92	1.46266728026926\\
93	1.52317764149554\\
94	1.58629010724054\\
95	1.6518422236698\\
96	1.71914447977244\\
97	1.7871186021022\\
98	1.85445806211491\\
99	1.91978406281908\\
100	1.98173509943461\\
101	2.03888796004497\\
102	2.08953814103039\\
103	2.13158442270567\\
104	2.16270352866978\\
105	2.18073148208424\\
106	2.18405813986075\\
107	2.17191103773384\\
108	2.14448567592514\\
109	2.10292718338616\\
110	2.04919419669813\\
111	1.98584602745186\\
112	1.91579422444864\\
113	1.84205417952196\\
114	1.76752122993451\\
115	1.69474766475572\\
116	1.62564512991698\\
117	1.56114340786105\\
118	1.50102543291454\\
119	1.44409777040994\\
120	1.38860508574431\\
121	1.33269477743854\\
122	1.27480227387035\\
123	1.21390354758196\\
124	1.14962780101086\\
125	1.0822499899122\\
126	1.01259509334775\\
127	0.941888964267579\\
128	0.871587079941896\\
129	0.803209158549565\\
130	0.738244122778487\\
131	0.678231702973314\\
132	0.624991378771421\\
133	0.580743521696281\\
134	0.547928862624569\\
135	0.528812950287981\\
136	0.525077007356962\\
137	0.537524072052994\\
138	0.565944966813593\\
139	0.609138886535052\\
140	0.665056417696476\\
141	0.731022245070482\\
142	0.803994562812215\\
143	0.880824732157752\\
144	0.958486947118338\\
145	1.03421359385035\\
146	1.10543204588607\\
147	1.16953549132192\\
148	1.22374691499668\\
149	1.26527429763801\\
150	1.29167437171015\\
151	1.30122707785151\\
152	1.29319477028059\\
153	1.26792386575134\\
154	1.22679570222929\\
155	1.17205968550569\\
156	1.10659188110261\\
157	1.033621810548\\
158	0.956464084608389\\
159	0.878280292787706\\
160	0.801856787357711\\
161	0.729341497643846\\
162	0.66196326339363\\
163	0.599901741407388\\
164	0.542428223801748\\
165	0.488243586414839\\
166	0.435860478678261\\
167	0.383926832251388\\
168	0.331446977760759\\
169	0.277893045769115\\
170	0.223220534449482\\
171	0.167811900003296\\
172	0.112374789263037\\
173	0.0578190431943226\\
174	0.00513527654749957\\
175	-0.0446585160270978\\
176	-0.0903864307426598\\
177	-0.130433943742317\\
178	-0.162601986985226\\
179	-0.184310997275146\\
180	-0.193055251784859\\
181	-0.186879050078181\\
182	-0.164727611250228\\
183	-0.126620333049021\\
184	-0.0736502393879799\\
185	-0.00784415729967103\\
186	0.0680694629983318\\
187	0.150937544675102\\
188	0.237463175623671\\
189	0.324458447961318\\
190	0.409039598098855\\
191	0.488734958696735\\
192	0.561510518727298\\
193	0.62576841870329\\
194	0.680364987230417\\
195	0.724640682187212\\
196	0.758431839823048\\
197	0.782048079325033\\
198	0.796214814627836\\
199	0.80198859774885\\
200	0.800656601198927\\
201	0.793631859831385\\
202	0.782354411965586\\
203	0.768205630652819\\
204	0.752442260273176\\
205	0.736185613419164\\
206	0.720534467409288\\
207	0.706775441498396\\
208	0.696504521732584\\
209	0.691519488564275\\
210	0.693546425526763\\
211	0.703947024272359\\
212	0.723501662358643\\
213	0.752302634437923\\
214	0.789755951869418\\
215	0.834670427999017\\
216	0.885404705182203\\
217	0.94004247493445\\
218	0.996569747445573\\
219	1.05303670076277\\
220	1.10773204440229\\
221	1.15944734366462\\
222	1.20780030421496\\
223	1.25339568581964\\
224	1.29766221768638\\
225	1.34245412637966\\
226	1.38960815194881\\
227	1.44058280542553\\
228	1.49623094810196\\
229	1.55671087979814\\
230	1.62151490992438\\
231	1.6895827153803\\
232	1.75946413360933\\
233	1.82950057805229\\
234	1.89799357220081\\
235	1.9632258580266\\
236	2.02308427091294\\
237	2.07436570293662\\
238	2.11240898367581\\
239	2.13153568421418\\
240	2.12607250720948\\
241	2.09143718987386\\
242	2.02495243761832\\
243	1.9262658258053\\
244	1.79738052984225\\
245	1.64237158415518\\
246	1.4668910040992\\
247	1.27756734635835\\
248	1.08139119906757\\
249	0.885158700070758\\
250	0.695054746714781\\
251	0.516472498933398\\
252	0.354048746798071\\
253	0.211712629226683\\
254	0.0925848889461572\\
255	-0.0012255469586997\\
256	-0.0687945798091254\\
257	-0.110446457632164\\
258	-0.127690300615876\\
259	-0.1230466679463\\
260	-0.099798096449951\\
261	-0.0617021544476051\\
262	-0.0127024235248725\\
263	0.0433339974108916\\
264	0.102835353282672\\
265	0.162732554271226\\
266	0.220739907940689\\
267	0.275681361162826\\
268	0.327635426044403\\
269	0.377734087642248\\
270	0.427712067426802\\
271	0.479409058300286\\
272	0.534359779370556\\
273	0.593526883094411\\
274	0.657183106907152\\
275	0.724921135559023\\
276	0.795756929643746\\
277	0.868289367992514\\
278	0.940882876624778\\
279	1.01184352411532\\
280	1.07952304302358\\
281	1.14224388985461\\
282	1.19807444358375\\
283	1.24470958299141\\
284	1.27965083172189\\
285	1.30059984076852\\
286	1.30586428728924\\
287	1.29464777896041\\
288	1.267179651094\\
289	1.22469021799499\\
290	1.16926393697436\\
291	1.10361340750001\\
292	1.03081729699303\\
293	0.954058655934251\\
294	0.876394038954184\\
295	0.800621395971712\\
296	0.729358540320594\\
297	0.665296792084699\\
298	0.611350198739859\\
299	0.570486985817646\\
300	0.54533324205066\\
301	0.537763309674166\\
302	0.548613667213433\\
303	0.577566656928425\\
304	0.623197298136901\\
305	0.683147843175855\\
306	0.754383768635812\\
307	0.833485029826188\\
308	0.916933525292543\\
309	1.00136572136119\\
310	1.08374020981991\\
311	1.16134645663445\\
312	1.23167704729122\\
313	1.29234196673388\\
314	1.34116333410961\\
315	1.37639881305563\\
316	1.39696369980387\\
317	1.40257026193451\\
318	1.39375917816494\\
319	1.37183052048083\\
320	1.338698823604\\
321	1.29670280038687\\
322	1.24839946520716\\
323	1.19636710096784\\
324	1.14303738661598\\
325	1.09061033902677\\
326	1.0411444695736\\
327	0.996790429999972\\
328	0.959929456137343\\
329	0.933035147015028\\
330	0.918336061129135\\
331	0.91746293970798\\
332	0.931198299871603\\
333	0.959369066878396\\
334	1.00087753173211\\
335	1.05384137863341\\
336	1.11580395420747\\
337	1.1839758770974\\
338	1.25547484040726\\
339	1.32753784851277\\
340	1.39767504516711\\
341	1.46372688680082\\
342	1.52383466419765\\
343	1.57641018443786\\
344	1.62017305431037\\
345	1.65423530195866\\
346	1.67817647662627\\
347	1.69207499271887\\
348	1.69648742864925\\
349	1.69238212665652\\
350	1.68104069992365\\
351	1.66394316539471\\
352	1.64265127147232\\
353	1.6187018807695\\
354	1.59351753000339\\
355	1.56832011930923\\
356	1.54401088204867\\
357	1.52102898092023\\
358	1.49928764393744\\
359	1.47825977916023\\
360	1.45717149425988\\
361	1.43521572907159\\
362	1.41172762684457\\
363	1.38629792237492\\
364	1.35882170974389\\
365	1.32949205535919\\
366	1.29875336613697\\
367	1.2672307434488\\
368	1.23564951397054\\
369	1.20475974373334\\
370	1.17533153373468\\
371	1.14834436686908\\
372	1.12533055822254\\
373	1.10855671587969\\
374	1.10080585492352\\
375	1.10487238309355\\
376	1.12302523412878\\
377	1.15660454992064\\
378	1.20581223934317\\
379	1.26969423705235\\
380	1.34627783350477\\
381	1.43281335122903\\
382	1.52606814212383\\
383	1.62262829673917\\
384	1.71916947979117\\
385	1.81259976478634\\
386	1.89990983722461\\
387	1.97778201222645\\
388	2.04236987592065\\
389	2.08956137629194\\
390	2.11558844765005\\
391	2.11766191952585\\
392	2.09442590176623\\
393	2.04616033720418\\
394	1.97473972859817\\
395	1.88339899076774\\
396	1.77637428224859\\
397	1.65848677015354\\
398	1.53472788347559\\
399	1.40988813548323\\
400	1.28822437911319\\
401	1.17311056124352\\
402	1.0666980827229\\
403	0.96976187016125\\
404	0.881855676712831\\
405	0.801693444514465\\
406	0.727589429850506\\
407	0.657842914005961\\
408	0.591016652072487\\
409	0.526097599220636\\
410	0.46255189815565\\
411	0.400297580624765\\
412	0.339622363766044\\
413	0.281071524231135\\
414	0.225333222078494\\
415	0.17324503463897\\
416	0.126153688164275\\
417	0.0865550275370884\\
418	0.0584247898898631\\
419	0.0467980485955921\\
420	0.056809077922418\\
421	0.0926707919858885\\
422	0.156904444243479\\
423	0.249933371643044\\
424	0.370037293790047\\
425	0.513598983114004\\
426	0.675548423546084\\
427	0.849907224203858\\
428	1.0303488592771\\
429	1.2107077367336\\
430	1.38535681352206\\
431	1.54935477643077\\
432	1.69838453356876\\
433	1.82869459285102\\
434	1.93721191607256\\
435	2.0217735773981\\
436	2.0813335740288\\
437	2.11605818318329\\
438	2.12728854480253\\
439	2.1173860893539\\
440	2.08949485638088\\
441	2.04726022111866\\
442	1.99454081443994\\
443	1.93514366333356\\
444	1.87260109336518\\
445	1.80996025796218\\
446	1.74950499950597\\
447	1.69243716149106\\
448	1.63873432483855\\
449	1.58734121452369\\
450	1.53660210861069\\
451	1.48473986690252\\
452	1.43025200819556\\
453	1.37217065226159\\
454	1.3101797593908\\
455	1.2446098982874\\
456	1.17634307649396\\
457	1.10666294630101\\
458	1.03708250604666\\
459	0.969175352258516\\
460	0.904432571646759\\
461	0.844163948589996\\
462	0.789445417941271\\
463	0.741091964648228\\
464	0.699636757880764\\
465	0.665317717363186\\
466	0.638080962212184\\
467	0.617604741044392\\
468	0.603341279456749\\
469	0.594570985802611\\
470	0.590462491428416\\
471	0.590132377726301\\
472	0.592699605164275\\
473	0.597331075611518\\
474	0.603276737495692\\
475	0.609901035422173\\
476	0.616726460709481\\
477	0.623485075430747\\
478	0.630139043208309\\
479	0.636841793693562\\
480	0.643856684588876\\
481	0.651468378670373\\
482	0.65991028622993\\
483	0.669317496300418\\
484	0.679706144166123\\
485	0.690975308654215\\
486	0.70292534296599\\
487	0.715286055372416\\
488	0.72774887900304\\
489	0.739997672958369\\
490	0.751723657655798\\
491	0.762599576291613\\
492	0.77222038383853\\
493	0.780071700328262\\
494	0.78557237089141\\
495	0.788170026305235\\
496	0.787440942319573\\
497	0.783162895512833\\
498	0.775349982924076\\
499	0.764250371473135\\
500	0.750314471039096\\
501	0.734143655414977\\
502	0.716429696179712\\
503	0.697893856609566\\
504	0.679231209957347\\
505	0.661044110063003\\
506	0.643724993086392\\
507	0.627303761101492\\
508	0.61137080889918\\
509	0.595157336767014\\
510	0.577729712629593\\
511	0.558203589924638\\
512	0.535915502343503\\
513	0.510527355292676\\
514	0.482062001995087\\
515	0.450880980528696\\
516	0.417621150820194\\
517	0.383108032329989\\
518	0.348261521881323\\
519	0.314007906430267\\
520	0.28123279795017\\
521	0.250833361423732\\
522	0.223852612522423\\
523	0.20155240054803\\
524	0.185316258025849\\
525	0.176430600036123\\
526	0.175857155044212\\
527	0.184068981950548\\
528	0.200975246093086\\
529	0.225932060277773\\
530	0.257821604829715\\
531	0.295175720659658\\
532	0.336320208859126\\
533	0.379518908162822\\
534	0.423105186282183\\
535	0.465648372054423\\
536	0.506268557007232\\
537	0.545057121139184\\
538	0.583289658506471\\
539	0.623200431390734\\
540	0.667439156603526\\
541	0.718474725395214\\
542	0.778120269648396\\
543	0.84724787528987\\
544	0.925697251389622\\
545	1.01234656023527\\
546	1.10529782017496\\
547	1.20212658680497\\
548	1.30015138252211\\
549	1.39668588033737\\
550	1.48921623115808\\
551	1.57542115917268\\
552	1.65305457345689\\
553	1.71987646309983\\
554	1.77377603192485\\
555	1.81303031149662\\
556	1.83655928552391\\
557	1.84409052908604\\
558	1.83620638106641\\
559	1.81428153477581\\
560	1.78033723530564\\
561	1.73684489781612\\
562	1.68651090534939\\
563	1.63206978233881\\
564	1.57610157476722\\
565	1.52081679683977\\
566	1.46767287776249\\
567	1.41687179334001\\
568	1.36711133285578\\
569	1.31586393485062\\
570	1.26003733528217\\
571	1.19669980675347\\
572	1.12366064510627\\
573	1.03982348020624\\
574	0.945306631927384\\
575	0.841368087495176\\
576	0.730191718725633\\
577	0.614594838135512\\
578	0.497710177414899\\
579	0.38268823156736\\
580	0.272517810298495\\
581	0.170122657905706\\
582	0.0786898717039961\\
583	0.0018506037090108\\
584	-0.0565763824210628\\
585	-0.093152741576833\\
586	-0.105404712767127\\
587	-0.0922350218191694\\
588	-0.0540985356854473\\
589	0.00704913357866621\\
590	0.0879814634544098\\
591	0.184545388973691\\
592	0.292056616789651\\
593	0.405691168749796\\
594	0.520835029831075\\
595	0.63341409400836\\
596	0.740291173657962\\
597	0.839693628925682\\
598	0.931413876154835\\
599	1.01659792244628\\
600	1.09723396213754\\
601	1.17557412518962\\
602	1.25364615854804\\
603	1.33292139701424\\
604	1.41414986758359\\
605	1.49734107651231\\
606	1.58185388069939\\
607	1.66655474760539\\
608	1.75000747318907\\
609	1.83066027821296\\
610	1.90694146148211\\
611	1.97711256479155\\
612	2.03892157586388\\
613	2.08942263400858\\
614	2.12523964674126\\
615	2.14314679482262\\
616	2.14067354918561\\
617	2.11654637062766\\
618	2.07090100652143\\
619	2.00527132624929\\
620	1.92239990070592\\
621	1.82593116565566\\
622	1.72004865767089\\
623	1.60910896334418\\
624	1.49731471020019\\
625	1.38849358044969\\
626	1.28608047478415\\
627	1.19327531589969\\
628	1.11314645072647\\
629	1.04850120438242\\
630	1.00159100062079\\
631	0.973819461778902\\
632	0.965558875552819\\
633	0.976107467799625\\
634	1.00377767331181\\
635	1.04608342569539\\
636	1.09998670546062\\
637	1.16216461615515\\
638	1.22926523540451\\
639	1.29812548467726\\
640	1.3658754761278\\
641	1.42979733923746\\
642	1.48698392327039\\
643	1.53413921527927\\
644	1.56777988413253\\
645	1.58472568413692\\
646	1.58261388154736\\
647	1.56026774979949\\
648	1.51785991623428\\
649	1.45687665281715\\
650	1.37992460491585\\
651	1.29043539921448\\
652	1.192323778788\\
653	1.08964702443573\\
654	0.986301386308594\\
655	0.885774948233213\\
656	0.790960559179643\\
657	0.704032945845044\\
658	0.626404407213838\\
659	0.558763452917228\\
660	0.501175568759379\\
661	0.453215375243904\\
662	0.414106019221707\\
663	0.382850196033751\\
664	0.358343567933615\\
665	0.339466137591352\\
666	0.325150496973163\\
667	0.31442828665393\\
668	0.306457204038164\\
669	0.300534365607026\\
670	0.296144192428385\\
671	0.293135949192808\\
672	0.292001600219032\\
673	0.294012138330106\\
674	0.301032207740892\\
675	0.315103699941374\\
676	0.337999595649078\\
677	0.370879315517041\\
678	0.41409491010815\\
679	0.467148365896302\\
680	0.528772990880413\\
681	0.597100304444862\\
682	0.669872423949077\\
683	0.744664992012055\\
684	0.819091996250243\\
685	0.890948121060249\\
686	0.958225215536732\\
687	1.01901899572991\\
688	1.07147124534027\\
689	1.11386021763444\\
690	1.14479573952606\\
691	1.16341147190221\\
692	1.1694870734591\\
693	1.16347958104594\\
694	1.14647030866587\\
695	1.12004774341696\\
696	1.08615190807963\\
697	1.0469050127661\\
698	1.00444879038936\\
699	0.960805600087983\\
700	0.917809647642597\\
701	0.877188608772352\\
702	0.840768197252927\\
703	0.81059238272518\\
704	0.788801802541763\\
705	0.777338109073285\\
706	0.777634438850583\\
707	0.790394704683347\\
708	0.815497328148434\\
709	0.852019515403041\\
710	0.898356747805416\\
711	0.952403838322071\\
712	1.01176361967936\\
713	1.07395491452722\\
714	1.13659373247947\\
715	1.19745391711341\\
716	1.25423629634683\\
717	1.30410372262266\\
718	1.343423715171\\
719	1.36805188420624\\
720	1.37400459634258\\
721	1.3581706838541\\
722	1.31883617979913\\
723	1.25594114978391\\
724	1.17107369464409\\
725	1.06725325263901\\
726	0.948574327231105\\
727	0.819782820796158\\
728	0.685847234490166\\
729	0.551573167104962\\
730	0.421310707615593\\
731	0.298808617291776\\
732	0.187204666109712\\
733	0.0890434781360581\\
734	0.0062327257063427\\
735	-0.0600418887813559\\
736	-0.109377156125051\\
737	-0.142150823834965\\
738	-0.159452282289282\\
739	-0.162957216248304\\
740	-0.154766062370516\\
741	-0.137227832544041\\
742	-0.112768435009454\\
743	-0.0837388985941958\\
744	-0.0522897253184507\\
745	-0.0202094597942506\\
746	0.0114093052292878\\
747	0.0426303760485198\\
748	0.0748989487894675\\
749	0.11076379458231\\
750	0.153236795001561\\
751	0.205096371275887\\
752	0.26833627110195\\
753	0.343837809690143\\
754	0.431269374966913\\
755	0.5291751957573\\
756	0.635197254951174\\
757	0.746371221029736\\
758	0.859444496949556\\
759	0.971171376570233\\
760	1.07848475184772\\
761	1.17837967543451\\
762	1.26755634713472\\
763	1.34222280628747\\
764	1.3983621700884\\
765	1.43233151652325\\
766	1.44148026085864\\
767	1.42458865408607\\
768	1.38205821432962\\
769	1.3158631189394\\
770	1.22931329870651\\
771	1.12669614922159\\
772	1.0128638287154\\
773	0.892822985116734\\
774	0.771372335126634\\
775	0.652862027467565\\
776	0.541183542503948\\
777	0.439957796013032\\
778	0.352659447126515\\
779	0.282474429924321\\
780	0.231967401765079\\
781	0.202750765920571\\
782	0.195275406224072\\
783	0.208780422941771\\
784	0.241391118117579\\
785	0.290329274963754\\
786	0.352190773669743\\
787	0.423246930447517\\
788	0.499732896634161\\
789	0.578098040939135\\
790	0.655229919795758\\
791	0.728702257951871\\
792	0.797027441731068\\
793	0.85976710359022\\
794	0.917396796961643\\
795	0.970992096353335\\
796	1.02187311277811\\
797	1.07130004352787\\
798	1.12025977234317\\
799	1.16935103929556\\
800	1.21875662182362\\
801	1.26828181178118\\
802	1.3174358103737\\
803	1.36553444251413\\
804	1.41180494001559\\
805	1.45545549255614\\
806	1.49565156429566\\
807	1.53141247793124\\
808	1.56156006301599\\
809	1.58481999018019\\
810	1.600031023048\\
811	1.60635806753858\\
812	1.6034427701078\\
813	1.59146931313743\\
814	1.57114888050069\\
815	1.54364017867309\\
816	1.51042877263041\\
817	1.47318790578117\\
818	1.43364017903417\\
819	1.39343451216077\\
820	1.35404584850603\\
821	1.31669828892243\\
822	1.28231351947242\\
823	1.25149224934791\\
824	1.22453182622145\\
825	1.20147085990974\\
826	1.18214679733909\\
827	1.16625561294594\\
828	1.15340689511246\\
829	1.14317058092733\\
830	1.13511374730495\\
831	1.12882732914408\\
832	1.12394360287477\\
833	1.12014572437398\\
834	1.11717141543087\\
835	1.11482169397939\\
836	1.11299523517932\\
837	1.11174289214295\\
838	1.11129249223395\\
839	1.11200673424165\\
840	1.11429331670312\\
841	1.11850936319259\\
842	1.12488754257512\\
843	1.13349415090088\\
844	1.14421916252544\\
845	1.15679253978398\\
846	1.17081868476861\\
847	1.18582059416925\\
848	1.20128642495808\\
849	1.21671192591254\\
850	1.23162000508743\\
851	1.24552477339289\\
852	1.25785027230216\\
853	1.26788591628711\\
854	1.27484072772665\\
855	1.27796842368422\\
856	1.27669854260446\\
857	1.27073193725768\\
858	1.26008593473719\\
859	1.24509043381636\\
860	1.22634487923777\\
861	1.20464957574656\\
862	1.18092477754696\\
863	1.15612964815145\\
864	1.13118682229314\\
865	1.10686078467754\\
866	1.08347457531056\\
867	1.06050722830215\\
868	1.03638639227121\\
869	1.00870902202875\\
870	0.974771825781954\\
871	0.932149473879066\\
872	0.87914861543092\\
873	0.81507153656124\\
874	0.740286852121989\\
875	0.65614026957859\\
876	0.564753779094987\\
877	0.468763990578715\\
878	0.371043996351661\\
879	0.274446938160002\\
880	0.1816544761562\\
881	0.0952654881594054\\
882	0.0180857694172075\\
883	-0.0467101803714713\\
884	-0.0957960473096264\\
885	-0.126180929558785\\
886	-0.135711465616711\\
887	-0.1234259018231\\
888	-0.0897046602449878\\
889	-0.0362252173732942\\
890	0.0342368238616962\\
891	0.11810536135225\\
892	0.211345260223443\\
893	0.309800489423889\\
894	0.409495798521703\\
895	0.506916579549301\\
896	0.599332500654796\\
897	0.68513677558347\\
898	0.764003488564626\\
899	0.836721975303633\\
900	0.904795901992566\\
901	0.969987867718175\\
902	1.03393156297473\\
903	1.09786377103353\\
904	1.16248553423282\\
905	1.2279367133144\\
906	1.29385620148818\\
907	1.3594968031508\\
908	1.42386599436681\\
909	1.48586909182662\\
910	1.54443952137154\\
911	1.59864925236495\\
912	1.64779297646126\\
913	1.69143533677523\\
914	1.729416415165\\
915	1.76182409201924\\
916	1.78894756227301\\
917	1.81122358073393\\
918	1.82918317640813\\
919	1.84340369770664\\
920	1.85446886448584\\
921	1.86293789508231\\
922	1.86932367769393\\
923	1.8740791342454\\
924	1.87759106391306\\
925	1.88018933550219\\
926	1.88218908419309\\
927	1.88395780647952\\
928	1.88595524706712\\
929	1.88870713887731\\
930	1.89273062102432\\
931	1.89845251249768\\
932	1.90614731998306\\
933	1.91590501322636\\
934	1.9276285861263\\
935	1.94105582617893\\
936	1.95579749660183\\
937	1.97138362725921\\
938	1.98731163127779\\
939	2.00308500554673\\
940	2.0181313681908\\
941	2.031376502367\\
942	2.04055244117442\\
943	2.04183109181991\\
944	2.03022528514204\\
945	2.00054365518037\\
946	1.94841740382359\\
947	1.87108507200099\\
948	1.76781836143658\\
949	1.63998963944428\\
950	1.49084695098057\\
951	1.32508963671768\\
952	1.14834060362432\\
953	0.966598995916511\\
954	0.78574026940617\\
955	0.611147828447191\\
956	0.447583263777695\\
957	0.29927132052632\\
958	0.16996829988341\\
959	0.0628307746750013\\
960	-0.0198546717762502\\
961	-0.0769198168036325\\
962	-0.108443359151169\\
963	-0.115733939611523\\
964	-0.10118300326038\\
965	-0.0680236625514008\\
966	-0.0200382303803399\\
967	0.0387454830220755\\
968	0.104336078750363\\
969	0.173018107844259\\
970	0.241582328365908\\
971	0.307637509172701\\
972	0.369972276604331\\
973	0.42871960807226\\
974	0.485144401225286\\
975	0.541158577242261\\
976	0.598783853846522\\
977	0.659708806293483\\
978	0.725000171540734\\
979	0.794975533443181\\
980	0.86921418912478\\
981	0.946669179703221\\
982	1.02584027234828\\
983	1.10497185530359\\
984	1.18224348880572\\
985	1.25587670553573\\
986	1.32403113492708\\
987	1.38452542494249\\
988	1.43468918426423\\
989	1.47157846637541\\
990	1.49245054636306\\
991	1.49525598965251\\
992	1.4789930219603\\
993	1.44387044255857\\
994	1.39128507280565\\
995	1.32365215104289\\
996	1.24413984523939\\
997	1.15635930538856\\
998	1.06405424180236\\
999	0.970824317327381\\
1000	0.879922447971888\\
1001	0.794174815505026\\
1002	0.716011599648334\\
1003	0.647501284588291\\
1004	0.590302817317092\\
1005	0.545560586574801\\
1006	0.513812833731583\\
1007	0.494955796429606\\
1008	0.488273474827889\\
1009	0.492524684350377\\
1010	0.50607004714078\\
1011	0.527019063875384\\
1012	0.553378819050436\\
1013	0.58318974155081\\
1014	0.614637154504463\\
1015	0.646112600932051\\
1016	0.676181355712895\\
1017	0.703472578725612\\
1018	0.726607955336414\\
1019	0.744257469709964\\
1020	0.755287086744543\\
1021	0.75891210097907\\
1022	0.754801244092767\\
1023	0.743113205003625\\
1024	0.724468589740296\\
1025	0.699871817147283\\
1026	0.670601691395699\\
1027	0.63808940908088\\
1028	0.60379913651885\\
1029	0.56912733527032\\
1030	0.535411672218697\\
1031	0.504224657960319\\
1032	0.477891181024038\\
1033	0.459769157884784\\
1034	0.4539476056198\\
1035	0.464524536071507\\
1036	0.494835218241519\\
1037	0.546871223556329\\
1038	0.620978341263806\\
1039	0.715830807489968\\
1040	0.828629287201296\\
1041	0.955449532436411\\
1042	1.09166667012286\\
1043	1.23239071548064\\
1044	1.37285766408565\\
1045	1.50863730575366\\
1046	1.63542291712965\\
1047	1.74847607695374\\
1048	1.84231350169718\\
1049	1.91108184194839\\
1050	1.94942554810345\\
};
\addlegendentry{\Huge Clipped}
\addplot [line width = 1.5pt, color=black,solid]
  table[row sep=crcr]{%
1	0\\
2	0\\
3	0\\
4	0\\
5	0\\
6	0\\
7	0.0492956121987101\\
8	0.117023608847397\\
9	0.185696302798263\\
10	0.252877605464125\\
11	0.316792713757472\\
12	0.376697590064492\\
13	0.433057998396404\\
14	0.487355912395488\\
15	0.54162548629292\\
16	0.597936880026256\\
17	0.657973161458312\\
18	0.722759280595411\\
19	0.792549685752308\\
20	0.866851145906614\\
21	0.94454378946453\\
22	1.0240603661914\\
23	1.1035880405761\\
24	1.18126087530027\\
25	1.2552670357485\\
26	1.32374421223811\\
27	1.38449911526351\\
28	1.43485759178368\\
29	1.47187814493774\\
30	1.49282480384062\\
31	1.49565747743151\\
32	1.47938488271363\\
33	1.44422632603823\\
34	1.3915883313773\\
35	1.32389450379758\\
36	1.24431976058135\\
37	1.15648030455127\\
38	1.06412327585316\\
39	0.970850322413793\\
40	0.879915140950151\\
41	0.794143756497055\\
42	0.715965520290821\\
43	0.647447660719481\\
44	0.590247651988647\\
45	0.545508362564503\\
46	0.513766593000571\\
47	0.494917305948398\\
48	0.488243442742059\\
49	0.492502996478297\\
50	0.506056000473078\\
51	0.527011580330895\\
52	0.553376628836385\\
53	0.583191530649053\\
54	0.614641675051536\\
55	0.646118746334397\\
56	0.676188204572014\\
57	0.703479412761343\\
58	0.72661425720354\\
59	0.744262906237915\\
60	0.755291483117682\\
61	0.758915409874958\\
62	0.754803514211788\\
63	0.743114550180409\\
64	0.724469163497451\\
65	0.699871787136762\\
66	0.67060123080954\\
67	0.638088654454646\\
68	0.603798321021745\\
69	0.569125403572982\\
70	0.535366421660618\\
71	0.503864907559111\\
72	0.476297175114895\\
73	0.454830957362793\\
74	0.441955513695004\\
75	0.440075035538006\\
76	0.451077380777375\\
77	0.476015110325821\\
78	0.514948069962139\\
79	0.566944714899216\\
80	0.630210893350422\\
81	0.702303215152261\\
82	0.780383565909171\\
83	0.861476808353902\\
84	0.94270510825479\\
85	1.02151907238326\\
86	1.09599633152589\\
87	1.16517735055575\\
88	1.22922857968642\\
89	1.28928139283012\\
90	1.34703484936613\\
91	1.40430815096717\\
92	1.46266728026926\\
93	1.52317764149554\\
94	1.58629010724054\\
95	1.6518422236698\\
96	1.71914447977244\\
97	1.7871186021022\\
98	1.85445806211491\\
99	1.91978406281908\\
100	1.98173509943461\\
101	2\\
102	2\\
103	2\\
104	2\\
105	2\\
106	2\\
107	2\\
108	2\\
109	2\\
110	2\\
111	1.98584602745186\\
112	1.91579422444864\\
113	1.84205417952196\\
114	1.76752122993451\\
115	1.69474766475572\\
116	1.62564512991698\\
117	1.56114340786105\\
118	1.50102543291454\\
119	1.44409777040994\\
120	1.38860508574431\\
121	1.33269477743854\\
122	1.27480227387035\\
123	1.21390354758196\\
124	1.14962780101086\\
125	1.0822499899122\\
126	1.01259509334775\\
127	0.941888964267579\\
128	0.871587079941896\\
129	0.803209158549565\\
130	0.738244122778487\\
131	0.678231702973314\\
132	0.624991378771421\\
133	0.580743521696281\\
134	0.547928862624569\\
135	0.528812950287981\\
136	0.525077007356962\\
137	0.537524072052994\\
138	0.565944966813593\\
139	0.609138886535052\\
140	0.665056417696476\\
141	0.731022245070482\\
142	0.803994562812215\\
143	0.880824732157752\\
144	0.958486947118338\\
145	1.03421359385035\\
146	1.10543204588607\\
147	1.16953549132192\\
148	1.22374691499668\\
149	1.26527429763801\\
150	1.29167437171015\\
151	1.30122707785151\\
152	1.29319477028059\\
153	1.26792386575134\\
154	1.22679570222929\\
155	1.17205968550569\\
156	1.10659188110261\\
157	1.033621810548\\
158	0.956464084608389\\
159	0.878280292787706\\
160	0.801856787357711\\
161	0.729341497643846\\
162	0.66196326339363\\
163	0.599901741407388\\
164	0.542428223801748\\
165	0.488243586414839\\
166	0.435860478678261\\
167	0.383926832251388\\
168	0.331446977760759\\
169	0.277893045769115\\
170	0.223220534449482\\
171	0.167811900003296\\
172	0.112374789263037\\
173	0.0578190431943226\\
174	0.00513527654749957\\
175	0\\
176	0\\
177	0\\
178	0\\
179	0\\
180	0\\
181	0\\
182	0\\
183	0\\
184	0\\
185	0\\
186	0.0680694629983318\\
187	0.150937544675102\\
188	0.237463175623671\\
189	0.324458447961318\\
190	0.409039598098855\\
191	0.488734958696735\\
192	0.561510518727298\\
193	0.62576841870329\\
194	0.680364987230417\\
195	0.724640682187212\\
196	0.758431839823048\\
197	0.782048079325033\\
198	0.796214814627836\\
199	0.80198859774885\\
200	0.800656601198927\\
201	0.793631859831385\\
202	0.782354411965586\\
203	0.768205630652819\\
204	0.752442260273176\\
205	0.736185613419164\\
206	0.720534467409288\\
207	0.706775441498396\\
208	0.696504521732584\\
209	0.691519488564275\\
210	0.693546425526763\\
211	0.703947024272359\\
212	0.723501662358643\\
213	0.752302634437923\\
214	0.789755951869418\\
215	0.834670427999017\\
216	0.885404705182203\\
217	0.94004247493445\\
218	0.996569747445573\\
219	1.05303670076277\\
220	1.10773204440229\\
221	1.15944734366462\\
222	1.20780030421496\\
223	1.25339568581964\\
224	1.29766221768638\\
225	1.34245412637966\\
226	1.38960815194881\\
227	1.44058280542553\\
228	1.49623094810196\\
229	1.55671087979814\\
230	1.62151490992438\\
231	1.6895827153803\\
232	1.75946413360933\\
233	1.82950057805229\\
234	1.89799357220081\\
235	1.9632258580266\\
236	2\\
237	2\\
238	2\\
239	2\\
240	2\\
241	2\\
242	2\\
243	1.9262658258053\\
244	1.79738052984225\\
245	1.64237158415518\\
246	1.4668910040992\\
247	1.27756734635835\\
248	1.08139119906757\\
249	0.885158700070758\\
250	0.695054746714781\\
251	0.516472498933398\\
252	0.354048746798071\\
253	0.211712629226683\\
254	0.0925848889461572\\
255	0\\
256	0\\
257	0\\
258	0\\
259	0\\
260	0\\
261	0\\
262	0\\
263	0.0433339974108916\\
264	0.102835353282672\\
265	0.162732554271226\\
266	0.220739907940689\\
267	0.275681361162826\\
268	0.327635426044403\\
269	0.377734087642248\\
270	0.427712067426802\\
271	0.479409058300286\\
272	0.534359779370556\\
273	0.593526883094411\\
274	0.657183106907152\\
275	0.724921135559023\\
276	0.795756929643746\\
277	0.868289367992514\\
278	0.940882876624778\\
279	1.01184352411532\\
280	1.07952304302358\\
281	1.14224388985461\\
282	1.19807444358375\\
283	1.24470958299141\\
284	1.27965083172189\\
285	1.30059984076852\\
286	1.30586428728924\\
287	1.29464777896041\\
288	1.267179651094\\
289	1.22469021799499\\
290	1.16926393697436\\
291	1.10361340750001\\
292	1.03081729699303\\
293	0.954058655934251\\
294	0.876394038954184\\
295	0.800621395971712\\
296	0.729358540320594\\
297	0.665296792084699\\
298	0.611350198739859\\
299	0.570486985817646\\
300	0.54533324205066\\
301	0.537763309674166\\
302	0.548613667213433\\
303	0.577566656928425\\
304	0.623197298136901\\
305	0.683147843175855\\
306	0.754383768635812\\
307	0.833485029826188\\
308	0.916933525292543\\
309	1.00136572136119\\
310	1.08374020981991\\
311	1.16134645663445\\
312	1.23167704729122\\
313	1.29234196673388\\
314	1.34116333410961\\
315	1.37639881305563\\
316	1.39696369980387\\
317	1.40257026193451\\
318	1.39375917816494\\
319	1.37183052048083\\
320	1.338698823604\\
321	1.29670280038687\\
322	1.24839946520716\\
323	1.19636710096784\\
324	1.14303738661598\\
325	1.09061033902677\\
326	1.0411444695736\\
327	0.996790429999972\\
328	0.959929456137343\\
329	0.933035147015028\\
330	0.918336061129135\\
331	0.91746293970798\\
332	0.931198299871603\\
333	0.959369066878396\\
334	1.00087753173211\\
335	1.05384137863341\\
336	1.11580395420747\\
337	1.1839758770974\\
338	1.25547484040726\\
339	1.32753784851277\\
340	1.39767504516711\\
341	1.46372688680082\\
342	1.52383466419765\\
343	1.57641018443786\\
344	1.62017305431037\\
345	1.65423530195866\\
346	1.67817647662627\\
347	1.69207499271887\\
348	1.69648742864925\\
349	1.69238212665652\\
350	1.68104069992365\\
351	1.66394316539471\\
352	1.64265127147232\\
353	1.6187018807695\\
354	1.59351753000339\\
355	1.56832011930923\\
356	1.54401088204867\\
357	1.52102898092023\\
358	1.49928764393744\\
359	1.47825977916023\\
360	1.45717149425988\\
361	1.43521572907159\\
362	1.41172762684457\\
363	1.38629792237492\\
364	1.35882170974389\\
365	1.32949205535919\\
366	1.29875336613697\\
367	1.2672307434488\\
368	1.23564951397054\\
369	1.20475974373334\\
370	1.17533153373468\\
371	1.14834436686908\\
372	1.12533055822254\\
373	1.10855671587969\\
374	1.10080585492352\\
375	1.10487238309355\\
376	1.12302523412878\\
377	1.15660454992064\\
378	1.20581223934317\\
379	1.26969423705235\\
380	1.34627783350477\\
381	1.43281335122903\\
382	1.52606814212383\\
383	1.62262829673917\\
384	1.71916947979117\\
385	1.81259976478634\\
386	1.89990983722461\\
387	1.97778201222645\\
388	2\\
389	2\\
390	2\\
391	2\\
392	2\\
393	2\\
394	1.97473972859817\\
395	1.88339899076774\\
396	1.77637428224859\\
397	1.65848677015354\\
398	1.53472788347559\\
399	1.40988813548323\\
400	1.28822437911319\\
401	1.17311056124352\\
402	1.0666980827229\\
403	0.96976187016125\\
404	0.881855676712831\\
405	0.801693444514465\\
406	0.727589429850506\\
407	0.657842914005961\\
408	0.591016652072487\\
409	0.526097599220636\\
410	0.46255189815565\\
411	0.400297580624765\\
412	0.339622363766044\\
413	0.281071524231135\\
414	0.225333222078494\\
415	0.17324503463897\\
416	0.126153688164275\\
417	0.0865550275370884\\
418	0.0584247898898631\\
419	0.0467980485955921\\
420	0.056809077922418\\
421	0.0926707919858885\\
422	0.156904444243479\\
423	0.249933371643044\\
424	0.370037293790047\\
425	0.513598983114004\\
426	0.675548423546084\\
427	0.849907224203858\\
428	1.0303488592771\\
429	1.2107077367336\\
430	1.38535681352206\\
431	1.54935477643077\\
432	1.69838453356876\\
433	1.82869459285102\\
434	1.93721191607256\\
435	2\\
436	2\\
437	2\\
438	2\\
439	2\\
440	2\\
441	2\\
442	1.99454081443994\\
443	1.93514366333356\\
444	1.87260109336518\\
445	1.80996025796218\\
446	1.74950499950597\\
447	1.69243716149106\\
448	1.63873432483855\\
449	1.58734121452369\\
450	1.53660210861069\\
451	1.48473986690252\\
452	1.43025200819556\\
453	1.37217065226159\\
454	1.3101797593908\\
455	1.2446098982874\\
456	1.17634307649396\\
457	1.10666294630101\\
458	1.03708250604666\\
459	0.969175352258516\\
460	0.904432571646759\\
461	0.844163948589996\\
462	0.789445417941271\\
463	0.741091964648228\\
464	0.699636757880764\\
465	0.665317717363186\\
466	0.638080962212184\\
467	0.617604741044392\\
468	0.603341279456749\\
469	0.594570985802611\\
470	0.590462491428416\\
471	0.590132377726301\\
472	0.592699605164275\\
473	0.597331075611518\\
474	0.603276737495692\\
475	0.609901035422173\\
476	0.616726460709481\\
477	0.623485075430747\\
478	0.630139043208309\\
479	0.636841793693562\\
480	0.643856684588876\\
481	0.651468378670373\\
482	0.65991028622993\\
483	0.669317496300418\\
484	0.679706144166123\\
485	0.690975308654215\\
486	0.70292534296599\\
487	0.715286055372416\\
488	0.72774887900304\\
489	0.739997672958369\\
490	0.751723657655798\\
491	0.762599576291613\\
492	0.77222038383853\\
493	0.780071700328262\\
494	0.78557237089141\\
495	0.788170026305235\\
496	0.787440942319573\\
497	0.783162895512833\\
498	0.775349982924076\\
499	0.764250371473135\\
500	0.750314471039096\\
501	0.734143655414977\\
502	0.716429696179712\\
503	0.697893856609566\\
504	0.679231209957347\\
505	0.661044110063003\\
506	0.643724993086392\\
507	0.627303761101492\\
508	0.61137080889918\\
509	0.595157336767014\\
510	0.577729712629593\\
511	0.558203589924638\\
512	0.535915502343503\\
513	0.510527355292676\\
514	0.482062001995087\\
515	0.450880980528696\\
516	0.417621150820194\\
517	0.383108032329989\\
518	0.348261521881323\\
519	0.314007906430267\\
520	0.28123279795017\\
521	0.250833361423732\\
522	0.223852612522423\\
523	0.20155240054803\\
524	0.185316258025849\\
525	0.176430600036123\\
526	0.175857155044212\\
527	0.184068981950548\\
528	0.200975246093086\\
529	0.225932060277773\\
530	0.257821604829715\\
531	0.295175720659658\\
532	0.336320208859126\\
533	0.379518908162822\\
534	0.423105186282183\\
535	0.465648372054423\\
536	0.506268557007232\\
537	0.545057121139184\\
538	0.583289658506471\\
539	0.623200431390734\\
540	0.667439156603526\\
541	0.718474725395214\\
542	0.778120269648396\\
543	0.84724787528987\\
544	0.925697251389622\\
545	1.01234656023527\\
546	1.10529782017496\\
547	1.20212658680497\\
548	1.30015138252211\\
549	1.39668588033737\\
550	1.48921623115808\\
551	1.57542115917268\\
552	1.65305457345689\\
553	1.71987646309983\\
554	1.77377603192485\\
555	1.81303031149662\\
556	1.83655928552391\\
557	1.84409052908604\\
558	1.83620638106641\\
559	1.81428153477581\\
560	1.78033723530564\\
561	1.73684489781612\\
562	1.68651090534939\\
563	1.63206978233881\\
564	1.57610157476722\\
565	1.52081679683977\\
566	1.46767287776249\\
567	1.41687179334001\\
568	1.36711133285578\\
569	1.31586393485062\\
570	1.26003733528217\\
571	1.19669980675347\\
572	1.12366064510627\\
573	1.03982348020624\\
574	0.945306631927384\\
575	0.841368087495176\\
576	0.730191718725633\\
577	0.614594838135512\\
578	0.497710177414899\\
579	0.38268823156736\\
580	0.272517810298495\\
581	0.170122657905706\\
582	0.0786898717039961\\
583	0.0018506037090108\\
584	0\\
585	0\\
586	0\\
587	0\\
588	0\\
589	0.00704913357866621\\
590	0.0879814634544098\\
591	0.184545388973691\\
592	0.292056616789651\\
593	0.405691168749796\\
594	0.520835029831075\\
595	0.63341409400836\\
596	0.740291173657962\\
597	0.839693628925682\\
598	0.931413876154835\\
599	1.01659792244628\\
600	1.09723396213754\\
601	1.17557412518962\\
602	1.25364615854804\\
603	1.33292139701424\\
604	1.41414986758359\\
605	1.49734107651231\\
606	1.58185388069939\\
607	1.66655474760539\\
608	1.75000747318907\\
609	1.83066027821296\\
610	1.90694146148211\\
611	1.97711256479155\\
612	2\\
613	2\\
614	2\\
615	2\\
616	2\\
617	2\\
618	2\\
619	2\\
620	1.92239990070592\\
621	1.82593116565566\\
622	1.72004865767089\\
623	1.60910896334418\\
624	1.49731471020019\\
625	1.38849358044969\\
626	1.28608047478415\\
627	1.19327531589969\\
628	1.11314645072647\\
629	1.04850120438242\\
630	1.00159100062079\\
631	0.973819461778902\\
632	0.965558875552819\\
633	0.976107467799625\\
634	1.00377767331181\\
635	1.04608342569539\\
636	1.09998670546062\\
637	1.16216461615515\\
638	1.22926523540451\\
639	1.29812548467726\\
640	1.3658754761278\\
641	1.42979733923746\\
642	1.48698392327039\\
643	1.53413921527927\\
644	1.56777988413253\\
645	1.58472568413692\\
646	1.58261388154736\\
647	1.56026774979949\\
648	1.51785991623428\\
649	1.45687665281715\\
650	1.37992460491585\\
651	1.29043539921448\\
652	1.192323778788\\
653	1.08964702443573\\
654	0.986301386308594\\
655	0.885774948233213\\
656	0.790960559179643\\
657	0.704032945845044\\
658	0.626404407213838\\
659	0.558763452917228\\
660	0.501175568759379\\
661	0.453215375243904\\
662	0.414106019221707\\
663	0.382850196033751\\
664	0.358343567933615\\
665	0.339466137591352\\
666	0.325150496973163\\
667	0.31442828665393\\
668	0.306457204038164\\
669	0.300534365607026\\
670	0.296144192428385\\
671	0.293135949192808\\
672	0.292001600219032\\
673	0.294012138330106\\
674	0.301032207740892\\
675	0.315103699941374\\
676	0.337999595649078\\
677	0.370879315517041\\
678	0.41409491010815\\
679	0.467148365896302\\
680	0.528772990880413\\
681	0.597100304444862\\
682	0.669872423949077\\
683	0.744664992012055\\
684	0.819091996250243\\
685	0.890948121060249\\
686	0.958225215536732\\
687	1.01901899572991\\
688	1.07147124534027\\
689	1.11386021763444\\
690	1.14479573952606\\
691	1.16341147190221\\
692	1.1694870734591\\
693	1.16347958104594\\
694	1.14647030866587\\
695	1.12004774341696\\
696	1.08615190807963\\
697	1.0469050127661\\
698	1.00444879038936\\
699	0.960805600087983\\
700	0.917809647642597\\
701	0.877188608772352\\
702	0.840768197252927\\
703	0.81059238272518\\
704	0.788801802541763\\
705	0.777338109073285\\
706	0.777634438850583\\
707	0.790394704683347\\
708	0.815497328148434\\
709	0.852019515403041\\
710	0.898356747805416\\
711	0.952403838322071\\
712	1.01176361967936\\
713	1.07395491452722\\
714	1.13659373247947\\
715	1.19745391711341\\
716	1.25423629634683\\
717	1.30410372262266\\
718	1.343423715171\\
719	1.36805188420624\\
720	1.37400459634258\\
721	1.3581706838541\\
722	1.31883617979913\\
723	1.25594114978391\\
724	1.17107369464409\\
725	1.06725325263901\\
726	0.948574327231105\\
727	0.819782820796158\\
728	0.685847234490166\\
729	0.551573167104962\\
730	0.421310707615593\\
731	0.298808617291776\\
732	0.187204666109712\\
733	0.0890434781360581\\
734	0.0062327257063427\\
735	0\\
736	0\\
737	0\\
738	0\\
739	0\\
740	0\\
741	0\\
742	0\\
743	0\\
744	0\\
745	0\\
746	0.0114093052292878\\
747	0.0426303760485198\\
748	0.0748989487894675\\
749	0.11076379458231\\
750	0.153236795001561\\
751	0.205096371275887\\
752	0.26833627110195\\
753	0.343837809690143\\
754	0.431269374966913\\
755	0.5291751957573\\
756	0.635197254951174\\
757	0.746371221029736\\
758	0.859444496949556\\
759	0.971171376570233\\
760	1.07848475184772\\
761	1.17837967543451\\
762	1.26755634713472\\
763	1.34222280628747\\
764	1.3983621700884\\
765	1.43233151652325\\
766	1.44148026085864\\
767	1.42458865408607\\
768	1.38205821432962\\
769	1.3158631189394\\
770	1.22931329870651\\
771	1.12669614922159\\
772	1.0128638287154\\
773	0.892822985116734\\
774	0.771372335126634\\
775	0.652862027467565\\
776	0.541183542503948\\
777	0.439957796013032\\
778	0.352659447126515\\
779	0.282474429924321\\
780	0.231967401765079\\
781	0.202750765920571\\
782	0.195275406224072\\
783	0.208780422941771\\
784	0.241391118117579\\
785	0.290329274963754\\
786	0.352190773669743\\
787	0.423246930447517\\
788	0.499732896634161\\
789	0.578098040939135\\
790	0.655229919795758\\
791	0.728702257951871\\
792	0.797027441731068\\
793	0.85976710359022\\
794	0.917396796961643\\
795	0.970992096353335\\
796	1.02187311277811\\
797	1.07130004352787\\
798	1.12025977234317\\
799	1.16935103929556\\
800	1.21875662182362\\
801	1.26828181178118\\
802	1.3174358103737\\
803	1.36553444251413\\
804	1.41180494001559\\
805	1.45545549255614\\
806	1.49565156429566\\
807	1.53141247793124\\
808	1.56156006301599\\
809	1.58481999018019\\
810	1.600031023048\\
811	1.60635806753858\\
812	1.6034427701078\\
813	1.59146931313743\\
814	1.57114888050069\\
815	1.54364017867309\\
816	1.51042877263041\\
817	1.47318790578117\\
818	1.43364017903417\\
819	1.39343451216077\\
820	1.35404584850603\\
821	1.31669828892243\\
822	1.28231351947242\\
823	1.25149224934791\\
824	1.22453182622145\\
825	1.20147085990974\\
826	1.18214679733909\\
827	1.16625561294594\\
828	1.15340689511246\\
829	1.14317058092733\\
830	1.13511374730495\\
831	1.12882732914408\\
832	1.12394360287477\\
833	1.12014572437398\\
834	1.11717141543087\\
835	1.11482169397939\\
836	1.11299523517932\\
837	1.11174289214295\\
838	1.11129249223395\\
839	1.11200673424165\\
840	1.11429331670312\\
841	1.11850936319259\\
842	1.12488754257512\\
843	1.13349415090088\\
844	1.14421916252544\\
845	1.15679253978398\\
846	1.17081868476861\\
847	1.18582059416925\\
848	1.20128642495808\\
849	1.21671192591254\\
850	1.23162000508743\\
851	1.24552477339289\\
852	1.25785027230216\\
853	1.26788591628711\\
854	1.27484072772665\\
855	1.27796842368422\\
856	1.27669854260446\\
857	1.27073193725768\\
858	1.26008593473719\\
859	1.24509043381636\\
860	1.22634487923777\\
861	1.20464957574656\\
862	1.18092477754696\\
863	1.15612964815145\\
864	1.13118682229314\\
865	1.10686078467754\\
866	1.08347457531056\\
867	1.06050722830215\\
868	1.03638639227121\\
869	1.00870902202875\\
870	0.974771825781954\\
871	0.932149473879066\\
872	0.87914861543092\\
873	0.81507153656124\\
874	0.740286852121989\\
875	0.65614026957859\\
876	0.564753779094987\\
877	0.468763990578715\\
878	0.371043996351661\\
879	0.274446938160002\\
880	0.1816544761562\\
881	0.0952654881594054\\
882	0.0180857694172075\\
883	0\\
884	0\\
885	0\\
886	0\\
887	0\\
888	0\\
889	0\\
890	0.0342368238616962\\
891	0.11810536135225\\
892	0.211345260223443\\
893	0.309800489423889\\
894	0.409495798521703\\
895	0.506916579549301\\
896	0.599332500654796\\
897	0.68513677558347\\
898	0.764003488564626\\
899	0.836721975303633\\
900	0.904795901992566\\
901	0.969987867718175\\
902	1.03393156297473\\
903	1.09786377103353\\
904	1.16248553423282\\
905	1.2279367133144\\
906	1.29385620148818\\
907	1.3594968031508\\
908	1.42386599436681\\
909	1.48586909182662\\
910	1.54443952137154\\
911	1.59864925236495\\
912	1.64779297646126\\
913	1.69143533677523\\
914	1.729416415165\\
915	1.76182409201924\\
916	1.78894756227301\\
917	1.81122358073393\\
918	1.82918317640813\\
919	1.84340369770664\\
920	1.85446886448584\\
921	1.86293789508231\\
922	1.86932367769393\\
923	1.8740791342454\\
924	1.87759106391306\\
925	1.88018933550219\\
926	1.88218908419309\\
927	1.88395780647952\\
928	1.88595524706712\\
929	1.88870713887731\\
930	1.89273062102432\\
931	1.89845251249768\\
932	1.90614731998306\\
933	1.91590501322636\\
934	1.9276285861263\\
935	1.94105582617893\\
936	1.95579749660183\\
937	1.97138362725921\\
938	1.98731163127779\\
939	2\\
940	2\\
941	2\\
942	2\\
943	2\\
944	2\\
945	2\\
946	1.94841740382359\\
947	1.87108507200099\\
948	1.76781836143658\\
949	1.63998963944428\\
950	1.49084695098057\\
951	1.32508963671768\\
952	1.14834060362432\\
953	0.966598995916511\\
954	0.78574026940617\\
955	0.611147828447191\\
956	0.447583263777695\\
957	0.29927132052632\\
958	0.16996829988341\\
959	0.0628307746750013\\
960	0\\
961	0\\
962	0\\
963	0\\
964	0\\
965	0\\
966	0\\
967	0.0387454830220755\\
968	0.104336078750363\\
969	0.173018107844259\\
970	0.241582328365908\\
971	0.307637509172701\\
972	0.369972276604331\\
973	0.42871960807226\\
974	0.485144401225286\\
975	0.541158577242261\\
976	0.598783853846522\\
977	0.659708806293483\\
978	0.725000171540734\\
979	0.794975533443181\\
980	0.86921418912478\\
981	0.946669179703221\\
982	1.02584027234828\\
983	1.10497185530359\\
984	1.18224348880572\\
985	1.25587670553573\\
986	1.32403113492708\\
987	1.38452542494249\\
988	1.43468918426423\\
989	1.47157846637541\\
990	1.49245054636306\\
991	1.49525598965251\\
992	1.4789930219603\\
993	1.44387044255857\\
994	1.39128507280565\\
995	1.32365215104289\\
996	1.24413984523939\\
997	1.15635930538856\\
998	1.06405424180236\\
999	0.970824317327381\\
1000	0.879922447971888\\
1001	0.794174815505026\\
1002	0.716011599648334\\
1003	0.647501284588291\\
1004	0.590302817317092\\
1005	0.545560586574801\\
1006	0.513812833731583\\
1007	0.494955796429606\\
1008	0.488273474827889\\
1009	0.492524684350377\\
1010	0.50607004714078\\
1011	0.527019063875384\\
1012	0.553378819050436\\
1013	0.58318974155081\\
1014	0.614637154504463\\
1015	0.646112600932051\\
1016	0.676181355712895\\
1017	0.703472578725612\\
1018	0.726607955336414\\
1019	0.744257469709964\\
1020	0.755287086744543\\
1021	0.75891210097907\\
1022	0.754801244092767\\
1023	0.743113205003625\\
1024	0.724468589740296\\
1025	0.699871817147283\\
1026	0.670601691395699\\
1027	0.63808940908088\\
1028	0.60379913651885\\
1029	0.56912733527032\\
1030	0.535411672218697\\
1031	0.504224657960319\\
1032	0.477891181024038\\
1033	0.459769157884784\\
1034	0.4539476056198\\
1035	0.464524536071507\\
1036	0.494835218241519\\
1037	0.546871223556329\\
1038	0.620978341263806\\
1039	0.715830807489968\\
1040	0.828629287201296\\
1041	0.955449532436411\\
1042	1.09166667012286\\
1043	1.23239071548064\\
1044	1.37285766408565\\
1045	1.50863730575366\\
1046	1.63542291712965\\
1047	1.74847607695374\\
1048	1.84231350169718\\
1049	1.91108184194839\\
1050	1.94942554810345\\
};
\end{axis}
\begin{axis}[
anchor=origin,
at={(1250,303)}, 
no markers, domain=-5:5, samples=100,
rotate around={-90:(current axis.origin)}, % Rotate around the origin
axis lines*=center,
every axis y label/.style={at=(current axis.above origin),anchor=south},
every axis x label/.style={at=(current axis.right of origin),anchor=west},
%height=5cm, width=8cm,
width=4.3in,
height=2in,
xtick={-3.2, 3.2}, 
xmin=-5,
xmax=5,
ymin=0,
xticklabels={},
ytick=\empty,
enlargelimits=false, clip=false, axis on top,
grid = major
]
\addplot [fill=gray!40, draw=none, domain=-5:-3.2] {gauss(0,1.7)} \closedcycle;
\addplot [fill=gray!40, draw=none, domain=3.2:5] {gauss(0,1.7)} \closedcycle;
\addplot [very thick,black] {gauss(0,1.7)}; % node [pos=0.5,pin={80:{$\displaystyle\sigma^2 = 2\sum_{n = 1}^{N_u/2} P_n$}},inner sep=0pt] {};
\node (r2) at (axis cs: -3.7, 0.1) {\Huge $2r\sigma$};
\node (r1) at (axis cs: 3.7, 0.1) {\Huge $0$};
\node (nothing) at (axis cs: 3.7, 0.25) {};
\node (nothing2) at (axis cs: 3.7, -1) {};
\end{axis}
\end{tikzpicture}%}
		\caption{DC-OFDM}
	\end{subfigure}%

	\begin{subfigure}[h!]{\textwidth}
		\centering
		\resizebox{\linewidth}{!}{% This file was created by matlab2tikz v0.4.7 (commit e48c22d065f6aed2b831ddad09bcb59bf008f39b) running on MATLAB 8.5.
% Copyright (c) 2008--2014, Nico Schlömer <nico.schloemer@gmail.com>
% All rights reserved.
% Minimal pgfplots version: 1.3
% 
% The latest updates can be retrieved from
%   http://www.mathworks.com/matlabcentral/fileexchange/22022-matlab2tikz
% where you can also make suggestions and rate matlab2tikz.
% 
\begin{tikzpicture}
\begin{axis}[%
width=4.52in,
height=3.54in,
xlabel=\Huge Time,
scale only axis,
%separate axis lines,
% every outer x axis line/.append style={white!15!black},
% every x tick label/.append style={font=\color{white!15!black}},
xmin=1,
xmax=1050,
every axis y label/.style={at=(current axis.above origin),anchor=south},
%every axis x label/.style={at=(current axis.right of origin),anchor=west},
ymin=-1.5,
ymax=1.5,
ytick={0.01, 0.35, 1}, 
yticklabels={\Huge 0, \Huge $\bar{P} = \frac{\sigma}{\sqrt{2\pi}}$, \Huge $P_{max}$},
xtick={1050},
xticklabels={\Huge $T_s$},
grid=major,
]
\addplot [line width=1.5pt, color=black,dotted]
  table[row sep=crcr]{%
1	-0.0352403667366428\\
2	-0.0328038543655187\\
3	-0.0315920055619777\\
4	-0.0314897834783689\\
5	-0.0323451943730028\\
6	-0.033980940681956\\
7	-0.0362073819638487\\
8	-0.0388349765150509\\
9	-0.0416860598923207\\
10	-0.0446252757724698\\
11	-0.0476496219981174\\
12	-0.0510234881169469\\
13	-0.0553487628483466\\
14	-0.0614882919839537\\
15	-0.070383367246686\\
16	-0.0828560103845895\\
17	-0.0994554237486997\\
18	-0.120371141581364\\
19	-0.145413343661888\\
20	-0.174048472105405\\
21	-0.205473063442276\\
22	-0.238707988976374\\
23	-0.272697591459755\\
24	-0.306400477467701\\
25	-0.338843677386331\\
26	-0.369094409091013\\
27	-0.39616279677214\\
28	-0.418946890257929\\
29	-0.43630495638408\\
30	-0.447218643083903\\
31	-0.450960887490186\\
32	-0.447213671616349\\
33	-0.436117039200636\\
34	-0.418252129299945\\
35	-0.394572458628329\\
36	-0.366302125230984\\
37	-0.334819478228673\\
38	-0.301542228177483\\
39	-0.267824860618901\\
40	-0.234857660943247\\
41	-0.203533404695172\\
42	-0.174295336445239\\
43	-0.147064483455502\\
44	-0.121317264150087\\
45	-0.0962723230528574\\
46	-0.0710998089701174\\
47	-0.0450951009942012\\
48	-0.0177930146290968\\
49	0.0109807067005646\\
50	0.0411118467874923\\
51	0.0722398477578173\\
52	0.103824048312423\\
53	0.135219929978378\\
54	0.165753365575636\\
55	0.194773801583412\\
56	0.221658967983065\\
57	0.245777045609962\\
58	0.266466576009615\\
59	0.28308069439652\\
60	0.295076522669545\\
61	0.302103692814132\\
62	0.304063099994009\\
63	0.301126757680675\\
64	0.293721150018402\\
65	0.282482567177641\\
66	0.268195142640202\\
67	0.251722051202619\\
68	0.23393876386275\\
69	0.215674264505155\\
70	0.197653325282861\\
71	0.180419125299796\\
72	0.164244172105257\\
73	0.149088254251521\\
74	0.134645990699175\\
75	0.120459389840247\\
76	0.106043440050803\\
77	0.0909900463254164\\
78	0.0750360434556311\\
79	0.0580934761066291\\
80	0.0402474857581388\\
81	0.0217304523732866\\
82	0.00288172374321047\\
83	-0.0158983718954113\\
84	-0.0341945530930896\\
85	-0.0516396837636372\\
86	-0.0680370808864163\\
87	-0.0835248556651831\\
88	-0.0986622193763831\\
89	-0.11434922845973\\
90	-0.131625466571666\\
91	-0.151448046179553\\
92	-0.174515115441567\\
93	-0.201160771913382\\
94	-0.231323014915939\\
95	-0.264572677934842\\
96	-0.300185330664613\\
97	-0.337237075365474\\
98	-0.37470756700828\\
99	-0.411575104624113\\
100	-0.446860009433577\\
101	-0.479539847422296\\
102	-0.50836008941867\\
103	-0.531731495642033\\
104	-0.547858909272917\\
105	-0.555035929795813\\
106	-0.551953866797672\\
107	-0.537927447416891\\
108	-0.513002808923331\\
109	-0.477950621625503\\
110	-0.434167534067386\\
111	-0.383517261682411\\
112	-0.32814297773165\\
113	-0.270278208125391\\
114	-0.212077496742455\\
115	-0.155491259620242\\
116	-0.102214105630053\\
117	-0.0536997827741382\\
118	-0.0111795678683123\\
119	0.0243665897019767\\
120	0.0522716466253436\\
121	0.0722343602886339\\
122	0.0843388073881051\\
123	0.0890349354105907\\
124	0.0870852791163751\\
125	0.0794882322208531\\
126	0.0673897269584738\\
127	0.0519942126821553\\
128	0.0344837995056857\\
129	0.0159504759010131\\
130	-0.00267539952428472\\
131	-0.0207407370629656\\
132	-0.038015833834885\\
133	-0.0547630801894654\\
134	-0.0716491758521005\\
135	-0.0895444013093755\\
136	-0.109302489993281\\
137	-0.131582957452529\\
138	-0.156740467774279\\
139	-0.184783259638981\\
140	-0.215389812856595\\
141	-0.247967295852939\\
142	-0.2817340853326\\
143	-0.315811121844732\\
144	-0.349305662681079\\
145	-0.381305732709256\\
146	-0.410630069312462\\
147	-0.435384914708668\\
148	-0.4527289411886\\
149	-0.459147360249905\\
150	-0.451092498636041\\
151	-0.425666483013333\\
152	-0.381135664856494\\
153	-0.317199744752144\\
154	-0.235017927541791\\
155	-0.137038215939671\\
156	-0.0266939714640592\\
157	0.0919665067334283\\
158	0.214660527844968\\
159	0.337215262352073\\
160	0.455799172346766\\
161	0.566927822431712\\
162	0.667275177214401\\
163	0.753556227150809\\
164	0.822686484168853\\
165	0.872138075384665\\
166	0.900294878905336\\
167	0.906682617918634\\
168	0.892034890524372\\
169	0.858205565722197\\
170	0.807964019858041\\
171	0.744719148310923\\
172	0.672216820777533\\
173	0.594248707321437\\
174	0.514397016113813\\
175	0.435777342749641\\
176	0.360674350121192\\
177	0.290109696952432\\
178	0.223636477493611\\
179	0.159574296911303\\
180	0.095563709722127\\
181	0.029181918188819\\
182	-0.0415519913451103\\
183	-0.117852109406084\\
184	-0.200028910641833\\
185	-0.287512969395104\\
186	-0.37899545290207\\
187	-0.47263815611585\\
188	-0.566310588083926\\
189	-0.657818168992844\\
190	-0.74506460662362\\
191	-0.826066633836737\\
192	-0.89883936359757\\
193	-0.96133359652876\\
194	-1.0115652778858\\
195	-1.04788001637776\\
196	-1.06921490960691\\
197	-1.07527122966523\\
198	-1.06657077001674\\
199	-1.04440314659256\\
200	-1.01068957323817\\
201	-0.967795247379479\\
202	-0.918321763266946\\
203	-0.86490590240421\\
204	-0.81004451306179\\
205	-0.755963016733282\\
206	-0.704544162250647\\
207	-0.657312900624055\\
208	-0.615443152884008\\
209	-0.579757297862567\\
210	-0.550721276867719\\
211	-0.528451639020882\\
212	-0.512742872131165\\
213	-0.503114429627671\\
214	-0.498872270209535\\
215	-0.499177851410514\\
216	-0.503117597974134\\
217	-0.509766799506426\\
218	-0.518244131066914\\
219	-0.527750715153155\\
220	-0.537527110827421\\
221	-0.546593488846809\\
222	-0.553322282689155\\
223	-0.555205942296735\\
224	-0.549090948990938\\
225	-0.531748963478696\\
226	-0.500491941799577\\
227	-0.453640372867949\\
228	-0.390773737955442\\
229	-0.312763845520866\\
230	-0.221631308602613\\
231	-0.120281935571447\\
232	-0.0121815218067753\\
233	0.098980094730799\\
234	0.20959472748152\\
235	0.316354058029553\\
236	0.416302718194506\\
237	0.506745460346729\\
238	0.585183260790587\\
239	0.649415233794336\\
240	0.697757186110702\\
241	0.729251252548386\\
242	0.743788980797554\\
243	0.742125447344262\\
244	0.725793754324686\\
245	0.696945627101284\\
246	0.658149401201937\\
247	0.61217519513385\\
248	0.561792439648992\\
249	0.509593961546378\\
250	0.457791498145272\\
251	0.407851615544578\\
252	0.360018850508571\\
253	0.313081848646791\\
254	0.264644163181455\\
255	0.211759976651083\\
256	0.151631076192911\\
257	0.0821647195371414\\
258	0.00231353143739392\\
259	-0.0878080500586278\\
260	-0.186994747690699\\
261	-0.293154573785872\\
262	-0.403588519708789\\
263	-0.515294912738075\\
264	-0.625254501758353\\
265	-0.730604420663081\\
266	-0.828553699554227\\
267	-0.916081313052034\\
268	-0.989769827093519\\
269	-1.04604407557994\\
270	-1.08169838964205\\
271	-1.09443791563171\\
272	-1.08325913283764\\
273	-1.04861037640394\\
274	-0.992341185032547\\
275	-0.917485902479111\\
276	-0.827941095984308\\
277	-0.728096071552267\\
278	-0.622467198741267\\
279	-0.51537302889915\\
280	-0.410657977166594\\
281	-0.311442345577506\\
282	-0.219911766569581\\
283	-0.137229527261297\\
284	-0.0636271921158999\\
285	0.00137379601124384\\
286	0.0586984648859942\\
287	0.109475001475951\\
288	0.154842177115535\\
289	0.195817421643663\\
290	0.233222759074039\\
291	0.267658974660067\\
292	0.299515559864146\\
293	0.329003821786681\\
294	0.356201968934234\\
295	0.381103219016517\\
296	0.403660460166843\\
297	0.423823374038501\\
298	0.441565979593611\\
299	0.456904177879043\\
300	0.469904043409883\\
301	0.480682349000474\\
302	0.489401183100897\\
303	0.496258606611068\\
304	0.501477163029112\\
305	0.505291806706999\\
306	0.507938446187966\\
307	0.50964403182771\\
308	0.510618455713807\\
309	0.511050361320239\\
310	0.511141522581073\\
311	0.511250933910325\\
312	0.512123026001023\\
313	0.515009316645183\\
314	0.521541220633984\\
315	0.533423039401608\\
316	0.552100676201736\\
317	0.578507517548137\\
318	0.612925474227257\\
319	0.65496126685228\\
320	0.703616970957612\\
321	0.757425026882826\\
322	0.814616808194429\\
323	0.873298118639386\\
324	0.931607897487924\\
325	0.9877932275013\\
326	1.04008429747246\\
327	1.08640593765295\\
328	1.12421846348138\\
329	1.15070949282505\\
330	1.16323752832301\\
331	1.15979664416669\\
332	1.13935403180591\\
333	1.10200826555424\\
334	1.04897296873843\\
335	0.982421472246392\\
336	0.905240384043001\\
337	0.820740322153669\\
338	0.73236551558397\\
339	0.64343272843217\\
340	0.556901671236925\\
341	0.475149506900871\\
342	0.399764345903192\\
343	0.331453059627787\\
344	0.27012899233419\\
345	0.215131677805216\\
346	0.165484516586597\\
347	0.120125498377832\\
348	0.0780809479035699\\
349	0.0385742292769399\\
350	0.00107459361360342\\
351	-0.0347018376728271\\
352	-0.0688245927434401\\
353	-0.101200733153964\\
354	-0.131635218170699\\
355	-0.15991063683397\\
356	-0.185921348784184\\
357	-0.209843841909681\\
358	-0.232231813291755\\
359	-0.253954230148923\\
360	-0.276018111431399\\
361	-0.299368512929115\\
362	-0.324727102288112\\
363	-0.352493892587646\\
364	-0.382714445548739\\
365	-0.415102236545234\\
366	-0.449100262360041\\
367	-0.483964924743464\\
368	-0.518856999610238\\
369	-0.552927593320027\\
370	-0.585388968060192\\
371	-0.615561766731557\\
372	-0.642898328624319\\
373	-0.666992864216651\\
374	-0.687588367460805\\
375	-0.704579999474918\\
376	-0.718010516135817\\
377	-0.728056100992394\\
378	-0.735003917483043\\
379	-0.739224116242189\\
380	-0.741139453715928\\
381	-0.741195497987904\\
382	-0.739833728550567\\
383	-0.737469441839592\\
384	-0.734473391348352\\
385	-0.731120898222768\\
386	-0.727433047370774\\
387	-0.72293589826587\\
388	-0.71653713852378\\
389	-0.706668528510155\\
390	-0.691620095534375\\
391	-0.669901088778528\\
392	-0.640519859112557\\
393	-0.603141845947974\\
394	-0.558124999429599\\
395	-0.506454363257767\\
396	-0.449607029665501\\
397	-0.389379963931194\\
398	-0.327708976537929\\
399	-0.266503074358295\\
400	-0.20754748136994\\
401	-0.152562301427374\\
402	-0.103391211163085\\
403	-0.0621077024706759\\
404	-0.0308768875099581\\
405	-0.0116426712578227\\
406	-0.00580506359413621\\
407	-0.0139927715224783\\
408	-0.0359668497376295\\
409	-0.070650400230947\\
410	-0.116257317376274\\
411	-0.170484588019177\\
412	-0.230732698219853\\
413	-0.29432408308406\\
414	-0.358695798369115\\
415	-0.421530038260706\\
416	-0.480770513135201\\
417	-0.534539967708987\\
418	-0.581083019699239\\
419	-0.61883096017179\\
420	-0.646553311226952\\
421	-0.66350669964973\\
422	-0.669525293830655\\
423	-0.665036014993706\\
424	-0.651004245513922\\
425	-0.628827452757942\\
426	-0.600198260574518\\
427	-0.566957599337326\\
428	-0.530955560209164\\
429	-0.493929350511918\\
430	-0.457349278225369\\
431	-0.422119413728099\\
432	-0.38817358099587\\
433	-0.354273981849216\\
434	-0.318239085298073\\
435	-0.2774819184853\\
436	-0.229598717768079\\
437	-0.172836722833976\\
438	-0.106374306387258\\
439	-0.0304095736640676\\
440	0.0539109631173334\\
441	0.144676538508324\\
442	0.239459453621495\\
443	0.335581516312056\\
444	0.43036381265302\\
445	0.521315887927478\\
446	0.606209613962229\\
447	0.683047880271862\\
448	0.75004071289309\\
449	0.805678166639476\\
450	0.848869807811485\\
451	0.87907163094692\\
452	0.89635239146346\\
453	0.90138694785565\\
454	0.895384479307032\\
455	0.879969672918048\\
456	0.85703815652661\\
457	0.828606143962353\\
458	0.796670743571619\\
459	0.763091146447972\\
460	0.729473610653506\\
461	0.697013758811368\\
462	0.666312357349394\\
463	0.637291207171048\\
464	0.609301134807352\\
465	0.581368856053643\\
466	0.552470329163217\\
467	0.521755788343499\\
468	0.488695930668701\\
469	0.453145709398077\\
470	0.415337656017927\\
471	0.375823713173862\\
472	0.335386101166612\\
473	0.294935832973381\\
474	0.255413910656997\\
475	0.217707595483278\\
476	0.182591801427695\\
477	0.150696858403962\\
478	0.122492274457376\\
479	0.0982765997249449\\
480	0.0781733499514658\\
481	0.0621369339105014\\
482	0.0499696639807892\\
483	0.0413479732105406\\
484	0.0358545953321696\\
485	0.0330131221468414\\
486	0.0323216493070103\\
487	0.0332829520935878\\
488	0.0354292761253555\\
489	0.0383418860019277\\
490	0.041685234999105\\
491	0.0452974217913434\\
492	0.0493230782578662\\
493	0.0542795264572149\\
494	0.0609751832570959\\
495	0.0703214942118476\\
496	0.0831296461747006\\
497	0.0999517937925001\\
498	0.120989581632888\\
499	0.146070566968916\\
500	0.174680739947207\\
501	0.206036040484635\\
502	0.239175002241243\\
503	0.27305691670737\\
504	0.306652164690864\\
505	0.338996310372155\\
506	0.369162082097369\\
507	0.396162490897778\\
508	0.418896347067514\\
509	0.436221078598118\\
510	0.447116401738103\\
511	0.45085268071249\\
512	0.447109060023701\\
513	0.436022774376649\\
514	0.41817239221981\\
515	0.394509231670893\\
516	0.366255632232249\\
517	0.33478863990958\\
518	0.301525093646824\\
519	0.267818988340622\\
520	0.234860433778247\\
521	0.203542272926162\\
522	0.174307989889094\\
523	0.147078958892002\\
524	0.121331997530129\\
525	0.0962861575685196\\
526	0.071111970395942\\
527	0.045105151075549\\
528	0.017800792012746\\
529	-0.010975150150185\\
530	-0.0411083082378435\\
531	-0.0722380299368309\\
532	-0.103823607633319\\
533	-0.135220514945489\\
534	-0.165754645248331\\
535	-0.194775484685027\\
536	-0.22166081378529\\
537	-0.245778868032282\\
538	-0.266468242489192\\
539	-0.283082121050593\\
540	-0.295077667133479\\
541	-0.302104545933222\\
542	-0.304063677309575\\
543	-0.301127091407825\\
544	-0.29372128198396\\
545	-0.282482542985448\\
546	-0.268195007164954\\
547	-0.251721845385415\\
548	-0.233938522690768\\
549	-0.215674016001002\\
550	-0.197653090301059\\
551	-0.180418917922818\\
552	-0.16424400044353\\
553	-0.149088121475263\\
554	-0.134645896159928\\
555	-0.120459330172404\\
556	-0.106043410175231\\
557	-0.0909900403061776\\
558	-0.0750360551854428\\
559	-0.0580934998163822\\
560	-0.0402475163604118\\
561	-0.0217304856596043\\
562	-0.0028817564604336\\
563	0.0158983420620371\\
564	0.0341945276032243\\
565	0.0516396633468902\\
566	0.0680370656881779\\
567	0.0835248453964836\\
568	0.098662213455333\\
569	0.11434922613616\\
570	0.131625467030166\\
571	0.151448048619069\\
572	0.174515119131988\\
573	0.201160776231187\\
574	0.231323019361959\\
575	0.264572682137739\\
576	0.300185334373728\\
577	0.337237078436532\\
578	0.374707569384987\\
579	0.411575106318169\\
580	0.446860010505062\\
581	0.479539847961796\\
582	0.508360089532027\\
583	0.531731495438138\\
584	0.547858908854637\\
585	0.555035929253878\\
586	0.551953866207156\\
587	0.537927446835855\\
588	0.513002808393117\\
589	0.477950621172216\\
590	0.434167533704122\\
591	0.38351726141185\\
592	0.328142977548723\\
593	0.2702782080198\\
594	0.212077496700892\\
595	0.155491259628224\\
596	0.102214105673343\\
597	0.05369978283975\\
598	0.0111795679451527\\
599	-0.0243665896227944\\
600	-0.0522716465504352\\
601	-0.0722343602224685\\
602	-0.0843388073332613\\
603	-0.0890349353680802\\
604	-0.087085279086001\\
605	-0.0794882322015556\\
606	-0.0673897269486493\\
607	-0.0519942126799281\\
608	-0.0344837995091243\\
609	-0.0159504759082928\\
610	0.00267539951477454\\
611	0.0207407370525577\\
612	0.0380158338246098\\
613	0.0547630801800578\\
614	0.0716491758440241\\
615	0.0895444013028642\\
616	0.109302489988382\\
617	0.131582957449155\\
618	0.156740467772249\\
619	0.18478325963806\\
620	0.215389812856529\\
621	0.247967295853479\\
622	0.281734085333518\\
623	0.315811121845835\\
624	0.349305662682213\\
625	0.381305732710308\\
626	0.410630069313356\\
627	0.435384914709363\\
628	0.452728941189082\\
629	0.459147360250181\\
630	0.451092498636132\\
631	0.42566648301327\\
632	0.381135664856312\\
633	0.317199744751881\\
634	0.235017927541478\\
635	0.137038215939339\\
636	0.0266939714637309\\
637	-0.0919665067337347\\
638	-0.214660527845239\\
639	-0.337215262352303\\
640	-0.455799172346949\\
641	-0.56692782243185\\
642	-0.667275177214496\\
643	-0.753556227150864\\
644	-0.822686484168875\\
645	-0.872138075384659\\
646	-0.900294878905307\\
647	-0.906682617918585\\
648	-0.89203489052431\\
649	-0.858205565722124\\
650	-0.80796401985796\\
651	-0.744719148310836\\
652	-0.672216820777443\\
653	-0.594248707321344\\
654	-0.514397016113718\\
655	-0.435777342749546\\
656	-0.360674350121098\\
657	-0.29010969695234\\
658	-0.223636477493521\\
659	-0.159574296911215\\
660	-0.0955637097220432\\
661	-0.029181918188739\\
662	0.0415519913451864\\
663	0.117852109406156\\
664	0.2000289106419\\
665	0.287512969395168\\
666	0.37899545290213\\
667	0.472638156115908\\
668	0.566310588083982\\
669	0.657818168992899\\
670	0.745064606623675\\
671	0.826066633836793\\
672	0.898839363597627\\
673	0.961333596528819\\
674	1.01156527788586\\
675	1.04788001637782\\
676	1.06921490960697\\
677	1.0752712296653\\
678	1.0665707700168\\
679	1.04440314659262\\
680	1.01068957323823\\
681	0.96779524737954\\
682	0.918321763267003\\
683	0.864905902404259\\
684	0.810044513061832\\
685	0.755963016733315\\
686	0.70454416225067\\
687	0.657312900624066\\
688	0.615443152884008\\
689	0.579757297862554\\
690	0.550721276867693\\
691	0.528451639020843\\
692	0.512742872131114\\
693	0.503114429627608\\
694	0.498872270209461\\
695	0.499177851410428\\
696	0.503117597974039\\
697	0.509766799506322\\
698	0.518244131066802\\
699	0.527750715153036\\
700	0.537527110827296\\
701	0.546593488846679\\
702	0.55332228268902\\
703	0.555205942296595\\
704	0.549090948990795\\
705	0.53174896347855\\
706	0.500491941799428\\
707	0.453640372867797\\
708	0.390773737955288\\
709	0.312763845520711\\
710	0.221631308602459\\
711	0.120281935571292\\
712	0.0121815218066224\\
713	-0.0989800947309485\\
714	-0.209594727481666\\
715	-0.316354058029692\\
716	-0.416302718194638\\
717	-0.506745460346852\\
718	-0.5851832607907\\
719	-0.649415233794438\\
720	-0.697757186110793\\
721	-0.729251252548465\\
722	-0.743788980797623\\
723	-0.742125447344321\\
724	-0.725793754324737\\
725	-0.696945627101327\\
726	-0.658149401201976\\
727	-0.612175195133886\\
728	-0.561792439649027\\
729	-0.509593961546415\\
730	-0.457791498145313\\
731	-0.407851615544623\\
732	-0.360018850508622\\
733	-0.313081848646848\\
734	-0.264644163181519\\
735	-0.211759976651153\\
736	-0.151631076192987\\
737	-0.082164719537222\\
738	-0.00231353143747802\\
739	0.0878080500585416\\
740	0.186994747690612\\
741	0.293154573785785\\
742	0.403588519708704\\
743	0.515294912737992\\
744	0.625254501758271\\
745	0.730604420663002\\
746	0.828553699554149\\
747	0.916081313051956\\
748	0.98976982709344\\
749	1.04604407557986\\
750	1.08169838964196\\
751	1.09443791563162\\
752	1.08325913283754\\
753	1.04861037640384\\
754	0.992341185032437\\
755	0.917485902478994\\
756	0.827941095984183\\
757	0.728096071552135\\
758	0.622467198741129\\
759	0.515373028899008\\
760	0.410657977166449\\
761	0.311442345577359\\
762	0.219911766569434\\
763	0.137229527261153\\
764	0.0636271921157609\\
765	-0.00137379601137605\\
766	-0.0586984648861178\\
767	-0.109475001476064\\
768	-0.154842177115636\\
769	-0.195817421643752\\
770	-0.233222759074113\\
771	-0.267658974660127\\
772	-0.299515559864191\\
773	-0.329003821786711\\
774	-0.356201968934249\\
775	-0.381103219016517\\
776	-0.403660460166828\\
777	-0.423823374038473\\
778	-0.441565979593569\\
779	-0.45690417787899\\
780	-0.469904043409818\\
781	-0.480682349000399\\
782	-0.489401183100814\\
783	-0.496258606610977\\
784	-0.501477163029016\\
785	-0.5052918067069\\
786	-0.507938446187864\\
787	-0.509644031827608\\
788	-0.510618455713707\\
789	-0.511050361320142\\
790	-0.511141522580981\\
791	-0.511250933910241\\
792	-0.512123026000949\\
793	-0.515009316645121\\
794	-0.521541220633936\\
795	-0.533423039401576\\
796	-0.552100676201722\\
797	-0.578507517548142\\
798	-0.612925474227284\\
799	-0.654961266852328\\
800	-0.703616970957684\\
801	-0.75742502688292\\
802	-0.814616808194546\\
803	-0.873298118639526\\
804	-0.931607897488085\\
805	-0.987793227501481\\
806	-1.04008429747266\\
807	-1.08640593765316\\
808	-1.1242184634816\\
809	-1.15070949282529\\
810	-1.16323752832326\\
811	-1.15979664416694\\
812	-1.13935403180617\\
813	-1.10200826555449\\
814	-1.04897296873868\\
815	-0.982421472246636\\
816	-0.905240384043237\\
817	-0.820740322153896\\
818	-0.732365515584186\\
819	-0.643432728432374\\
820	-0.556901671237117\\
821	-0.475149506901051\\
822	-0.399764345903361\\
823	-0.331453059627946\\
824	-0.27012899233434\\
825	-0.215131677805357\\
826	-0.165484516586732\\
827	-0.120125498377963\\
828	-0.0780809479036967\\
829	-0.0385742292770643\\
830	-0.00107459361372698\\
831	0.0347018376727047\\
832	0.0688245927433174\\
833	0.101200733153842\\
834	0.131635218170577\\
835	0.15991063683385\\
836	0.185921348784066\\
837	0.209843841909568\\
838	0.232231813291648\\
839	0.253954230148822\\
840	0.276018111431307\\
841	0.299368512929033\\
842	0.324727102288041\\
843	0.352493892587586\\
844	0.382714445548691\\
845	0.415102236545198\\
846	0.449100262360017\\
847	0.48396492474345\\
848	0.518856999610233\\
849	0.552927593320031\\
850	0.585388968060203\\
851	0.615561766731574\\
852	0.642898328624342\\
853	0.666992864216678\\
854	0.687588367460836\\
855	0.704579999474953\\
856	0.718010516135856\\
857	0.728056100992439\\
858	0.735003917483093\\
859	0.739224116242246\\
860	0.741139453715992\\
861	0.741195497987977\\
862	0.739833728550647\\
863	0.737469441839679\\
864	0.734473391348446\\
865	0.731120898222868\\
866	0.727433047370878\\
867	0.722935898265974\\
868	0.716537138523883\\
869	0.706668528510253\\
870	0.691620095534464\\
871	0.669901088778605\\
872	0.640519859112621\\
873	0.603141845948021\\
874	0.558124999429627\\
875	0.506454363257774\\
876	0.449607029665486\\
877	0.389379963931158\\
878	0.327708976537872\\
879	0.266503074358218\\
880	0.207547481369846\\
881	0.152562301427265\\
882	0.103391211162962\\
883	0.0621077024705431\\
884	0.0308768875098172\\
885	0.0116426712576768\\
886	0.00580506359398707\\
887	0.0139927715223282\\
888	0.0359668497374802\\
889	0.0706504002307993\\
890	0.11625731737613\\
891	0.170484588019038\\
892	0.230732698219718\\
893	0.294324083083932\\
894	0.358695798368995\\
895	0.421530038260594\\
896	0.4807705131351\\
897	0.534539967708897\\
898	0.581083019699163\\
899	0.618830960171729\\
900	0.646553311226908\\
901	0.663506699649704\\
902	0.669525293830649\\
903	0.665036014993722\\
904	0.651004245513959\\
905	0.628827452758\\
906	0.600198260574596\\
907	0.566957599337423\\
908	0.530955560209277\\
909	0.493929350512045\\
910	0.457349278225505\\
911	0.422119413728242\\
912	0.388173580996015\\
913	0.354273981849358\\
914	0.31823908529821\\
915	0.277481918485426\\
916	0.229598717768191\\
917	0.172836722834071\\
918	0.106374306387333\\
919	0.0304095736641224\\
920	-0.0539109631172999\\
921	-0.144676538508312\\
922	-0.239459453621502\\
923	-0.335581516312081\\
924	-0.430363812653059\\
925	-0.52131588792753\\
926	-0.606209613962291\\
927	-0.68304788027193\\
928	-0.750040712893161\\
929	-0.805678166639547\\
930	-0.848869807811552\\
931	-0.879071630946982\\
932	-0.896352391463516\\
933	-0.901386947855698\\
934	-0.895384479307073\\
935	-0.87996967291808\\
936	-0.857038156526634\\
937	-0.828606143962369\\
938	-0.796670743571628\\
939	-0.763091146447976\\
940	-0.729473610653505\\
941	-0.697013758811363\\
942	-0.666312357349387\\
943	-0.637291207171038\\
944	-0.609301134807341\\
945	-0.58136885605363\\
946	-0.552470329163202\\
947	-0.521755788343482\\
948	-0.488695930668683\\
949	-0.453145709398057\\
950	-0.415337656017904\\
951	-0.375823713173836\\
952	-0.335386101166584\\
953	-0.29493583297335\\
954	-0.255413910656964\\
955	-0.217707595483243\\
956	-0.182591801427658\\
957	-0.150696858403925\\
958	-0.122492274457339\\
959	-0.0982765997249075\\
960	-0.0781733499514297\\
961	-0.0621369339104666\\
962	-0.0499696639807561\\
963	-0.0413479732105085\\
964	-0.0358545953321395\\
965	-0.0330131221468128\\
966	-0.0323216493069837\\
967	-0.0332829520935626\\
968	-0.035429276125332\\
969	-0.0383418860019053\\
970	-0.0416852349990852\\
971	-0.0452974217913252\\
972	-0.0493230782578519\\
973	-0.0542795264572039\\
974	-0.0609751832570911\\
975	-0.0703214942118483\\
976	-0.0831296461747114\\
977	-0.0999517937925194\\
978	-0.120989581632921\\
979	-0.146070566968961\\
980	-0.174680739947269\\
981	-0.206036040484712\\
982	-0.23917500224134\\
983	-0.273056916707483\\
984	-0.306652164691\\
985	-0.338996310372305\\
986	-0.369162082097542\\
987	-0.396162490897963\\
988	-0.41889634706772\\
989	-0.436221078598331\\
990	-0.447116401738335\\
991	-0.450852680712722\\
992	-0.44710906002395\\
993	-0.436022774376888\\
994	-0.418172392220066\\
995	-0.394509231671127\\
996	-0.366255632232502\\
997	-0.334788639909797\\
998	-0.301525093647064\\
999	-0.267818988340807\\
1000	-0.234860433778467\\
1001	-0.203542272926303\\
1002	-0.17430798988929\\
1003	-0.147078958892082\\
1004	-0.121331997530306\\
1005	-0.0962861575685227\\
1006	-0.0711119703961128\\
1007	-0.0451051510754476\\
1008	-0.0178007920129444\\
1009	0.0109751501504374\\
1010	0.0411083082375464\\
1011	0.0722380299373302\\
1012	0.103823607632769\\
1013	0.135220514946454\\
1014	0.165754645247167\\
1015	0.194775484687005\\
1016	0.221660813782587\\
1017	0.245778868036812\\
1018	0.26646824248222\\
1019	0.283082121062631\\
1020	0.295077667112676\\
1021	0.302104545972175\\
1022	0.304063677232649\\
1023	0.301127091573691\\
1024	0.29372128158877\\
1025	0.282482544061506\\
1026	0.268195003665684\\
1027	0.251721860037789\\
1028	0.233938431500787\\
1029	0.21567526348113\\
1030	0.197708196619704\\
1031	0.180864784116396\\
1032	0.166223090235151\\
1033	0.155221194761011\\
1034	0.149540744271029\\
1035	0.150827615189522\\
1036	0.160394607108879\\
1037	0.178999957977964\\
1038	0.206735672305094\\
1039	0.243024134161719\\
1040	0.286701968931336\\
1041	0.336162161571369\\
1042	0.389524868223126\\
1043	0.444811362675721\\
1044	0.500100961744264\\
1045	0.553647102003322\\
1046	0.603923431307901\\
1047	0.649606546340879\\
1048	0.689558105829513\\
1049	0.722856415525642\\
1050	0.748862315446572\\
};
\addlegendentry{\Huge Clipped}
\addplot [line width=1.5pt, color=black,solid]
  table[row sep=crcr]{%
1	0\\
2	0\\
3	0\\
4	0\\
5	0\\
6	0\\
7	0\\
8	0\\
9	0\\
10	0\\
11	0\\
12	0\\
13	0\\
14	0\\
15	0\\
16	0\\
17	0\\
18	0\\
19	0\\
20	0\\
21	0\\
22	0\\
23	0\\
24	0\\
25	0\\
26	0\\
27	0\\
28	0\\
29	0\\
30	0\\
31	0\\
32	0\\
33	0\\
34	0\\
35	0\\
36	0\\
37	0\\
38	0\\
39	0\\
40	0\\
41	0\\
42	0\\
43	0\\
44	0\\
45	0\\
46	0\\
47	0\\
48	0\\
49	0.0109807067005646\\
50	0.0411118467874923\\
51	0.0722398477578173\\
52	0.103824048312423\\
53	0.135219929978378\\
54	0.165753365575636\\
55	0.194773801583412\\
56	0.221658967983065\\
57	0.245777045609962\\
58	0.266466576009615\\
59	0.28308069439652\\
60	0.295076522669545\\
61	0.302103692814132\\
62	0.304063099994009\\
63	0.301126757680675\\
64	0.293721150018402\\
65	0.282482567177641\\
66	0.268195142640202\\
67	0.251722051202619\\
68	0.23393876386275\\
69	0.215674264505155\\
70	0.197653325282861\\
71	0.180419125299796\\
72	0.164244172105257\\
73	0.149088254251521\\
74	0.134645990699175\\
75	0.120459389840247\\
76	0.106043440050803\\
77	0.0909900463254164\\
78	0.0750360434556311\\
79	0.0580934761066291\\
80	0.0402474857581388\\
81	0.0217304523732866\\
82	0.00288172374321047\\
83	0\\
84	0\\
85	0\\
86	0\\
87	0\\
88	0\\
89	0\\
90	0\\
91	0\\
92	0\\
93	0\\
94	0\\
95	0\\
96	0\\
97	0\\
98	0\\
99	0\\
100	0\\
101	0\\
102	0\\
103	0\\
104	0\\
105	0\\
106	0\\
107	0\\
108	0\\
109	0\\
110	0\\
111	0\\
112	0\\
113	0\\
114	0\\
115	0\\
116	0\\
117	0\\
118	0\\
119	0.0243665897019767\\
120	0.0522716466253436\\
121	0.0722343602886339\\
122	0.0843388073881051\\
123	0.0890349354105907\\
124	0.0870852791163751\\
125	0.0794882322208531\\
126	0.0673897269584738\\
127	0.0519942126821553\\
128	0.0344837995056857\\
129	0.0159504759010131\\
130	0\\
131	0\\
132	0\\
133	0\\
134	0\\
135	0\\
136	0\\
137	0\\
138	0\\
139	0\\
140	0\\
141	0\\
142	0\\
143	0\\
144	0\\
145	0\\
146	0\\
147	0\\
148	0\\
149	0\\
150	0\\
151	0\\
152	0\\
153	0\\
154	0\\
155	0\\
156	0\\
157	0.0919665067334283\\
158	0.214660527844968\\
159	0.337215262352073\\
160	0.455799172346766\\
161	0.566927822431712\\
162	0.667275177214401\\
163	0.753556227150809\\
164	0.822686484168853\\
165	0.872138075384665\\
166	0.900294878905336\\
167	0.906682617918634\\
168	0.892034890524372\\
169	0.858205565722197\\
170	0.807964019858041\\
171	0.744719148310923\\
172	0.672216820777533\\
173	0.594248707321437\\
174	0.514397016113813\\
175	0.435777342749641\\
176	0.360674350121192\\
177	0.290109696952432\\
178	0.223636477493611\\
179	0.159574296911303\\
180	0.095563709722127\\
181	0.029181918188819\\
182	0\\
183	0\\
184	0\\
185	0\\
186	0\\
187	0\\
188	0\\
189	0\\
190	0\\
191	0\\
192	0\\
193	0\\
194	0\\
195	0\\
196	0\\
197	0\\
198	0\\
199	0\\
200	0\\
201	0\\
202	0\\
203	0\\
204	0\\
205	0\\
206	0\\
207	0\\
208	0\\
209	0\\
210	0\\
211	0\\
212	0\\
213	0\\
214	0\\
215	0\\
216	0\\
217	0\\
218	0\\
219	0\\
220	0\\
221	0\\
222	0\\
223	0\\
224	0\\
225	0\\
226	0\\
227	0\\
228	0\\
229	0\\
230	0\\
231	0\\
232	0\\
233	0.098980094730799\\
234	0.20959472748152\\
235	0.316354058029553\\
236	0.416302718194506\\
237	0.506745460346729\\
238	0.585183260790587\\
239	0.649415233794336\\
240	0.697757186110702\\
241	0.729251252548386\\
242	0.743788980797554\\
243	0.742125447344262\\
244	0.725793754324686\\
245	0.696945627101284\\
246	0.658149401201937\\
247	0.61217519513385\\
248	0.561792439648992\\
249	0.509593961546378\\
250	0.457791498145272\\
251	0.407851615544578\\
252	0.360018850508571\\
253	0.313081848646791\\
254	0.264644163181455\\
255	0.211759976651083\\
256	0.151631076192911\\
257	0.0821647195371414\\
258	0.00231353143739392\\
259	0\\
260	0\\
261	0\\
262	0\\
263	0\\
264	0\\
265	0\\
266	0\\
267	0\\
268	0\\
269	0\\
270	0\\
271	0\\
272	0\\
273	0\\
274	0\\
275	0\\
276	0\\
277	0\\
278	0\\
279	0\\
280	0\\
281	0\\
282	0\\
283	0\\
284	0\\
285	0.00137379601124384\\
286	0.0586984648859942\\
287	0.109475001475951\\
288	0.154842177115535\\
289	0.195817421643663\\
290	0.233222759074039\\
291	0.267658974660067\\
292	0.299515559864146\\
293	0.329003821786681\\
294	0.356201968934234\\
295	0.381103219016517\\
296	0.403660460166843\\
297	0.423823374038501\\
298	0.441565979593611\\
299	0.456904177879043\\
300	0.469904043409883\\
301	0.480682349000474\\
302	0.489401183100897\\
303	0.496258606611068\\
304	0.501477163029112\\
305	0.505291806706999\\
306	0.507938446187966\\
307	0.50964403182771\\
308	0.510618455713807\\
309	0.511050361320239\\
310	0.511141522581073\\
311	0.511250933910325\\
312	0.512123026001023\\
313	0.515009316645183\\
314	0.521541220633984\\
315	0.533423039401608\\
316	0.552100676201736\\
317	0.578507517548137\\
318	0.612925474227257\\
319	0.65496126685228\\
320	0.703616970957612\\
321	0.757425026882826\\
322	0.814616808194429\\
323	0.873298118639386\\
324	0.931607897487924\\
325	0.9877932275013\\
326	1\\
327	1\\
328	1\\
329	1\\
330	1\\
331	1\\
332	1\\
333	1\\
334	1\\
335	0.982421472246392\\
336	0.905240384043001\\
337	0.820740322153669\\
338	0.73236551558397\\
339	0.64343272843217\\
340	0.556901671236925\\
341	0.475149506900871\\
342	0.399764345903192\\
343	0.331453059627787\\
344	0.27012899233419\\
345	0.215131677805216\\
346	0.165484516586597\\
347	0.120125498377832\\
348	0.0780809479035699\\
349	0.0385742292769399\\
350	0.00107459361360342\\
351	0\\
352	0\\
353	0\\
354	0\\
355	0\\
356	0\\
357	0\\
358	0\\
359	0\\
360	0\\
361	0\\
362	0\\
363	0\\
364	0\\
365	0\\
366	0\\
367	0\\
368	0\\
369	0\\
370	0\\
371	0\\
372	0\\
373	0\\
374	0\\
375	0\\
376	0\\
377	0\\
378	0\\
379	0\\
380	0\\
381	0\\
382	0\\
383	0\\
384	0\\
385	0\\
386	0\\
387	0\\
388	0\\
389	0\\
390	0\\
391	0\\
392	0\\
393	0\\
394	0\\
395	0\\
396	0\\
397	0\\
398	0\\
399	0\\
400	0\\
401	0\\
402	0\\
403	0\\
404	0\\
405	0\\
406	0\\
407	0\\
408	0\\
409	0\\
410	0\\
411	0\\
412	0\\
413	0\\
414	0\\
415	0\\
416	0\\
417	0\\
418	0\\
419	0\\
420	0\\
421	0\\
422	0\\
423	0\\
424	0\\
425	0\\
426	0\\
427	0\\
428	0\\
429	0\\
430	0\\
431	0\\
432	0\\
433	0\\
434	0\\
435	0\\
436	0\\
437	0\\
438	0\\
439	0\\
440	0.0539109631173334\\
441	0.144676538508324\\
442	0.239459453621495\\
443	0.335581516312056\\
444	0.43036381265302\\
445	0.521315887927478\\
446	0.606209613962229\\
447	0.683047880271862\\
448	0.75004071289309\\
449	0.805678166639476\\
450	0.848869807811485\\
451	0.87907163094692\\
452	0.89635239146346\\
453	0.90138694785565\\
454	0.895384479307032\\
455	0.879969672918048\\
456	0.85703815652661\\
457	0.828606143962353\\
458	0.796670743571619\\
459	0.763091146447972\\
460	0.729473610653506\\
461	0.697013758811368\\
462	0.666312357349394\\
463	0.637291207171048\\
464	0.609301134807352\\
465	0.581368856053643\\
466	0.552470329163217\\
467	0.521755788343499\\
468	0.488695930668701\\
469	0.453145709398077\\
470	0.415337656017927\\
471	0.375823713173862\\
472	0.335386101166612\\
473	0.294935832973381\\
474	0.255413910656997\\
475	0.217707595483278\\
476	0.182591801427695\\
477	0.150696858403962\\
478	0.122492274457376\\
479	0.0982765997249449\\
480	0.0781733499514658\\
481	0.0621369339105014\\
482	0.0499696639807892\\
483	0.0413479732105406\\
484	0.0358545953321696\\
485	0.0330131221468414\\
486	0.0323216493070103\\
487	0.0332829520935878\\
488	0.0354292761253555\\
489	0.0383418860019277\\
490	0.041685234999105\\
491	0.0452974217913434\\
492	0.0493230782578662\\
493	0.0542795264572149\\
494	0.0609751832570959\\
495	0.0703214942118476\\
496	0.0831296461747006\\
497	0.0999517937925001\\
498	0.120989581632888\\
499	0.146070566968916\\
500	0.174680739947207\\
501	0.206036040484635\\
502	0.239175002241243\\
503	0.27305691670737\\
504	0.306652164690864\\
505	0.338996310372155\\
506	0.369162082097369\\
507	0.396162490897778\\
508	0.418896347067514\\
509	0.436221078598118\\
510	0.447116401738103\\
511	0.45085268071249\\
512	0.447109060023701\\
513	0.436022774376649\\
514	0.41817239221981\\
515	0.394509231670893\\
516	0.366255632232249\\
517	0.33478863990958\\
518	0.301525093646824\\
519	0.267818988340622\\
520	0.234860433778247\\
521	0.203542272926162\\
522	0.174307989889094\\
523	0.147078958892002\\
524	0.121331997530129\\
525	0.0962861575685196\\
526	0.071111970395942\\
527	0.045105151075549\\
528	0.017800792012746\\
529	0\\
530	0\\
531	0\\
532	0\\
533	0\\
534	0\\
535	0\\
536	0\\
537	0\\
538	0\\
539	0\\
540	0\\
541	0\\
542	0\\
543	0\\
544	0\\
545	0\\
546	0\\
547	0\\
548	0\\
549	0\\
550	0\\
551	0\\
552	0\\
553	0\\
554	0\\
555	0\\
556	0\\
557	0\\
558	0\\
559	0\\
560	0\\
561	0\\
562	0\\
563	0.0158983420620371\\
564	0.0341945276032243\\
565	0.0516396633468902\\
566	0.0680370656881779\\
567	0.0835248453964836\\
568	0.098662213455333\\
569	0.11434922613616\\
570	0.131625467030166\\
571	0.151448048619069\\
572	0.174515119131988\\
573	0.201160776231187\\
574	0.231323019361959\\
575	0.264572682137739\\
576	0.300185334373728\\
577	0.337237078436532\\
578	0.374707569384987\\
579	0.411575106318169\\
580	0.446860010505062\\
581	0.479539847961796\\
582	0.508360089532027\\
583	0.531731495438138\\
584	0.547858908854637\\
585	0.555035929253878\\
586	0.551953866207156\\
587	0.537927446835855\\
588	0.513002808393117\\
589	0.477950621172216\\
590	0.434167533704122\\
591	0.38351726141185\\
592	0.328142977548723\\
593	0.2702782080198\\
594	0.212077496700892\\
595	0.155491259628224\\
596	0.102214105673343\\
597	0.05369978283975\\
598	0.0111795679451527\\
599	0\\
600	0\\
601	0\\
602	0\\
603	0\\
604	0\\
605	0\\
606	0\\
607	0\\
608	0\\
609	0\\
610	0.00267539951477454\\
611	0.0207407370525577\\
612	0.0380158338246098\\
613	0.0547630801800578\\
614	0.0716491758440241\\
615	0.0895444013028642\\
616	0.109302489988382\\
617	0.131582957449155\\
618	0.156740467772249\\
619	0.18478325963806\\
620	0.215389812856529\\
621	0.247967295853479\\
622	0.281734085333518\\
623	0.315811121845835\\
624	0.349305662682213\\
625	0.381305732710308\\
626	0.410630069313356\\
627	0.435384914709363\\
628	0.452728941189082\\
629	0.459147360250181\\
630	0.451092498636132\\
631	0.42566648301327\\
632	0.381135664856312\\
633	0.317199744751881\\
634	0.235017927541478\\
635	0.137038215939339\\
636	0.0266939714637309\\
637	0\\
638	0\\
639	0\\
640	0\\
641	0\\
642	0\\
643	0\\
644	0\\
645	0\\
646	0\\
647	0\\
648	0\\
649	0\\
650	0\\
651	0\\
652	0\\
653	0\\
654	0\\
655	0\\
656	0\\
657	0\\
658	0\\
659	0\\
660	0\\
661	0\\
662	0.0415519913451864\\
663	0.117852109406156\\
664	0.2000289106419\\
665	0.287512969395168\\
666	0.37899545290213\\
667	0.472638156115908\\
668	0.566310588083982\\
669	0.657818168992899\\
670	0.745064606623675\\
671	0.826066633836793\\
672	0.898839363597627\\
673	0.961333596528819\\
674	1\\
675	1\\
676	1\\
677	1\\
678	1\\
679	1\\
680	1\\
681	0.96779524737954\\
682	0.918321763267003\\
683	0.864905902404259\\
684	0.810044513061832\\
685	0.755963016733315\\
686	0.70454416225067\\
687	0.657312900624066\\
688	0.615443152884008\\
689	0.579757297862554\\
690	0.550721276867693\\
691	0.528451639020843\\
692	0.512742872131114\\
693	0.503114429627608\\
694	0.498872270209461\\
695	0.499177851410428\\
696	0.503117597974039\\
697	0.509766799506322\\
698	0.518244131066802\\
699	0.527750715153036\\
700	0.537527110827296\\
701	0.546593488846679\\
702	0.55332228268902\\
703	0.555205942296595\\
704	0.549090948990795\\
705	0.53174896347855\\
706	0.500491941799428\\
707	0.453640372867797\\
708	0.390773737955288\\
709	0.312763845520711\\
710	0.221631308602459\\
711	0.120281935571292\\
712	0.0121815218066224\\
713	0\\
714	0\\
715	0\\
716	0\\
717	0\\
718	0\\
719	0\\
720	0\\
721	0\\
722	0\\
723	0\\
724	0\\
725	0\\
726	0\\
727	0\\
728	0\\
729	0\\
730	0\\
731	0\\
732	0\\
733	0\\
734	0\\
735	0\\
736	0\\
737	0\\
738	0\\
739	0.0878080500585416\\
740	0.186994747690612\\
741	0.293154573785785\\
742	0.403588519708704\\
743	0.515294912737992\\
744	0.625254501758271\\
745	0.730604420663002\\
746	0.828553699554149\\
747	0.916081313051956\\
748	0.98976982709344\\
749	1\\
750	1\\
751	1\\
752	1\\
753	1\\
754	0.992341185032437\\
755	0.917485902478994\\
756	0.827941095984183\\
757	0.728096071552135\\
758	0.622467198741129\\
759	0.515373028899008\\
760	0.410657977166449\\
761	0.311442345577359\\
762	0.219911766569434\\
763	0.137229527261153\\
764	0.0636271921157609\\
765	0\\
766	0\\
767	0\\
768	0\\
769	0\\
770	0\\
771	0\\
772	0\\
773	0\\
774	0\\
775	0\\
776	0\\
777	0\\
778	0\\
779	0\\
780	0\\
781	0\\
782	0\\
783	0\\
784	0\\
785	0\\
786	0\\
787	0\\
788	0\\
789	0\\
790	0\\
791	0\\
792	0\\
793	0\\
794	0\\
795	0\\
796	0\\
797	0\\
798	0\\
799	0\\
800	0\\
801	0\\
802	0\\
803	0\\
804	0\\
805	0\\
806	0\\
807	0\\
808	0\\
809	0\\
810	0\\
811	0\\
812	0\\
813	0\\
814	0\\
815	0\\
816	0\\
817	0\\
818	0\\
819	0\\
820	0\\
821	0\\
822	0\\
823	0\\
824	0\\
825	0\\
826	0\\
827	0\\
828	0\\
829	0\\
830	0\\
831	0.0347018376727047\\
832	0.0688245927433174\\
833	0.101200733153842\\
834	0.131635218170577\\
835	0.15991063683385\\
836	0.185921348784066\\
837	0.209843841909568\\
838	0.232231813291648\\
839	0.253954230148822\\
840	0.276018111431307\\
841	0.299368512929033\\
842	0.324727102288041\\
843	0.352493892587586\\
844	0.382714445548691\\
845	0.415102236545198\\
846	0.449100262360017\\
847	0.48396492474345\\
848	0.518856999610233\\
849	0.552927593320031\\
850	0.585388968060203\\
851	0.615561766731574\\
852	0.642898328624342\\
853	0.666992864216678\\
854	0.687588367460836\\
855	0.704579999474953\\
856	0.718010516135856\\
857	0.728056100992439\\
858	0.735003917483093\\
859	0.739224116242246\\
860	0.741139453715992\\
861	0.741195497987977\\
862	0.739833728550647\\
863	0.737469441839679\\
864	0.734473391348446\\
865	0.731120898222868\\
866	0.727433047370878\\
867	0.722935898265974\\
868	0.716537138523883\\
869	0.706668528510253\\
870	0.691620095534464\\
871	0.669901088778605\\
872	0.640519859112621\\
873	0.603141845948021\\
874	0.558124999429627\\
875	0.506454363257774\\
876	0.449607029665486\\
877	0.389379963931158\\
878	0.327708976537872\\
879	0.266503074358218\\
880	0.207547481369846\\
881	0.152562301427265\\
882	0.103391211162962\\
883	0.0621077024705431\\
884	0.0308768875098172\\
885	0.0116426712576768\\
886	0.00580506359398707\\
887	0.0139927715223282\\
888	0.0359668497374802\\
889	0.0706504002307993\\
890	0.11625731737613\\
891	0.170484588019038\\
892	0.230732698219718\\
893	0.294324083083932\\
894	0.358695798368995\\
895	0.421530038260594\\
896	0.4807705131351\\
897	0.534539967708897\\
898	0.581083019699163\\
899	0.618830960171729\\
900	0.646553311226908\\
901	0.663506699649704\\
902	0.669525293830649\\
903	0.665036014993722\\
904	0.651004245513959\\
905	0.628827452758\\
906	0.600198260574596\\
907	0.566957599337423\\
908	0.530955560209277\\
909	0.493929350512045\\
910	0.457349278225505\\
911	0.422119413728242\\
912	0.388173580996015\\
913	0.354273981849358\\
914	0.31823908529821\\
915	0.277481918485426\\
916	0.229598717768191\\
917	0.172836722834071\\
918	0.106374306387333\\
919	0.0304095736641224\\
920	0\\
921	0\\
922	0\\
923	0\\
924	0\\
925	0\\
926	0\\
927	0\\
928	0\\
929	0\\
930	0\\
931	0\\
932	0\\
933	0\\
934	0\\
935	0\\
936	0\\
937	0\\
938	0\\
939	0\\
940	0\\
941	0\\
942	0\\
943	0\\
944	0\\
945	0\\
946	0\\
947	0\\
948	0\\
949	0\\
950	0\\
951	0\\
952	0\\
953	0\\
954	0\\
955	0\\
956	0\\
957	0\\
958	0\\
959	0\\
960	0\\
961	0\\
962	0\\
963	0\\
964	0\\
965	0\\
966	0\\
967	0\\
968	0\\
969	0\\
970	0\\
971	0\\
972	0\\
973	0\\
974	0\\
975	0\\
976	0\\
977	0\\
978	0\\
979	0\\
980	0\\
981	0\\
982	0\\
983	0\\
984	0\\
985	0\\
986	0\\
987	0\\
988	0\\
989	0\\
990	0\\
991	0\\
992	0\\
993	0\\
994	0\\
995	0\\
996	0\\
997	0\\
998	0\\
999	0\\
1000	0\\
1001	0\\
1002	0\\
1003	0\\
1004	0\\
1005	0\\
1006	0\\
1007	0\\
1008	0\\
1009	0.0109751501504374\\
1010	0.0411083082375464\\
1011	0.0722380299373302\\
1012	0.103823607632769\\
1013	0.135220514946454\\
1014	0.165754645247167\\
1015	0.194775484687005\\
1016	0.221660813782587\\
1017	0.245778868036812\\
1018	0.26646824248222\\
1019	0.283082121062631\\
1020	0.295077667112676\\
1021	0.302104545972175\\
1022	0.304063677232649\\
1023	0.301127091573691\\
1024	0.29372128158877\\
1025	0.282482544061506\\
1026	0.268195003665684\\
1027	0.251721860037789\\
1028	0.233938431500787\\
1029	0.21567526348113\\
1030	0.197708196619704\\
1031	0.180864784116396\\
1032	0.166223090235151\\
1033	0.155221194761011\\
1034	0.149540744271029\\
1035	0.150827615189522\\
1036	0.160394607108879\\
1037	0.178999957977964\\
1038	0.206735672305094\\
1039	0.243024134161719\\
1040	0.286701968931336\\
1041	0.336162161571369\\
1042	0.389524868223126\\
1043	0.444811362675721\\
1044	0.500100961744264\\
1045	0.553647102003322\\
1046	0.603923431307901\\
1047	0.649606546340879\\
1048	0.689558105829513\\
1049	0.722856415525642\\
1050	0.748862315446572\\
};
% \addplot [color=blue,solid,forget plot]
%   table[row sep=crcr]{%
% 1	-0\\
% 1050	-0\\
% };
% \addplot [color=blue,solid,forget plot]
%   table[row sep=crcr]{%
% 1	1\\
% 1050	1\\
% };
\end{axis}
\begin{axis}[
anchor=origin,
at={(1250,303)}, 
no markers, domain=-5:5, samples=100,
rotate around={-90:(current axis.origin)}, % Rotate around the origin
axis lines*=center,
every axis y label/.style={at=(current axis.above origin),anchor=south},
every axis x label/.style={at=(current axis.right of origin),anchor=west},
%height=5cm, width=8cm,
width=4.3in,
height=2in,
xtick={-3.2,0}, 
xmin=-5,
xmax=5,
ymin=0,
xticklabels={},
ytick=\empty,
enlargelimits=false, clip=false, axis on top,
grid = major
]
\addplot [fill=gray!40, draw=none, domain=-5:-3.2] {gauss(0,1.7)} \closedcycle;
\addplot [fill=gray!40, draw=none, domain=0:5] {gauss(0,1.7)} \closedcycle;
\addplot [very thick,black] {gauss(0,1.7)}; % node [pos=0.5,pin={80:{$\displaystyle\sigma^2 = 2\sum_{n = 1}^{N_u/2} P_n$}},inner sep=0pt] {};
\node (r1) at (axis cs: -3.7, 0.1) {\Huge $r\sigma$};
\node (nothing) at (axis cs: 3.7, 0.25) {};
\end{axis}
\end{tikzpicture}%}
		\caption{ACO-OFDM}
	\end{subfigure}
	\caption{(a) DC-OFDM and (b) ACO-OFDM waveforms.}
\end{figure}
\FloatBarrier


\FloatBarrier
\begin{figure}[h!]
	\centering
	\resizebox{\linewidth}{!}{\tikzstyle{int}=[draw, minimum size=4em]

\begin{tikzpicture}
\begin{axis}[
  no markers, domain=-3.5:3.5, samples=100,
  axis lines*=center, xlabel=$x$,
  every axis y label/.style={at=(current axis.above origin),anchor=south},
  every axis x label/.style={at=(current axis.right of origin),anchor=west},
  height=5cm, width=12cm,
  xtick={-2, 2}, 
  xticklabels={\Large $-r\sigma$, \Large $r\sigma$},
  ytick=\empty,
  enlargelimits=false, clip=false, axis on top,
  grid = major
  ]
  \addplot [fill=white!50!black, draw=none, domain=-5:-2] {gauss(0,1)} \closedcycle;
  \addplot [fill=white!50!black, draw=none, domain=2:5] {gauss(0,1)} \closedcycle;
  \addplot [very thick,black] {gauss(0,1)} node [pos=0.5,pin={80:{\Large $\displaystyle\sigma^2 = 2\sum_{n = 1}^{N_u/2} P_n$}},inner sep=0pt] {};
\end{axis}
\end{tikzpicture}

\begin{tikzpicture}
\begin{axis}[
  no markers, domain=-3.5:3.5, samples=100,
  axis lines*=center, xlabel=$x$,
  every axis y label/.style={at=(current axis.above origin),anchor=south},
  every axis x label/.style={at=(current axis.right of origin),anchor=west},
  height=5cm, width=12cm,
  xtick={-1.4, 1.4}, 
  xticklabels={\Large $-r\sigma^{\prime}$, \Large $r\sigma^{\prime}$},
  ytick=\empty,
  enlargelimits=false, clip=false, axis on top,
  grid = major
  ]
  \addplot [fill=white!50!black, draw=none, domain=-5:-1.4] {gauss(0,0.7)} \closedcycle;
  \addplot [fill=white!50!black, draw=none, domain=1.4:5] {gauss(0,0.7)} \closedcycle;
  \addplot [very thick,black] {gauss(0,0.7)} node [pos=0.5,pin={80:{\Large $\displaystyle\sigma^{\prime 2} = 2\sum_{n = 1}^{N_u/2} P_n|G_{DAC}(f_n)|^2$}},inner sep=0pt] {};
\end{axis}
\end{tikzpicture}
}
	\caption{Effect of DAC frequency response in the variance of the OFDM signal under Gaussian approximation.}
\end{figure}
\FloatBarrier

\section{Effect of noise}

We assume that the total noise at the receiver is AWG with double-sided PSD equal to $N_0/2$. The noise is filtered by an antialiasing filter with frequency response $G(f)$. The noise PSD after the filter is given by
\begin{equation}
S_w(f) = \frac{N_0}{2}|G(f)|^2
\end{equation}
In the time domain, the autocorrelation function of the noise $R_n(t)$ is given by
\begin{equation}
R_w(t) = \frac{N_0}{2}g(t)*g(-t)^*
\end{equation}

The noise signal is then sampled at a rate $f_s = 1/T_s$. Thus, from the sampling property of the autocorrelation, the discrete-time noise autocorrelation is given by
\begin{equation}
R_w[n] = R_w(nT_s) = \frac{N_0}{2}g(nT_s)*g(-nT_s)^*
\end{equation}
Note: this equation assumes that there is \textbf{no aliasing after sampling}.

Assuming that $g(t)$ is an ideal filter whose cutoff frequency is $f_s/2$:
\begin{align}
& G(f) = \begin{cases}
1, & |f| < f_s/2 \\
0, & \mathrm{otherwise}
\end{cases} \\
& g(t) = \frac{1}{T_s}\mathrm{sinc}\Big(\frac{t}{T_s}\Big).
\end{align}

Thus the autocorrelation function would be reduced to
\begin{align} \nonumber
R_w[n] & = \frac{N_0}{2}\frac{1}{T_s}\mathrm{sinc}\Big(\frac{nT_s}{T_s}\Big) = \frac{N_0}{2}\frac{1}{T_s}\mathrm{sinc}(n) \\  
& = \frac{N_0}{2}f_s\delta[n]
\end{align}
since the sinc function evaluated at integers is only nonzero for $n = 0$. Therefore, \textbf{the sampled noise is also AWG and has variance} $\sigma^2 = N_0f_s/2$. 

Calculating the $N$-point $\frac{1}{N}\cdot$ DFT of this signal results
\begin{align}
W[n] &= \frac{1}{N}\sum_{k = 0}^{N-1}w[k]e^{-j\frac{2\pi}{N}kn} \\
R_W[n] & = E[W[n+m]W[m]^*] = E\bigg[\frac{1}{N^2}\sum_{k_1 = 0}^{N-1}\sum_{k_2 = 0}^{N-1}w[k_1]w[k_2]e^{-j\frac{2\pi}{N}k_1(n+m)}e^{-j\frac{2\pi}{N}k_2m}\bigg] \\
& = \frac{1}{N^2}\sum_{k_1 = 0}^{N-1}\sum_{k_2 = 0}^{N-1}E[w[k_1]w[k_2]]e^{-j\frac{2\pi}{N}(k_1n + m(k_2-k_1))} \\
& = \frac{1}{N^2}\sum_{k_1 = 0}^{N-1}\sum_{k_2 = 0}^{N-1}\frac{N_0}{2}f_s\delta[k_1-k_2]e^{-j\frac{2\pi}{N}(k_1n + m(k_2-k_1))} \\
& = \frac{1}{N^2}\sum_{k = 0}^{N-1}\frac{N_0}{2}f_se^{-j\frac{2\pi}{N}kn} \\
\end{align}

Therefore the variance of $W[n]$ is given by
\begin{equation}
\mathrm{Var}(W[n]) = R_W[0] = \frac{1}{N}\frac{N_0}{2}f_s
\end{equation}

As a result, the OFDM demoulator performs detection over random variables $Y_n = X_n + W_n$, where $X_n$ is the signal component.

For a more realistic filter the factor $f_s/2$ should be replaced by the corresponding noise bandwidth of the filter.

%For the simulations, the noise is band limited between $f_sM_{ct}$, where $M_{ct}$ is the oversampling ratio to simulate continuous time. Thus its PSD is given by
%\begin{equation}
%S_w(f) = \frac{N_0}{2}f_sM_{ct}, |f| \leq 1
%\end{equation}
%After downsampling by a factor $M_{ct}$, the amplitude of the PSD is divided by $M_{ct}$. Therefore, 
%\begin{equation}
%S_w(f) = \frac{N_0}{2}f_s, |f| \leq 1
%\end{equation}
%The factor $M_{ct}$ divides the PSD (i.e., the power) because autocorrelation function is also downsampled i.e., $R_w[n] = \tilde{R}_w[nM_{ct}]$.

\section{Including thermal noise}

Disregarding all the frequencies responses of the filters and assuming that quanitzation noise is the same at the transmitter and receiver, the $SNR_n$ would reduce to

\begin{align} \nonumber
SNR_n & =  \frac{NK^2|G_{tot}(f_n)|^2P_n}{\frac{N_0}{2}f_s|G_{ADC}(f_n)|^2 + \sigma_{Q,tx}^2|G_{tot}(f_n)|^2 + \sigma_{Q,rx}^2} \\
& =  \frac{NK^2P_n}{\frac{N_0}{2}f_s + 2\sigma_{Q}^2} \\
\end{align}
where \begin{equation}
\sigma_Q^2 \approx \frac{1}{12}\bigg(\frac{2r\sigma}{2^{ENOB}}\bigg)^2 \approx \frac{1}{3}\frac{r^2N_uP_n}{2^{2ENOB}}
\end{equation}
Therefore

\begin{align} \nonumber
SNR_n & =  \frac{NK^2P_n}{\frac{N_0}{2}f_s + 2\sigma_{Q}^2} \\
& =  \frac{NP_n}{\frac{N_0}{2}f_s + \frac{2}{3}\frac{r^2N_uP_n}{2^{2ENOB}}} \\
& =  \frac{1}{\frac{1}{SNR_{th}} + \frac{2}{3}\frac{r^2}{r_{os}\cdot 2^{2ENOB}}} \\
\end{align}

\begin{align} \nonumber
SNR_n\bigg(\frac{1}{SNR_{th}} + \frac{2}{3}\frac{r^2}{r_{os}\cdot 2^{2ENOB}}\bigg) = 1 \\
\frac{2}{3}\frac{r^2}{r_{os}\cdot 2^{2ENOB}} = \frac{1}{SNR_n} - \frac{1}{SNR_{th}}  \\
\frac{3}{2}r_{os}\frac{2^{2ENOB}}{r^2} = \frac{SNR_nSNR_{th}}{SNR_{th} - SNR_n}  \\
ENOB= \frac{1}{2}\log_2\bigg[\frac{2}{3}\frac{r^2}{r_{os}}\bigg(\frac{SNR_nSNR_{th}}{SNR_{th} - SNR_n}\bigg)\bigg]  \\
\end{align}

\section{Clipping distortion}
If we clip the signal at two levels $-r_-\sigma$ and $r_+\sigma$ it can be shown that 
\begin{equation}
K = 1 -Q(r_-) - Q(r_+)
\end{equation}
\begin{equation} \label{Bussgangs}
x_c(k) = Kx(k) + d(k) + r_-\sigma
\end{equation}

where  $d(k)$ is the clipping noise that is uncorrelated with $x(k)$ i.e., $E[x(k)d(k)] = 0 ~\forall~k$.

The pdf $x_c(k)$ is a clipped Gaussian centered ar $r_-\sigma$ and clipped at 0. That is,
\begin{equation}
f_{x_c(k)}(x) = \begin{cases}
Q(r_-)\delta(x) + Q(r_+)\delta(x - r_- - r_+) + \frac{1}{\sqrt{2\pi\sigma^2}}\exp\Big(-\frac{(x-r_-\sigma)^2}{2\sigma^2}\Big), & 0 \leq x   \leq (r_- + r_+)\sigma \\
0, & \mathrm{otherwise}
\end{cases}
\end{equation}
If we remove the dc bias of $x_c(k)$ it becomes easier 

\begin{equation} \label{Bussgangs}
\tilde{x}(k) = x_c(k) - r_-\sigma = Kx(k) + d(k)
\end{equation}
\begin{equation}
f_{\tilde{x}(k)}(x) = \begin{cases}
Q(r_-)\delta(x + r_-) + Q(r_+)\delta(x - r_+) + \frac{1}{\sqrt{2\pi\sigma^2}}\exp\Big(-\frac{x^2}{2\sigma^2}\Big), & -r_-\sigma \leq x   \leq  r_+\sigma \\
0, & \mathrm{otherwise}
\end{cases}
\end{equation}
The moments are given by
\begin{align}
& E(\tilde{x}(k)) = \sigma\Big(-r_-Q(r_-) + r_+Q(r_+) + \frac{1}{\sqrt{2\pi}}(e^{-\frac{r_-^2}{2}} - e^{-\frac{r_+^2}{2}})\Big) \\
& E(|\tilde{x}(k)|^2) = \sigma^2\Big(1 - Q(r_-) - Q(r_+) + r_-^2Q(r_-) + r_+^2Q(r_+) -  \frac{1}{\sqrt{2\pi}} \Big(r_-e^{-r_-^2/2} + r_+e^{-r_+^2/2}\Big) \Big) \\
& = \sigma^2\Big(1 + Q(r_-)(r_-^2 - 1) + Q(r_+)(r_+^2 - 1) -  \frac{1}{\sqrt{2\pi}} \Big(r_-e^{-r_-^2/2} + r_+e^{-r_+^2/2}\Big) \Big) \\
\end{align}

\subsection{DC-OFDM}
Assuming that $r_- = r_+ = r$. We have that $E(\tilde{x}(k))  = 0$ and 

\begin{equation}
\mathrm{Var}(x_c(k)) = E(|\tilde{x}(k)|^2) = \sigma^2\Big(1 + 2Q(r)(r^2 - 1) -  \frac{2r}{\sqrt{2\pi}} e^{-r^2/2} \Big) \\
\end{equation}
And for the clipping noise

\begin{equation}
\mathrm{Var}(d(k)) = \mathrm{Var}(\tilde{x}(k)) - K^2\sigma^2 = \sigma^2\Big(1 - K^2 + 2Q(r)(r^2 - 1) -  \frac{2r}{\sqrt{2\pi}} e^{-r^2/2} \Big) \\
\end{equation}

\subsection{ACO-OFDM}
Assuming that $r_- = 0$ and $r_+ = r$. We have that $E(\tilde{x}(k))  = \sigma\Big(rQ(r) + \frac{1}{\sqrt{2\pi}}(1 - e^{-r^2/2})\Big)$ and 

\begin{equation}
E(|\tilde{x}(k)|^2) = \sigma^2\Big(1/2 + Q(r)(r^2 - 1) -  \frac{r}{\sqrt{2\pi}}e^{-r^2/2}\Big)
\end{equation}
And for the clipping noise

\begin{align} \nonumber
\mathrm{Var}(d(k)) & = E(|\tilde{x}(k)|^2) - E(\tilde{x}(k))^2 - K^2\sigma^2 \\ \nonumber
& = \sigma^2\bigg[rQ(r) + \frac{1}{\sqrt{2\pi}}(1 - e^{-r^2/2}) - \Big(1/2 + Q(r)(r^2 - 1) -  \frac{r}{\sqrt{2\pi}}e^{-r^2/2} \Big)^2 - K^2\bigg]
\end{align}

\section{Quantization}
\subsection{Transmitter}
Quantization noise variance at the transmitter is given by
\begin{align} \nonumber
\sigma^2_{Q, tx} &= (1-P_c)\frac{1}{12}\frac{\Delta X^2}{2^{2ENOB}} = \begin{cases}
(1-P_c)\frac{(2r_{tx}\sigma)^2}{12\cdot 2^{2ENOB}}, &\text{DC-OFDM} \\ 
(1-P_c)\frac{(r_{tx}\sigma)^2}{12\cdot 2^{2ENOB}}, &\text{ACO-OFDM}
\end{cases} \\ \nonumber
& =\begin{cases}
(1-P_c)\frac{r_{tx}^2\sigma^2}{3\cdot 2^{2ENOB}}, &\text{DC-OFDM} \\ 
(1-P_c)\frac{r_{tx}^2\sigma^2}{12\cdot 2^{2ENOB}}, &\text{ACO-OFDM}
\end{cases} \\ 
& \approx\begin{cases}
\frac{r_{tx}^2\sigma^2}{3\cdot 2^{2ENOB}}, &\text{DC-OFDM} \\ 
\frac{r_{tx}^2\sigma^2}{24\cdot 2^{2ENOB}}, &\text{ACO-OFDM}
\end{cases}
\end{align}
where the approximation follows from the fact that we are considering relatively large values of clipping ratio in order to make clipping small.

\subsection{Receiver}
\begin{align} \nonumber
\sigma^2_{Q,rx} & = (1-P_c)\frac{\Delta X^2}{12\cdot 2^{2ENOB}} = \begin{cases}
(1-P_c)\frac{(2r_{rx}\sigma_{rx})^2}{12\cdot 2^{2ENOB}}, &\text{DC-OFDM} \\
(1-P_c)\frac{(\sigma/\sqrt{2\pi} + r_{rx}\sigma_{rx})^2}{12\cdot 2^{2ENOB}}, &\text{ACO-OFDM}
\end{cases} \\ \nonumber
& \approx \begin{cases}
\frac{r_{rx}^2\sigma_{rx}^2}{3\cdot 2^{2ENOB}}, &\text{DC-OFDM} \\
\frac{(\sigma/\sqrt{2\pi} + r_{rx}\sigma_{rx})^2}{12\cdot 2^{2ENOB}}, &\text{ACO-OFDM}
\end{cases}
\end{align}
At the receiver the clipping probability $P_c$ for both OFDM techniques is assumed very small.

\section{SNR at the receiver}

Discrete-time OFDM signal generated by performing a $N\cdot IDFT(\cdot)$ operation:
\begin{equation}
\tilde{x}(k) = \sum_{n = 0}^{N-1}X_n\exp\bigg(j\frac{2\pi}{N}kn\bigg)
\end{equation}

Total power of the OFDM signal at the transmitter:
\begin{equation}
\sigma^2 = E[x(k)^2] = \sum_{n = 0}^{N-1} P_n = 2\sum_{n = 1}^{N/2-1} P_n
\end{equation}
$P_n$ is the power at the $n$th subcarrier.

Disregarding aliasing, the received power at the $n$th subcarrier is given by
\begin{equation}
P_{n,rx} = P_nR^2K^2|G_{ch}(f_n)|^2
\end{equation}
where $|G_{ch}(f_n)|^2 = |G_{DAC}(f_n)|^2\cdot|H_{mod}(f_n)|^2\cdot|H_{fib}(f_n)|^2\cdot|G_{ADC}(f_n)|^2$, and $K$ is the attenuation factor due to clipping ($K \approx 1$ for DC-OFDM, and $K = 1/2$ for ACO-OFDM). 


Following the Noise calculation we have the following noise components at the $n$th subcarrier
\begin{align}
& \sigma_{th}^2 = \frac{1}{N}f_s\frac{N_0}{2}|G_{ADC}(f_n)|^2 \\
& \sigma_{shot}^2 = \frac{1}{N}f_s\frac{S_{shot}}{2}|G_{ADC}(f_n)|^2 \\
& \sigma_{RIN}^2 = \frac{1}{N}f_s\frac{S_{RIN}}{2}|G_{ADC}(f_n)|^2 \\
& \sigma_{Q, tx}^2 = \frac{1}{N}(1-P_c)\frac{\Delta X^2}{12\cdot 2^{2ENOB}}|G_{ch}(f_n)|^2 \\
& \sigma_{Q, rx}^2 = \frac{1}{N}(1-P_c)\frac{\Delta X^2}{12\cdot 2^{2ENOB}}
\end{align}
where the one-sided psd are
\begin{align}
& N_0 = R^2\cdot NEP^2 \\
& S_{shot} = 2q(RP_{in} + I_d)~~~~\text({\sl pin}) \\
& S_{shot} = 2qM^2F_A(RP_{in} + I_d)~~~~\text({\sl apd}) \\
& S_{RIN} = 2\text{RIN}(RP)^2
\end{align}
where $F_A(M) = k_AM + (1 - k_A)(2 - 1/M)$.

SNR at the $n$th subcarrier

\begin{equation}
SNR_n = \frac{R^2K^2\cdot N\cdot P_n|G_{ch}(f_n)|^2}{f_s\Big(\frac{N_0}{2} + q(R\bar{P} + I_d) + \text{RIN}(R\bar{P})^2\Big)|G_{ADC}(f_n)|^2 + \sigma_{Q, tx}^2 + \sigma_{Q, rx}^2}
\end{equation}
The signal-dependent variances are calculated using the average received optical power $\bar{P}$ as an approximation.

\section{Required ENOB}

SNR at the receiver assuming thermal and quantization noises:
\begin{equation}
SNR_n = \frac{K^2\cdot N\cdot P_n|G_{tot}(f_n)|^2}{\frac{N_0}{2}f_s|G_{ADC}(f_n)|^2 + \sigma_{Q,tx}^2|G_{tot}(f_n)|^2 + \sigma_{Q,rx}^2}
\end{equation}
where the $N$ (number of subcarriers) appear due to the calculation of the noise in the frequency domain. 

Note that the signal power also affects the noise, which suggests that the SNR is bounded. In the limit where quantization is dominant, and ignoring frequency responses:

\begin{equation} \nonumber
SNR_n = \frac{K^2NP_n}{\sigma_{Q,tx}^2 + \sigma_{Q,rx}^2}
\end{equation}

\subsection{DC-OFDM}
\begin{align}
SNR_n = \frac{K^2NP_n}{\sigma_{Q,tx}^2 + \sigma_{Q,rx}^2}
& = \frac{N\frac{r_{os}\sigma^2}{N}}{\frac{r_{tx}^2\sigma^2}{3\cdot 2^{2ENOB}} + \frac{r_{rx}^2\sigma_{rx}^2}{3\cdot 2^{2ENOB}}} \\ \nonumber
& = \frac{r_{os}\sigma^2(3\cdot 2^{2ENOB})}{r_{tx}^2\sigma^2 + r_{rx}^2\sigma_{rx}^2} \\ \nonumber
& = \frac{r_{os}(3\cdot 2^{2ENOB})}{2r^2} \\ \nonumber
\end{align}
$\sigma^2 = P_nN_u$, $N_u = N/r_{os}$, $r = r_{tx} = r_{rx}$, $\sigma^2 = \sigma_{rx}^2$.

\begin{align} \nonumber
& 2ENOB = \log_2\bigg(\frac{2r^2}{3r_{os}}SNR_n\bigg) \\
& ENOB = \frac{1}{2}\log_2\bigg(\frac{2r^2}{3r_{os}}SNR_n\bigg)
\end{align}

\subsection{ACO-OFDM}
\begin{align} \nonumber
SNR_n = \frac{K^2NP_n}{\sigma_{Q,tx}^2 + \sigma_{Q,rx}^2} &= \frac{1/4N\frac{2r_{os}\sigma^2}{N}}{(1-P_c)\frac{r_{tx}^2\sigma^2}{12\cdot 2^{2ENOB}} + \frac{(\sigma/\sqrt{2\pi} + r_{rx}\sigma_{rx})^2}{12\cdot 2^{2ENOB}}} \\ \nonumber
&= \frac{1/2r_{os}\sigma^2({12\cdot 2^{2ENOB}})}{\frac{1}{2}r_{tx}^2\sigma^2 + (\sigma/\sqrt{2\pi} + r_{rx}\sigma_{rx})^2} \\ \nonumber
&= \frac{6r_{os}\sigma^2(2^{2ENOB})}{\frac{1}{2}r^2\sigma^2 + \sigma^2(1/\sqrt{2\pi} + r)^2} \\ \nonumber
&= \frac{12r_{os}(2^{2ENOB})}{r^2 + 2(1/\sqrt{2\pi} + r)^2} \\ \nonumber
\end{align}
$\sigma^2 = P_nN_u$, $N_u = N/(2r_{os})$, $r = r_{tx} = r_{rx}$, $\sigma^2 = \sigma_{rx}^2$ (Does not correspond to signal variance at receiver, but tail probability).

\begin{align} \nonumber
& 2ENOB = \log_2\bigg(\frac{r^2 + 2(1/\sqrt{2\pi} + r)^2}{12r_{os}}SNR_n\bigg) \\
& ENOB = \frac{1}{2}\log_2\bigg(\frac{r^2 + 2(1/\sqrt{2\pi} +r)^2}{12r_{os}}SNR_n\bigg)
\end{align}

\section{Noise Regimes}

\begin{align} \nonumber
SNR_n &= \frac{K^2\cdot N\cdot P_n|G_{ch}(f_n)|^2}{f_s\Big(\frac{N_0}{2} + q(R\bar{P} + I_d) + \text{RIN}(R\bar{P})^2\Big)|G_{ADC}(f_n)|^2 + \sigma_{Q, tx}^2 + \sigma_{Q, rx}^2} \\
&= \frac{N\cdot \frac{\bar{P}^2}{r^2N_u}}{f_s\Big(\frac{N_0}{2} + qR\bar{P} + \text{RIN}(R\bar{P})^2\Big) + 2\sigma_{Q}^2} \\
&= \frac{r_{os}\frac{\bar{P}^2}{r^2}}{f_s\Big(\frac{N_0}{2} + qR\bar{P} + \text{RIN}(R\bar{P})^2\Big) + 2\sigma_{Q}^2}
\end{align}

\section{Number of DSP Operations}
An FFT operation requires approximately $4N\log_2 N$ real operations.

OFDM symbol duration
\begin{equation}
T_{OFDM} = \frac{N + N_{cp}}{f_s} = \frac{N}{r_{os}R_s}  = \frac{N\log_2 M}{2pr_{os}R_b}
\end{equation}

Thus, the number of real operations per bit is given by
\begin{equation}
\mathcal{O}_{TX} = \frac{4N\log_2 N}{T_{OFDM}R_b} = 8pr_{os}\frac{\log_2 N}{\log_2 M}
\end{equation}

At the receiver we also have $N_u$ additional complex multiplications from the one-tap equalizer. Thus, the number of real operations per bit is given by \footnote{complex multiplication can be carried out by using only 3 real multiplications: $(a + jb)(c + jd) = (ac-bd) + j[(a+b)(c+d)-ac-bd]$, which takes 8 real operations altogether. Here, however, we assumed 6 real operations altogether, which is the least number of operations required.}
\begin{equation}
\mathcal{O}_{RX} = \frac{4N\log_2 N + 6N_u}{T_{OFDM}R_b} = \frac{2(4pr_{os}\log_2 N + 6)}{\log_2 M} = 4\frac{2pr_{os}\log_2 N + 3}{\log_2 M}
\end{equation}

\section{BER in ideal AWGN channel}

\subsection{OOK}

\begin{align} \nonumber
BER & = Q(\sqrt{SNR}) = Q\bigg(\sqrt{\frac{2R^2\bar{P}^2}{R_bN_0}}\bigg) \\
&= Q\bigg(R\bar{P}\sqrt{\frac{2}{R_bN_0}}\bigg)
\end{align}


\begin{align}
\bar{P} = \frac{1}{R}\sqrt{\frac{R_bN_0}{2}}Q^{-1}(BER)
\end{align}

\subsection{OFDM}

\begin{align} \nonumber
SNR_n &= \frac{K^2\cdot N\cdot P_n|G_{ch}(f_n)|^2}{f_s\Big(\frac{N_0}{2} + q(R\bar{P} + I_d) + \text{RIN}(R\bar{P})^2\Big)|G_{ADC}(f_n)|^2 + \sigma_{Q, tx}^2 + \sigma_{Q, rx}^2} \\
&= \frac{2K^2\cdot N\cdot P_n}{f_sN_0} \\
&= \begin{cases}
\frac{2r_{os}\frac{\bar{P}^2}{r^2}}{f_sN_0}, &\text{DC-OFDM} \\
\frac{(2\pi\bar{P}^2)r_{os}}{f_sN_0}, &\text{ACO-OFDM}
\end{cases} \\
&= \begin{cases}
\frac{\frac{\bar{P}^2}{r^2}\log_2M}{R_bN_0}, &\text{DC-OFDM} \\
\frac{\pi\bar{P}^2\log_2M}{2R_bN_0}, &\text{ACO-OFDM}
\end{cases} \\
\end{align}
In the ideal case, $r_{os} = 1$, and $f_s = R_s = 2p\frac{R_b}{\log_2 M}$

\begin{align} \nonumber
BER &= \frac{4}{\log_2M}\bigg(1 - \frac{1}{\sqrt{M}}\bigg)Q\bigg(\sqrt{\frac{3}{M-1}SNR}\bigg) \\ 
& =\begin{cases}
\frac{4}{\log_2M}\bigg(1 - \frac{1}{\sqrt{M}}\bigg)Q\bigg(\sqrt{\frac{3}{M-1}\frac{\frac{R^2\bar{P}^2}{r^2}\log_2M}{R_bN_0}}\bigg), &\text{DC-OFDM} \\
\frac{4}{\log_2M}\bigg(1 - \frac{1}{\sqrt{M}}\bigg)Q\bigg(\sqrt{\frac{3}{M-1}\frac{\pi R^2\bar{P}^2\log_2M}{2R_bN_0}}\bigg), &\text{ACO-OFDM}
\end{cases} \\
& =\begin{cases}
\frac{4}{\log_2M}\bigg(1 - \frac{1}{\sqrt{M}}\bigg)Q\bigg(\frac{R\bar{P}}{r}\sqrt{\frac{3}{M-1}\frac{\log_2M}{R_bN_0}}\bigg), &\text{DC-OFDM} \\
\frac{4}{\log_2M}\bigg(1 - \frac{1}{\sqrt{M}}\bigg)Q\bigg(R\bar{P}\sqrt{\frac{3}{M-1}\frac{\pi\log_2M}{2R_bN_0}}\bigg), &\text{ACO-OFDM}
\end{cases}
\end{align}

\section{Single-sideband OFDM}

Given a double-side band OFDM signal with Hermitian symmetry $s(t)$, we can obtain the single-side band signal $x(t)$ through the Hilbert transform
\begin{equation}
x(t) = s(t) + j\mathcal{H}\{s(t)\},
\end{equation}
where
\begin{equation}
\mathcal{F}\big\{\mathcal{H}\{s(t)\}\big\} = \begin{cases}
jS(f), & f > 0 \\
0, & f = 0 \\
-jS(f), & f < 0
\end{cases}.
\end{equation}

The complex signal $p(t) = x(t) + C$ is transmitted through the optical channel, where $C$ denotes the carrier component. The dispersive channel has impulse response given by $h(t)$. Hence, at the receiver we have

\begin{equation}
r(t) = \sqrt{G_{AMP}} h(t)\ast x(t) + \sqrt{G_{AMP}}C + n(t)
\end{equation}
since $H(0) = 1$.

After square law detection
\begin{align} \label{eq:ssb-ofdm-rx-signal} \nonumber
|r(t)|^2 &= RG_{AMP}|h(t)\ast x(t)|^2 + RG_{AMP}|C|^2 + R|n(t)|^2 \\ \nonumber
& + RG_{AMP}((x\ast h)C^* + (x\ast h)^*C)\\ \nonumber
& + R\sqrt{G_{AMP}}((x\ast h)n^*(t) + (x\ast h)^*n(t)) \\ 
& + R\sqrt{G_{AMP}}(n(t)C^* + n^*(t)C) 
\end{align}

The dominant noise has two components: the signal-ASE beat noise and the carrier-ASE beat noise. Each of these have variances given by
\begin{align}
\sigma^2_{sig-ASE} = 4R^2G_{AMP}\sigma^2P_s \\
\sigma^2_{carrier-ASE} = 4R^2G_{AMP}\sigma^2|C|^2,
\end{align}
where $\sigma^2$ is the ASE noise variance per real dimension. Hence, we can write the noise component as $2R\sqrt{G_{AMP}(P_s + |C|^2)}n(t)$, where $n(t)\sim\mathcal{N}(0, \sigma^2)$.

The signal-carrier beat signal $(x\ast h)C^* + (x\ast h)^*C$ corresponds to the desired component. In the frequency domain
\begin{align}
C^*X(f)H(f) + CX^*(-f)H^*(-f) = 2|C|S(f)G(f),
\end{align}
where 
\begin{equation}
G(f) = \begin{cases}
H(f)e^{-j\phi_C}, &f >0 \\
2\cos(\phi_C),& f = 0 \\
H^*(-f)e^{j\phi_C}, &f < 0
\end{cases}
\end{equation}

Rewriting \eqref{eq:ssb-ofdm-rx-signal} with the results above yields:
\begin{align}
|r(t)|^2 &= RG_{AMP}|h(t)\ast x(t)|^2 \\ \nonumber
& + 2RG_{AMP}|C|s(t)\ast g(t) \\
& + 2R\sqrt{G_{AMP}(P_s + |C|^2)}n(t) 
\end{align}
The constant term $|C|^2$ and the ASE-ASE beat noise term were neglected.

Hence, the SNR at the $n$th subcarrier after quantization and FFT operation is given by

\begin{align}
SNR_n = \frac{N_{FFT}(4R^2G_{AMP}^2|C|^2)P_n/4}{2R^2G_{AMP}(P_s + |C|^2)S_{sp}f_s + \sigma_{Q,rx}^2 + \gamma R^2G_{AMP}^2P_s}
\end{align}
where $\sigma_{Q,rx}^2 = 2/3r^2(2R^2G_{AMP}^2|C|^2P_s)2^{-2ENOB}$. Therefore,

\begin{align}
SNR_n = \frac{N_{FFT}\cdot CSPR\cdot P_n}{2G_{AMP}^{-1}(1 + CSPR)S_{sp}f_s + 2/3r^2(2P_sCSPR)2^{-2ENOB} + \gamma}
\end{align}
where $P_n$ is referred to the input of the optical amplifier. Hence, the OSNR is given by
\begin{equation}
OSNR = \frac{G_{AMP}P_s}{2S_{sp}B_{ref}}
\end{equation}
Note that the carrier power is not included in the OSNR calculation.

\bibliographystyle{plain}
\bibliography{bib}

\end{document}