\documentclass[10pt]{beamer}
\usefonttheme{professionalfonts}
%\usetheme{CambridgeUS}
%
% Choose how your presentation looks.
%
% For more themes, color themes and font themes, see:
% http://deic.uab.es/~iblanes/beamer_gallery/index_by_theme.html
%
\mode<presentation>
{
  \usetheme{default}      % or try Darmstadt, Madrid, Warsaw, ...
  \usecolortheme{beaver} % or try albatross, beaver, crane, ...
  \usefonttheme{default}  % or try serif, structurebold, ...
  \setbeamertemplate{navigation symbols}{}
  \setbeamertemplate{caption}[numbered]
} 

\usepackage[english]{babel}
\usepackage[utf8x]{inputenc}
\usepackage{tikz}
\usepackage{pgfplots}
\usepackage{array}  % for table column M
\usepackage{makecell} % to break line within a cell
\usepackage{verbatim}
\usepackage{graphicx}
\usepackage{epstopdf}
\usepackage{amsfonts}
\usepackage{xcolor}
\usepackage{ifthen}
\usepackage{dsfont}
%\usepackage{mathtools}
\usepackage[makeroom]{cancel}
\usetikzlibrary{spy}
%\captionsetup{compatibility=false}
%\usepackage{dsfont}
\usepackage[absolute,overlay]{textpos}
\usetikzlibrary{calc, angles,quotes}
\usetikzlibrary{pgfplots.fillbetween, backgrounds}
\usetikzlibrary{positioning}
\usetikzlibrary{arrows}
\usetikzlibrary{pgfplots.groupplots}
\usetikzlibrary{arrows.meta}
\usetikzlibrary{plotmarks}
\usetikzlibrary{shapes.geometric}
\usetikzlibrary{decorations.markings}
\usepgfplotslibrary{groupplots}
\pgfplotsset{compat=newest} 
%\pgfplotsset{plot coordinates/math parser=false}

\usepackage{hyperref}
\hypersetup{
    colorlinks=true,
    linkcolor=blue,
    filecolor=magenta,      
    urlcolor=cyan,
}

\definecolor{blue2}{RGB}{51, 105, 232}  
\definecolor{red2}{RGB}{213, 15, 37}  
\definecolor{green2}{RGB}{0, 153, 37}  
\definecolor{green3}{rgb}{0.1922, 0.6392, 0.3294}% 
\definecolor{yellow2}{RGB}{238, 178, 17} 
\definecolor{gray2}{RGB}{102, 102, 102}
\definecolor{orange2}{RGB}{230, 85, 13}

% Qualitative pallete set1 from www.ColorBrewer.org
\definecolor{Qred}{RGB}{228,26,28}
\definecolor{Qblue}{RGB}{55,126,184}
\definecolor{Qgreen}{RGB}{77,175,74}
\definecolor{Qpurple}{RGB}{152,78,163}
\definecolor{Qorange}{RGB}{255,127,0}
\definecolor{Qyellow}{RGB}{255,255,51}
\definecolor{Qbrown}{RGB}{166,86,40}
\definecolor{Qpink}{RGB}{247,129,191}
\definecolor{Qgray}{RGB}{153,153,153}

%
%\def\EXTERNALIZE{1}
%%% Externalizing
\usepgfplotslibrary{external} 
\ifdefined\EXTERNALIZE
	\makeatletter	
	\newcommand*{\overlaynumber}{\number\beamer@slideinframe}
	\tikzset{
		beamer externalizing/.style={%
			execute at end picture={%
				\tikzifexternalizing{%
					\ifbeamer@anotherslide
					\pgfexternalstorecommand{\string\global\string\beamer@anotherslidetrue}%
					\fi
				}{}%
			}%
		},
		external/optimize=false
	}
	\let\orig@tikzsetnextfilename=\tikzsetnextfilename
	\renewcommand\tikzsetnextfilename[1]{\orig@tikzsetnextfilename{#1-\overlaynumber}}
	\makeatother
	
	\tikzset{every picture/.style={beamer externalizing}}
	
	\tikzexternalize[prefix=external/]
\fi

%%% Page numbering
\usepackage{etoolbox} % necessary for excluding beamer-only frames from page numbering

\makeatletter
\pretocmd{\beamer@@@@frame}{\alt<#1>{}{\beamer@noframenumberingtrue}}{}{}
\makeatother

\addtobeamertemplate{navigation symbols}{}{%
	\usebeamerfont{footline}%
	\usebeamercolor[fg]{footline}%
	\hspace{1em}%
	\scriptsize\insertframenumber/\inserttotalframenumber
}
%%%

% For circular convolution
\newcommand\encircle[1]{%
	\tikz[baseline=(X.base)] 
	\node (X) [anchor=south, draw, shape=circle, inner sep=0mm, align=center] {\scriptsize\strut #1};}

\definecolor{matlabcomment}{RGB}{34,139,34}

\pgfmathdeclarefunction{gauss}{1}{%
	\pgfmathparse{1/(sqrt(2*pi))*exp(-((#1)^2)/2)}%
}

\pgfmathdeclarefunction{sign}{1}{%
	\pgfmathparse{1*(#1 > 0) - 1*(#1 < 0)}%
}

\pgfmathdeclarefunction{laplacian}{2}{%
	\pgfmathparse{1/(#2*2)*exp(-(abs(x-#1))/(#2))}%
}

\pgfmathdeclarefunction{pretty_func}{1}{%
	\pgfmathparse{cos(deg(#1/2)) - sin(deg(#1)) + cos(deg(#1/2)-45) - sin(deg(#1/4)-154)}%
}

\pgfplotsset{
	dirac/.style={
		mark=triangle*,
		mark options={scale=2},
		ycomb,
		scatter,
		visualization depends on={y/abs(y)-1 \as \sign},
		scatter/@pre marker code/.code={\scope[rotate=90*\sign,yshift=-2pt]}
	}
}

\def\thickness{very thick}

\tikzset{
amark/.style 2 args={
	decoration={             
		markings, 
		mark=at position {0.5} with { 
			\arrow{stealth},
			\node[#2] {#1};
		}
	}, \thickness,
	postaction={decorate}
},
earlymark/.style 2 args={
	decoration={             
		markings, 
		mark=at position {0.25} with { 
			\arrow{stealth},
			\node[#2] {#1};
		}
	}, \thickness,
	postaction={decorate}
},
latemark/.style 2 args={
	decoration={             
		markings, 
		mark=at position {0.8} with { 
			\arrow{stealth},
			\node[#2] {#1};
		}
	}, \thickness,
	postaction={decorate}
},
zpath/.style={
	decoration={             
		markings, 
		mark=at position {0.5} with { 
			\arrow{stealth},
			\node[#1] {$z^{-1}$};
		}
	}, \thickness,
	postaction={decorate}
},
terminal/.style 2 args={draw,circle,inner sep=2pt,label={#1:#2}},
}


\tikzset{
	invisible/.style={opacity=0},
	visible on/.style={alt={#1{}{invisible}}},
	alt/.code args={<#1>#2#3}{%
		\alt<#1>{\pgfkeysalso{#2}}{\pgfkeysalso{#3}} % \pgfkeysalso doesn't change the path
	},
}

\newcommand\PlotSampledSpectrum[4]{%
	\def\fs{#2}%
	\def\fmax{#3}%
	\def\ros{#4}%
	\input{#1}%
}

\pgfmathdeclarefunction{invgauss}{2}{%
	\pgfmathparse{sqrt(-2*ln(#1))*cos(deg(2*pi*#2))}%
}

\pgfmathdeclarefunction{triang}{2}{%
	\pgfmathparse{(1/#2)*(#1 + #2)*and(#1 <= 0, #1 >= -#2) + (1/#2)*(-#1 + #2)*and(#1 > 0, #1 <= #2)}%
}

\pgfmathdeclarefunction{rect}{2}{%
	\pgfmathparse{and(#1 >= 0, #1 < #2)}%
}

\pgfmathdeclarefunction{dsinc}{2}{%
	\pgfmathparse{(and(#1 != 0, 1)*(sin(deg(#1*#2/2))/sin(deg(#1/2))) + and(#1 == 0, 1) * (#2)}%
}

\tikzset{
	declare function={
		sinc(\x) = (and(\x!=0, 1) * (sin(deg(pi*\x))/(pi*\x)) +
		(and(\x==0, 1) * 1);
	}
}

\DeclareMathOperator{\E}{\mathbb{E}} % expectation

\newcolumntype{M}[1]{>{\centering\arraybackslash}m{#1}}

\newcommand\SimpleSys[4]{%
	\def\xin{#2}%
	\def\Hz{#3}%
	\def\yout{#4}
	\input{#1}%
}

\newcommand\PlotExp[5]{%
	\def\A{#2}%
	\def\Legend{#3}%
	\def\ymin{#4}
	\def\ymax{#5}
	\input{#1}%
}

\newcommand\PlotSinc[5]{%
	\def\xmin{#2}%
	\def\xmax{#3}%
	\def\samples{#4}
	\def\w{#5}
	\input{#1}%
}

% Gaussian characteristic function
\newcommand\PlotGaussianCF[4]{%
	\def\sig{#2}%
	\def\ws{#3}%
	\def\cap{#4}
	\input{#1}%
}
 % some definitions

%% 
\title{Optical system optimization}
\author{Jose Krause Perin}
\institute{Stanford University}
\date{August 17, 2017}

\begin{document}

\begin{frame}
  \titlepage
\end{frame}


\begin{frame}{EDFA Physics}

\begin{equation} \tag{gain}
G_k = L\bigg((\alpha_k + g_k)\frac{\alpha_p}{\alpha_p + g_p} - \alpha_k\bigg)
\end{equation}

\begin{equation} \tag{ASE PSD}
P_{ASE} = 2n_{sp}(\lambda_k)(G_k-1)h\nu_k
\end{equation}

\begin{equation} \tag{excess noise}
n_{sp}(\lambda_k) = \frac{1}{1 - \displaystyle\frac{g_p\alpha_k}{g_k\alpha_p}}
\end{equation}

\begin{equation} \tag{noise figure}
\mathrm{NF}(\lambda_k) = 2n_{sp}(\lambda_k)\frac{G_k}{G_k - 1}
\end{equation}


\end{frame}

\begin{frame}{Standard confined-doping model}
	\begin{align} \nonumber
		\frac{dP_k}{dz} = \underbrace{u_k(\alpha_k + g_k)\frac{\bar{n}_2}{\bar{n}_t}P_k(z)}_{\text{stimulated emission}} - \underbrace{u_k\alpha_kP_k(z)}_{\text{absorption}} + \underbrace{2u_kg_k\frac{\bar{n}_2}{\bar{n}_t}h\nu_k\Delta\nu_k}_{\text{spontaneous emission}}
	\end{align}
	\begin{equation} \tag{inversion coefficient}
		\frac{\bar{n}_2}{\bar{n}_t} = \displaystyle\frac{\displaystyle\sum_k \displaystyle\frac{P_k(z)\alpha_k}{h\nu_k\xi_k}}{1 + \sum_k \displaystyle\frac{P_k(z)(\alpha_k + g_k)}{h\nu_k\xi_k}}
	\end{equation}
\end{frame}

\begin{frame}{Semi-analytical model}

\begin{equation} \nonumber
Q^{out}_k = Q^{in}_k\exp\Big(\frac{\alpha_k + g_k}{\xi_k}(Q^{in} - Q^{out}) - \alpha_kL_{EDF}\Big)
\end{equation}

\begin{equation} \nonumber
G_k = \exp\Big(\frac{\alpha_k + g_k}{\xi_k}(Q^{in} - Q^{out}) - \alpha_kL_{EDF}\Big)
\end{equation}

\begin{equation} \nonumber
Q^{in}_k = \frac{P_k}{h\nu_k}\qquad Q^{in} = \sum_k Q^{in}_k \qquad Q^{out} = \sum_k Q^{out}_k
\end{equation}

\begin{equation} \nonumber
Q^{out} = \sum_kQ^{in}_k\exp\Big(\frac{\alpha_k + g_k}{\xi_k}(Q^{in} - Q^{out}) - \alpha_kL_{EDF}\Big)
\end{equation}

\end{frame}

%
\begin{frame}
\begin{center}
		\resizebox{\textwidth}{!}{\tikzset{
	ncbar angle/.initial=90,
	ncbar/.style={
		to path=(\tikztostart)
		-- ($(\tikztostart)!#1!\pgfkeysvalueof{/tikz/ncbar angle}:(\tikztotarget)$)
		-- ($(\tikztotarget)!($(\tikztostart)!#1!\pgfkeysvalueof{/tikz/ncbar angle}:(\tikztotarget)$)!\pgfkeysvalueof{/tikz/ncbar angle}:(\tikztostart)$)
		-- (\tikztotarget)
	},
	ncbar/.default=0.5cm,
}

\tikzset{square left brace/.style={ncbar=0.5cm}}
\tikzset{square right brace/.style={ncbar=-0.5cm}}

\tikzset{round left paren/.style={ncbar=0.5cm,out=120,in=-120}}
\tikzset{round right paren/.style={ncbar=0.5cm,out=60,in=-60}}

\begin{tikzpicture}[->, >=stealth, shorten >= 0pt, draw=black!50, node distance=2cm, font=\sffamily, triangle/.style = {regular polygon, regular polygon sides=3 },
node rotated/.style = {rotate=0},
border rotated/.style = {shape border rotate=0}]
    \tikzstyle{node}=[circle,fill=black,minimum size=2pt,inner sep=0pt]
    \tikzstyle{block}=[draw=black,rectangle,fill=none,minimum size=1cm, inner sep=0pt]
    \tikzstyle{amp}=[draw=black,triangle,fill=none,minimum size=1.2cm, inner sep=0pt]
    \tikzstyle{annot} = []

	\node[node] (xc) at (0, 0) {};
	\coordinate[right=1.2cm of xc] (vc) {};
    \node[amp, rotate=-90] (AMP) at (vc)  {};
    
    \node[block, right of=AMP, align=center, text width=2cm, scale=0.8] (f) {\small Gain flattening}; 
    \coordinate[right=2.5cm of f] (yc) {}; 
    \node[node, minimum size=1.5pt] (fin) at ($(f.east)!0.2!(yc)$) {};
    \node[node, minimum size=1.5pt] (fout) at ($(f.east)!0.5!(yc)$) {};
	   
	
	\draw (xc) -- (AMP);
	\draw (AMP) -- (f);
	\draw (f) -- (fin);
	\draw[-] (fin) -- (fout);
	\draw (fout) -- (yc);
	
	\draw[black] ($(f.east)!0.3!(yc) + (0, 0.3cm)$) circle(0.3cm); 
	\draw[black] ($(f.east)!0.35!(yc) + (0, 0.3cm)$) circle(0.3cm);
	\draw[black] ($(f.east)!0.4!(yc) + (0, 0.3cm)$) circle(0.3cm);
	
	\node[below, scale=0.75] at ($(AMP) + (1mm, -5mm)$) {$G(\lambda), n(\lambda)$};
	\node[below, scale=0.75] at ($(f.east)!0.35!(yc)$) {$e^{-\alpha_{\text{SMF}}(\lambda)l}$};
	\node[below, scale=0.75] at (f.south) {$F(\lambda)$};
	
	\draw [-, black, thick] ($(xc)!0.5!(AMP)-(0,1cm)$) to [round left paren ] ($(xc)!0.5!(AMP)+(0,1cm)$);
	\draw [-, black, thick] ($(f)!0.7!(yc)-(0,1cm)$) to [round right paren ] ($(f)!0.7!(yc)+(0,1cm)$);
	\node[right] at ($(f)!0.7!(yc)-(0,1cm)$) {$\times N = \frac{L}{l}$};
	
	\node[above] at (xc) {$P_k$};
	\ifdefined\NONLINEAR
		\node[above] at ($(yc)+(1cm, 0)$) {$\approx P_k + n_{eq, k} + \mathrm{NL}_k$};
	\else
		\node[above] at ($(yc)+(0.5cm, 0)$) {$\approx P_k + n_{eq, k}$};
	\fi
\end{tikzpicture}}
\end{center}

Ideal gain flattening: $\displaystyle F(\lambda) = \displaystyle\frac{1}{G(\lambda)e^{-\alpha_{\text{SMF}}(\lambda)l}}, \quad \text{only if} \quad G(\lambda) \geq e^{\alpha_{\text{SMF}}(\lambda)l}$

\vspace{0.5cm}
The equivalent noise is given by
\begin{equation*}
	n_{eq, k} = Nn(\lambda_k)F(\lambda_k)e^{-\alpha_{\text{SMF}}(\lambda)l} = N\frac{n(\lambda_k)}{G(\lambda_k)}
\end{equation*}

It follows that the capacity of $K$ channels spaced by $\Delta f$ is given by
~
\begin{align} \nonumber
C =\sum_{k = 1}^{K} 2\Delta f\log_2\Bigg(1 + \frac{P_kG(\lambda_k)}{Nn(\lambda_k)}\Bigg)
\end{align}
\end{frame}

\begin{frame}

From the analytical model: $n(\lambda_k) = 2n_{sp}(\lambda_k)(G(\lambda_k)-1)h\nu_k\Delta f$
	
Substituting it in the capacity equation:	
\begin{align} \nonumber
	C =\sum_{k = 1}^{K} 2\Delta f\log_2\Bigg(1 + \frac{P_kG(\lambda_k)}{2Nn_{sp}(\lambda_k)(G(\lambda_k)-1)h\nu_k\Delta f}\Bigg)
\end{align}

Making the approximation: $\displaystyle\frac{G(\lambda_k)}{G(\lambda_k)-1} \approx \frac{e^{\alpha_{\text{SMF}}(\lambda_k)l}}{e^{\alpha_{\text{SMF}}(\lambda_k)l}-1} = a_k$
	
\begin{align} \nonumber
C =\sum_{k = 1}^{K} 2\Delta f\log_2\Bigg(1 + \frac{P_ka_k}{2Nn_{sp}(\lambda_k)h\nu_k\Delta f}\Bigg)
\end{align}

The capacity depends on excess noise $n_{sp}(\lambda_k)$, but not on gain or ASE
~
\begin{align} \tag{theory}
	n_{sp}(\lambda_k) &= \frac{1}{1 - \displaystyle\frac{g_p\alpha_k}{g_k\alpha_p}} \\
	&= \frac{\mathrm{NF}(\lambda_k)}{2a_k} \tag{measurement}
\end{align}
	
\end{frame}

\begin{frame}

Only channels satisfying $G(\lambda_k) \geq e^{\alpha_{\text{SMF}}(\lambda_k)l}$ are useful 
~
\begin{align} \nonumber
C =2\Delta f\sum_{k = 1}^{K} \mathds{1}\{G(\lambda_k) \geq e^{\alpha_{\text{SMF}}(\lambda_k)l}\} \log_2\Bigg(1 + \frac{P_ka_k}{2Nn_{sp}(\lambda_k)h\nu_k\Delta f}\Bigg)
\end{align}

The indicator function $\mathds{1}\{\cdot\}$ is not differentiable. We approximate it by
\begin{equation*}
	\mathds{1}\{x \geq 0\} \approx 0.5(\tanh(Dx) + 1),
\end{equation*}
where $D>0$ controls the sharpness

%\begin{center}
%	\resizebox{0.45\textwidth}{!}{\input{indicator_approx.tex}}
%\end{center}

\end{frame}


\begin{frame}
The optimization problem becomes
~
\begin{align} \nonumber
\max_{P_k, L_{\text{EDF}}}\sum_{k = 1}^{K} &\Big[\tanh\Big(D(G(\lambda_k) - e^{\alpha_{\text{SMF}}(\lambda_k)l})\Big) + 1\Big] \\ \nonumber
&\cdot\log_2\Bigg(1 + \frac{P_k}{2a_kNn_{sp}(\lambda_k)h\nu_k\Delta f}\Bigg)
\end{align}

All parameters are constant, except for $G(\lambda_k)$, which varies with the power $P_k$ and the EDF length $L_{\text{EDF}}$

$G(\lambda_k)$ is calculated from the semi-analytical model

\begin{equation} \nonumber
Q^{in}_k = \frac{P_k}{h\nu_k}, \qquad Q^{out}_k = Q^{in}_k\exp\Big(\frac{\alpha_k + g_k}{\xi_k}(Q^{in} - Q^{out}) - \alpha_kL_{\text{EDF}}\Big)
\end{equation}

\begin{equation} \nonumber
G(\lambda_k) = \exp\Big(\frac{\alpha_k + g_k}{\xi_k}(Q^{in} - Q^{out}) - \alpha_kL_{\text{EDF}}\Big)
\end{equation}

\end{frame}

%%%%%%%%%%%%%% CORNING %%%%%%%%%%%%%%%%%%%%%%%%%%
\begin{frame}
\begin{center}
	\resizebox{\textwidth}{!}{\tikzset{
	ncbar angle/.initial=90,
	ncbar/.style={
		to path=(\tikztostart)
		-- ($(\tikztostart)!#1!\pgfkeysvalueof{/tikz/ncbar angle}:(\tikztotarget)$)
		-- ($(\tikztotarget)!($(\tikztostart)!#1!\pgfkeysvalueof{/tikz/ncbar angle}:(\tikztotarget)$)!\pgfkeysvalueof{/tikz/ncbar angle}:(\tikztostart)$)
		-- (\tikztotarget)
	},
	ncbar/.default=0.5cm,
}

\tikzset{square left brace/.style={ncbar=0.5cm}}
\tikzset{square right brace/.style={ncbar=-0.5cm}}

\tikzset{round left paren/.style={ncbar=0.5cm,out=120,in=-120}}
\tikzset{round right paren/.style={ncbar=0.5cm,out=60,in=-60}}

\begin{tikzpicture}[->, >=stealth, shorten >= 0pt, draw=black!50, node distance=2cm, font=\sffamily, triangle/.style = {regular polygon, regular polygon sides=3 },
node rotated/.style = {rotate=0},
border rotated/.style = {shape border rotate=0}]
    \tikzstyle{node}=[circle,fill=black,minimum size=2pt,inner sep=0pt]
    \tikzstyle{block}=[draw=black,rectangle,fill=none,minimum size=1cm, inner sep=0pt]
    \tikzstyle{amp}=[draw=black,triangle,fill=none,minimum size=1.2cm, inner sep=0pt]
    \tikzstyle{annot} = []

	\node[node] (xc) at (0, 0) {};
	\coordinate[right=1.2cm of xc] (vc) {};
    \node[amp, rotate=-90] (AMP) at (vc)  {};
    
    \node[block, right of=AMP, align=center, text width=2cm, scale=0.8] (f) {\small Gain flattening}; 
    \coordinate[right=2.5cm of f] (yc) {}; 
    \node[node, minimum size=1.5pt] (fin) at ($(f.east)!0.2!(yc)$) {};
    \node[node, minimum size=1.5pt] (fout) at ($(f.east)!0.5!(yc)$) {};
	   
	
	\draw (xc) -- (AMP);
	\draw (AMP) -- (f);
	\draw (f) -- (fin);
	\draw[-] (fin) -- (fout);
	\draw (fout) -- (yc);
	
	\draw[black] ($(f.east)!0.3!(yc) + (0, 0.3cm)$) circle(0.3cm); 
	\draw[black] ($(f.east)!0.35!(yc) + (0, 0.3cm)$) circle(0.3cm);
	\draw[black] ($(f.east)!0.4!(yc) + (0, 0.3cm)$) circle(0.3cm);
	
	\node[below, scale=0.75] at ($(AMP) + (1mm, -5mm)$) {$G(\lambda), n(\lambda)$};
	\node[below, scale=0.75] at ($(f.east)!0.35!(yc)$) {$e^{-\alpha_{\text{SMF}}(\lambda)l}$};
	\node[below, scale=0.75] at (f.south) {$F(\lambda)$};
	
	\draw [-, black, thick] ($(xc)!0.5!(AMP)-(0,1cm)$) to [round left paren ] ($(xc)!0.5!(AMP)+(0,1cm)$);
	\draw [-, black, thick] ($(f)!0.7!(yc)-(0,1cm)$) to [round right paren ] ($(f)!0.7!(yc)+(0,1cm)$);
	\node[right] at ($(f)!0.7!(yc)-(0,1cm)$) {$\times N = \frac{L}{l}$};
	
	\node[above] at (xc) {$P_k$};
	\ifdefined\NONLINEAR
		\node[above] at ($(yc)+(1cm, 0)$) {$\approx P_k + n_{eq, k} + \mathrm{NL}_k$};
	\else
		\node[above] at ($(yc)+(0.5cm, 0)$) {$\approx P_k + n_{eq, k}$};
	\fi
\end{tikzpicture}}
\end{center}

Assumptions and simplifications
\begin{itemize}
	\item Assume ideal gain flattening: 
	\begin{equation*}
		\displaystyle F(\lambda) = \displaystyle\frac{1}{G(\lambda)e^{-\alpha_{\text{SMF}}(\lambda)l}}, \quad \text{only if} \quad G(\lambda) \geq e^{\alpha_{\text{SMF}}(\lambda)l}
	\end{equation*}
	\item Compute gain from semi-analytical model
	\item Compute ASE power from analytical model 
	\begin{equation*}
	n(\lambda_k) = 2n_{sp}(\lambda_k)(G(\lambda_k)-1)h\nu_k\Delta f
	\end{equation*}
	excess noise $n_{sp}(\lambda_k)$ can be obtained from theory or noise figure measurements
\end{itemize}

\end{frame}

\begin{frame}

After some algebra, we obtain an equation for the capacity

\begin{align} \nonumber
C =\sum_{k = 1}^{K} 2\Delta f\log_2\Bigg(1 + \frac{P_k}{2a_kNn_{sp}(\lambda_k)h\nu_k\Delta f}\Bigg)
\end{align}

\end{frame}

\begin{frame}

Only channels satisfying $G(\lambda_k) \geq e^{\alpha_{\text{SMF}}(\lambda_k)l}$ are useful 
~
\begin{align} \nonumber
C =2\Delta f\sum_{k = 1}^{K} \mathds{1}\{G(\lambda_k) \geq e^{\alpha_{\text{SMF}}(\lambda_k)l}\} \log_2\Bigg(1 + \frac{P_k}{2a_kNn_{sp}(\lambda_k)h\nu_k\Delta f}\Bigg)
\end{align}

The indicator function $\mathds{1}\{\cdot\}$ is not differentiable. We approximate it by
\begin{equation*}
\mathds{1}\{x \geq 0\} \approx 0.5(\tanh(Dx) + 1),
\end{equation*}
where $D>0$ controls the sharpness

\end{frame}


\begin{frame}
The optimization problem becomes
~
\begin{align} \nonumber
\max_{P_k, L_{\text{EDF}}}\sum_{k = 1}^{K} &\Big[\tanh\Big(D(G(\lambda_k) - e^{\alpha_{\text{SMF}}(\lambda_k)l})\Big) + 1\Big] \\ \nonumber
&\cdot\log_2\Bigg(1 + \frac{P_k}{2a_kNn_{sp}(\lambda_k)h\nu_k\Delta f}\Bigg)
\end{align}

\begin{itemize}
	\item All parameters are constant, except for $G(\lambda_k)$
	\item $G(\lambda_k)$ is calculated from the semi-analytical model
\end{itemize}

\begin{equation} \nonumber
Q^{in}_k = \frac{P_k}{h\nu_k}, \qquad Q^{out}_k = Q^{in}_k\exp\Big(\frac{\alpha_k + g_k}{\xi_k}(Q^{in} - Q^{out}) - \alpha_kL_{\text{EDF}}\Big)
\end{equation}

\begin{equation} \nonumber
G(\lambda_k) = \exp\Big(\frac{\alpha_k + g_k}{\xi_k}(Q^{in} - Q^{out}) - \alpha_kL_{\text{EDF}}\Big)
\end{equation}

\end{frame}

\begin{frame}{Nonlinear noise}
	\begin{equation*}
	\mathrm{NL}_n = \sum_{n_1 = 1}^{N}\sum_{n_1 = 1}^N\sum_{q=-1}^{1}P_{n_1}P_{n_2}P_{i+j-n+l}D^{(q)}_{n_1-n, n_2-n}, \quad n = 1, \ldots, N
	\end{equation*}
	
	\begin{align*} \label{eq:Dcoeff-original}\nonumber 
	D^{(q)}_{n_1-n, n_2-n} &= \gamma^2\frac{16}{27}\iiint_{-\Delta f/2}^{\Delta f/2} \\ \nonumber
	&\cdot\rho(x + n_1\Delta f, y + n_2\Delta f, z + n\Delta f) \\ \nonumber
	& \cdot\chi(x + n_1\Delta f, y + n_2\Delta f, z + n\Delta f) \\
	&\cdot g(x)g(y)g(z)g(x+y-z+q\Delta f)\partial x\partial y\partial z,
	\end{align*}
	\begin{align*}
	\rho(f_1, f_2, f) &= \Bigg|\frac{1 - \exp\Big(-\alpha l + j4\pi^2\beta_2l(f_1-f)(f_2-f)\Big)}{\alpha - j4\pi^2\beta_2(f_1-f)(f_2-f)}\Bigg|^2 \\
	\chi(f_1, f_2, f) &= \frac{\sin^2(N_s\theta)}{\sin^2(\theta)}, \quad \theta = 2\pi^2(f_1-f)(f_2-f)\beta_2L_s
	\end{align*}
	
	\begin{equation*}
		\mathrm{NL}_n\Big|_{N_{spans}} = N_{spans}^{1+\epsilon}\cdot\mathrm{NL}_n\Big|_{N_{spans} = 1}
	\end{equation*}
	
\end{frame}

\begin{frame}{Nonlinear noise}

\begin{center}
	\def\NONLINEAR{1}
	\resizebox{\textwidth}{!}{\tikzset{
	ncbar angle/.initial=90,
	ncbar/.style={
		to path=(\tikztostart)
		-- ($(\tikztostart)!#1!\pgfkeysvalueof{/tikz/ncbar angle}:(\tikztotarget)$)
		-- ($(\tikztotarget)!($(\tikztostart)!#1!\pgfkeysvalueof{/tikz/ncbar angle}:(\tikztotarget)$)!\pgfkeysvalueof{/tikz/ncbar angle}:(\tikztostart)$)
		-- (\tikztotarget)
	},
	ncbar/.default=0.5cm,
}

\tikzset{square left brace/.style={ncbar=0.5cm}}
\tikzset{square right brace/.style={ncbar=-0.5cm}}

\tikzset{round left paren/.style={ncbar=0.5cm,out=120,in=-120}}
\tikzset{round right paren/.style={ncbar=0.5cm,out=60,in=-60}}

\begin{tikzpicture}[->, >=stealth, shorten >= 0pt, draw=black!50, node distance=2cm, font=\sffamily, triangle/.style = {regular polygon, regular polygon sides=3 },
node rotated/.style = {rotate=0},
border rotated/.style = {shape border rotate=0}]
    \tikzstyle{node}=[circle,fill=black,minimum size=2pt,inner sep=0pt]
    \tikzstyle{block}=[draw=black,rectangle,fill=none,minimum size=1cm, inner sep=0pt]
    \tikzstyle{amp}=[draw=black,triangle,fill=none,minimum size=1.2cm, inner sep=0pt]
    \tikzstyle{annot} = []

	\node[node] (xc) at (0, 0) {};
	\coordinate[right=1.2cm of xc] (vc) {};
    \node[amp, rotate=-90] (AMP) at (vc)  {};
    
    \node[block, right of=AMP, align=center, text width=2cm, scale=0.8] (f) {\small Gain flattening}; 
    \coordinate[right=2.5cm of f] (yc) {}; 
    \node[node, minimum size=1.5pt] (fin) at ($(f.east)!0.2!(yc)$) {};
    \node[node, minimum size=1.5pt] (fout) at ($(f.east)!0.5!(yc)$) {};
	   
	
	\draw (xc) -- (AMP);
	\draw (AMP) -- (f);
	\draw (f) -- (fin);
	\draw[-] (fin) -- (fout);
	\draw (fout) -- (yc);
	
	\draw[black] ($(f.east)!0.3!(yc) + (0, 0.3cm)$) circle(0.3cm); 
	\draw[black] ($(f.east)!0.35!(yc) + (0, 0.3cm)$) circle(0.3cm);
	\draw[black] ($(f.east)!0.4!(yc) + (0, 0.3cm)$) circle(0.3cm);
	
	\node[below, scale=0.75] at ($(AMP) + (1mm, -5mm)$) {$G(\lambda), n(\lambda)$};
	\node[below, scale=0.75] at ($(f.east)!0.35!(yc)$) {$e^{-\alpha_{\text{SMF}}(\lambda)l}$};
	\node[below, scale=0.75] at (f.south) {$F(\lambda)$};
	
	\draw [-, black, thick] ($(xc)!0.5!(AMP)-(0,1cm)$) to [round left paren ] ($(xc)!0.5!(AMP)+(0,1cm)$);
	\draw [-, black, thick] ($(f)!0.7!(yc)-(0,1cm)$) to [round right paren ] ($(f)!0.7!(yc)+(0,1cm)$);
	\node[right] at ($(f)!0.7!(yc)-(0,1cm)$) {$\times N = \frac{L}{l}$};
	
	\node[above] at (xc) {$P_k$};
	\ifdefined\NONLINEAR
		\node[above] at ($(yc)+(1cm, 0)$) {$\approx P_k + n_{eq, k} + \mathrm{NL}_k$};
	\else
		\node[above] at ($(yc)+(0.5cm, 0)$) {$\approx P_k + n_{eq, k}$};
	\fi
\end{tikzpicture}}
\end{center}
\begin{equation*}
\mathrm{NL}_n = \sum_{n_1 = 1}^{N}\sum_{n_1 = 1}^N\sum_{l=-1}^{1}P_{n_1}P_{n_2}P_{i+j-n+l}D_l(n_1, n_2, n), \quad n = 1, \ldots, N
\end{equation*}

\end{frame}

\begin{frame}
	\begin{equation*}
		\max_{P_n, L_{\text{EDF}}} C(P_n, L_{\text{EDF}}; P_p, \{\ldots\})
	\end{equation*}
	where
	
	\begin{align*} 
	C(P_n, L_{\text{EDF}}; P_p, \{\ldots\}) = 2\Delta f\sum_{n=1}^N &\mathds{1}\{\mathcal{G}(\lambda_n) \geq \alpha_{\text{SMF}}l\}\log_2\Big(1 + \frac{P_n}{P_{\text{ASE}, n} + \mathrm{NL}_n}\Big) 
	\end{align*}

	Indicator function is approximated by a differentiable sigmoidal function
	\begin{equation*}
	\mathds{1}\{x \geq 0\} \approx 0.5(\tanh(Dx) + 1),
	\end{equation*}
	
	Need to compute amplifier gain $\mathcal{G}(\lambda_n)$, amplifier noise $P_{\text{ASE}, n}$, and nonlinear noise $\mathrm{NL}_n$
\end{frame}

\begin{frame}

	
	
	\begin{equation} \label{eq:semi_analytical_gain}
	G_k = \exp\Big(\frac{\alpha_k + g_k}{\xi_k}(Q^{in} - Q^{out}) - \alpha_kL_{\text{EDF}}\Big)
	\end{equation}
	\begin{equation} \label{eq:out-photon-flux}
	Q^{out} = \sum_kQ^{in}_k\exp\Big(\frac{\alpha_k + g_k}{\zeta_k}(Q^{in} - Q^{out}) - \alpha_kL_{\text{EDF}}\Big)
	\end{equation}

	\begin{equation}
	P_{\text{ASE}, n} = \mathrm{NF}_nh\nu_n\Delta f
	\end{equation}
	\begin{equation}
	\mathrm{NF}_n = 2n_{sp, n}\frac{G_n-1}{G_n}
	\end{equation}
\end{frame}

\begin{frame}
	\begin{columns}
		\begin{column}{0.33\textwidth}
			\textbf{Standard confined doping (SCD)}
			\begin{itemize}
				\item Boundary value problem 
				\item Used as reference
			\end{itemize}
			\begin{align*} \label{eq:scd} \nonumber
			\frac{d}{dz}P_k(z) &= u_k(\alpha_k + g^*_k)\frac{\bar{n}_2}{\bar{n}_t}P_k(z) \\
			&- u_k\alpha_kP_k(z) + 2u_kg^*_k\frac{\bar{n}_2}{\bar{n}_t}h\nu_k\Delta f
			\end{align*}
			\begin{equation*}
			\frac{\bar{n}_2}{\bar{n}_t} = \displaystyle\frac{\displaystyle\sum_k \displaystyle\frac{P_k(z)\alpha_k}{h\nu_k\zeta_k}}{1 + \displaystyle\sum_k \displaystyle\frac{P_k(z)(\alpha_k + g^*_k)}{h\nu_k\zeta_k}}
			\end{equation*}
		\end{column}
			\begin{column}{0.33\textwidth}
			\textbf{Semi-analytical model}
			\begin{itemize}
				\item Particular case of SCD model, where noise is disregarded
				\item Gain is given by an implicit equation
				\item Used in optimization to compute gain
			\end{itemize}
			\begin{align*} \label{eq:scd} \nonumber
			\frac{d}{dz}P_k(z) &= u_k(\alpha_k + g^*_k)\frac{\bar{n}_2}{\bar{n}_t}P_k(z) \\
			&- u_k\alpha_kP_k(z) + 2u_kg^*_k\frac{\bar{n}_2}{\bar{n}_t}h\nu_k\Delta f
			\end{align*}
			\begin{equation*}
			\frac{\bar{n}_2}{\bar{n}_t} = \displaystyle\frac{\displaystyle\sum_k \displaystyle\frac{P_k(z)\alpha_k}{h\nu_k\zeta_k}}{1 + \displaystyle\sum_k \displaystyle\frac{P_k(z)(\alpha_k + g^*_k)}{h\nu_k\zeta_k}}
			\end{equation*}
		\end{column}
	\end{columns}
\end{frame}

\end{document}
